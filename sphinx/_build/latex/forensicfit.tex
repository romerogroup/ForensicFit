%% Generated by Sphinx.
\def\sphinxdocclass{report}
\documentclass[letterpaper,10pt,english]{sphinxmanual}
\ifdefined\pdfpxdimen
   \let\sphinxpxdimen\pdfpxdimen\else\newdimen\sphinxpxdimen
\fi \sphinxpxdimen=.75bp\relax
\ifdefined\pdfimageresolution
    \pdfimageresolution= \numexpr \dimexpr1in\relax/\sphinxpxdimen\relax
\fi
%% let collapsible pdf bookmarks panel have high depth per default
\PassOptionsToPackage{bookmarksdepth=5}{hyperref}

\PassOptionsToPackage{booktabs}{sphinx}
\PassOptionsToPackage{colorrows}{sphinx}

\PassOptionsToPackage{warn}{textcomp}
\usepackage[utf8]{inputenc}
\ifdefined\DeclareUnicodeCharacter
% support both utf8 and utf8x syntaxes
  \ifdefined\DeclareUnicodeCharacterAsOptional
    \def\sphinxDUC#1{\DeclareUnicodeCharacter{"#1}}
  \else
    \let\sphinxDUC\DeclareUnicodeCharacter
  \fi
  \sphinxDUC{00A0}{\nobreakspace}
  \sphinxDUC{2500}{\sphinxunichar{2500}}
  \sphinxDUC{2502}{\sphinxunichar{2502}}
  \sphinxDUC{2514}{\sphinxunichar{2514}}
  \sphinxDUC{251C}{\sphinxunichar{251C}}
  \sphinxDUC{2572}{\textbackslash}
\fi
\usepackage{cmap}
\usepackage[T1]{fontenc}
\usepackage{amsmath,amssymb,amstext}
\usepackage{babel}



\usepackage{tgtermes}
\usepackage{tgheros}
\renewcommand{\ttdefault}{txtt}



\usepackage[Bjarne]{fncychap}
\usepackage{sphinx}

\fvset{fontsize=auto}
\usepackage{geometry}


% Include hyperref last.
\usepackage{hyperref}
% Fix anchor placement for figures with captions.
\usepackage{hypcap}% it must be loaded after hyperref.
% Set up styles of URL: it should be placed after hyperref.
\urlstyle{same}

\addto\captionsenglish{\renewcommand{\contentsname}{Contents:}}

\usepackage{sphinxmessages}
\setcounter{tocdepth}{3}
\setcounter{secnumdepth}{3}


\title{Forensic Fit}
\date{Jul 13, 2023}
\release{}
\author{Pedram Tavadze}
\newcommand{\sphinxlogo}{\vbox{}}
\renewcommand{\releasename}{}
\makeindex
\begin{document}

\ifdefined\shorthandoff
  \ifnum\catcode`\=\string=\active\shorthandoff{=}\fi
  \ifnum\catcode`\"=\active\shorthandoff{"}\fi
\fi

\pagestyle{empty}
\sphinxmaketitle
\pagestyle{plain}
\sphinxtableofcontents
\pagestyle{normal}
\phantomsection\label{\detokenize{index::doc}}


\sphinxAtStartPar
ForensicFit is a Python package designed to preprocess scanned images from various sources and generate a database to be used in different machine learning approaches. This package prepares the data using four distinct techniques, which will be explained in the tutorial sections. ForensicFit leverages state\sphinxhyphen{}of\sphinxhyphen{}the\sphinxhyphen{}art image processing methods to analyze and store the generated data, ensuring compatibility with popular machine learning packages such as TensorFlow, PyTorch, and SciKit\sphinxhyphen{}learn. It utilizes NumPy, SciPy, matplotlib, OpenCV, scikit\sphinxhyphen{}image, PyMongo, and GridFS. For ease of use and future development, the package adheres to PEP\sphinxhyphen{}257 and PEP\sphinxhyphen{}484 for documentation and type hints, respectively.


\chapter{Package Structure}
\label{\detokenize{index:package-structure}}
\sphinxAtStartPar
ForensicFit is organized into three main sub\sphinxhyphen{}packages: \sphinxcode{\sphinxupquote{core}}, \sphinxcode{\sphinxupquote{database}}, and \sphinxcode{\sphinxupquote{utils}}.
\begin{itemize}
\item {} 
\sphinxAtStartPar
\sphinxcode{\sphinxupquote{core}}: This sub\sphinxhyphen{}package contains the essential functionalities of ForensicFit, including Python classes that manage read/write, analysis, and metadata storage. These classes provide a data structure skeleton for the package and define standards for future implementations related to different types of materials.

\item {} 
\sphinxAtStartPar
\sphinxcode{\sphinxupquote{database}}: This sub\sphinxhyphen{}package offers an efficient and flexible method for storing and retrieving raw and preprocessed data. Although the rest of the package does not depend on this sub\sphinxhyphen{}package, it has been included to simplify the data storage and query process. Users can still store and access raw or analyzed data using traditional image storage methods.

\item {} 
\sphinxAtStartPar
\sphinxcode{\sphinxupquote{utils}}: The \sphinxcode{\sphinxupquote{utils}} sub\sphinxhyphen{}package contains various image manipulation, plotting, and memory access tools used throughout the package.

\end{itemize}

\noindent{\hspace*{\fill}\sphinxincludegraphics{{ForensicFit_tree}.svg}\hspace*{\fill}}


\chapter{Installation}
\label{\detokenize{index:installation}}
\sphinxAtStartPar
To install ForensicFit, use the following command:

\begin{sphinxVerbatim}[commandchars=\\\{\}]
pip\PYG{+w}{ }install\PYG{+w}{ }forensicfit
\end{sphinxVerbatim}


\chapter{Quick Start}
\label{\detokenize{index:quick-start}}
\sphinxAtStartPar
Here’s a quick example of how to use ForensicFit:

\begin{sphinxVerbatim}[commandchars=\\\{\}]
\PYG{g+gp}{\PYGZgt{}\PYGZgt{}\PYGZgt{} }\PYG{k+kn}{import} \PYG{n+nn}{forensicfit} \PYG{k}{as} \PYG{n+nn}{ff}

\PYG{g+go}{\PYGZsh{} Load your image file}
\PYG{g+gp}{\PYGZgt{}\PYGZgt{}\PYGZgt{} }\PYG{n}{path} \PYG{o}{=} \PYG{l+s+s1}{\PYGZsq{}}\PYG{l+s+s1}{path/to/LQ\PYGZhy{}HT\PYGZhy{}1.jpg}\PYG{l+s+s1}{\PYGZsq{}}
\PYG{g+gp}{\PYGZgt{}\PYGZgt{}\PYGZgt{} }\PYG{n}{tape} \PYG{o}{=} \PYG{n}{ff}\PYG{o}{.}\PYG{n}{core}\PYG{o}{.}\PYG{n}{Tape}\PYG{o}{.}\PYG{n}{from\PYGZus{}file}\PYG{p}{(}\PYG{n}{path}\PYG{p}{)}
\PYG{g+gp}{\PYGZgt{}\PYGZgt{}\PYGZgt{} }\PYG{n+nb}{print}\PYG{p}{(}\PYG{n}{tape}\PYG{p}{)}
\PYG{g+go}{Mode: material}
\PYG{g+go}{Resolution: (2471, 6289, 3)}
\PYG{g+go}{Path: path/to/LQ\PYGZhy{}HT\PYGZhy{}1.jpg/LQ\PYGZus{}099.tif}
\PYG{g+go}{Filename: LQ\PYGZus{}099.tif}
\PYG{g+go}{Compression: raw}
\PYG{g+go}{DPI: (1200.0, 1200.0)}
\PYG{g+go}{Flip horizontal: False}
\PYG{g+go}{Flip vertical: False}
\PYG{g+go}{Split vertical: \PYGZob{}\PYGZsq{}side\PYGZsq{}: None, \PYGZsq{}pixel\PYGZus{}index\PYGZsq{}: None\PYGZcb{}}
\PYG{g+go}{Label: None}
\PYG{g+go}{Material: tape}
\PYG{g+go}{Surface: None}
\PYG{g+go}{Stretched: False}
\end{sphinxVerbatim}

\noindent{\hspace*{\fill}\sphinxincludegraphics{{LQ_099-raw}.png}\hspace*{\fill}}

\sphinxstepscope


\section{Installation}
\label{\detokenize{installation:installation}}\label{\detokenize{installation::doc}}

\subsection{Installing using pip}
\label{\detokenize{installation:installing-using-pip}}
\sphinxAtStartPar
Not implemented yet


\subsection{Installing using githutb}
\label{\detokenize{installation:installing-using-githutb}}
\sphinxAtStartPar
The package is hosted on a github repository. In linux one can use
the following command to clone the package:

\begin{sphinxVerbatim}[commandchars=\\\{\}]
\PYG{n}{git} \PYG{n}{clone} \PYG{n}{https}\PYG{p}{:}\PYG{o}{/}\PYG{o}{/}\PYG{n}{github}\PYG{o}{.}\PYG{n}{com}\PYG{o}{/}\PYG{n}{romerogroup}\PYG{o}{/}\PYG{n}{ForensicFit}\PYG{o}{.}\PYG{n}{git}
\end{sphinxVerbatim}

\sphinxAtStartPar
To download this package from github in windows one either install the github
desktop app or directly download it from:

\begin{sphinxVerbatim}[commandchars=\\\{\}]
\PYG{n}{https}\PYG{p}{:}\PYG{o}{/}\PYG{o}{/}\PYG{n}{github}\PYG{o}{.}\PYG{n}{com}\PYG{o}{/}\PYG{n}{romerogroup}\PYG{o}{/}\PYG{n}{ForensicFit}\PYG{o}{.}\PYG{n}{git}
\end{sphinxVerbatim}


\subsection{Dependencies}
\label{\detokenize{installation:dependencies}}
\sphinxAtStartPar
This package relies on the following python packages:

\begin{sphinxVerbatim}[commandchars=\\\{\}]
\PYG{n}{matplotlib}
\PYG{n}{numpy}
\PYG{n}{scipy}
\PYG{n}{opencv}
\PYG{n}{pymongo}
\PYG{n}{scikit}\PYG{o}{\PYGZhy{}}\PYG{n}{image}
\end{sphinxVerbatim}

\sphinxstepscope


\section{Developers}
\label{\detokenize{developers:developers}}\label{\detokenize{developers::doc}}
\sphinxAtStartPar
Pedram Tavadze

\sphinxAtStartPar
Meghan Prusinowski

\sphinxAtStartPar
Freddie Farah

\sphinxAtStartPar
Zachary Andrews

\sphinxAtStartPar
Tatiana Trejos

\sphinxAtStartPar
Aldo H Romero

\sphinxstepscope


\section{Tutorials}
\label{\detokenize{tutorials:tutorials}}\label{\detokenize{tutorials::doc}}
\sphinxAtStartPar
The following set of tutorials demonstrates the basic usage of the ForensicFit package. The examples illustrate how to apply the preprocessing method for various approaches.


\subsection{Preprocessing}
\label{\detokenize{tutorials:preprocessing}}
\sphinxAtStartPar
Initially, we will describe the steps required to process a single image. After that, we will proceed to discuss the processing of directories containing multiple images.
Each grayscale image consists of a grid of pixels, where each pixel holds a value ranging from 0 to 255, representing the specific gray shade of the pixel.


\subsubsection{1. Single image processing}
\label{\detokenize{tutorials:single-image-processing}}

\paragraph{1.1. Loading the Image and Displaying It}
\label{\detokenize{tutorials:loading-the-image-and-displaying-it}}
\sphinxAtStartPar
Given the path to the image, ForensicFit will create a \sphinxcode{\sphinxupquote{Tape}} object.
\sphinxcode{\sphinxupquote{Tape}} is instantiated by providing a numpy array. Alternatively, one can use the \sphinxcode{\sphinxupquote{from\_file}} method to load the image from a file.

\begin{sphinxVerbatim}[commandchars=\\\{\}]
\PYG{g+gp}{\PYGZgt{}\PYGZgt{}\PYGZgt{} }\PYG{k+kn}{import} \PYG{n+nn}{forensicfit} \PYG{k}{as} \PYG{n+nn}{ff}
\PYG{g+gp}{\PYGZgt{}\PYGZgt{}\PYGZgt{} }\PYG{n}{tape} \PYG{o}{=} \PYG{n}{ff}\PYG{o}{.}\PYG{n}{core}\PYG{o}{.}\PYG{n}{Tape}\PYG{o}{.}\PYG{n}{from\PYGZus{}file}\PYG{p}{(}\PYG{n}{fname}\PYG{o}{=}\PYG{l+s+s1}{\PYGZsq{}}\PYG{l+s+s1}{L001.tiff}\PYG{l+s+s1}{\PYGZsq{}}\PYG{p}{)}
\end{sphinxVerbatim}

\sphinxAtStartPar
\sphinxcode{\sphinxupquote{tape}} is a \sphinxcode{\sphinxupquote{Tape}} object with various properties and methods. In this section,
we will explain the usage of each method and property. \sphinxcode{\sphinxupquote{Tape}} inherits from the class \sphinxcode{\sphinxupquote{Image}}.

\sphinxAtStartPar
Before discussing each method, let’s list the input parameters of this class.
A comprehensive description of each parameter is provided in the \sphinxhref{forensicfit.core.html\#module-forensicfit.core.tape}{API package} section. To display the image, we can use the \sphinxcode{\sphinxupquote{show}} or the \sphinxcode{\sphinxupquote{plot}} method.
We can save the image by including the argument \sphinxcode{\sphinxupquote{savefig}} when plotting.

\begin{sphinxVerbatim}[commandchars=\\\{\}]
\PYG{g+gp}{\PYGZgt{}\PYGZgt{}\PYGZgt{} }\PYG{n}{tape} \PYG{o}{=} \PYG{n}{ff}\PYG{o}{.}\PYG{n}{core}\PYG{o}{.}\PYG{n}{Tape}\PYG{o}{.}\PYG{n}{from\PYGZus{}file}\PYG{p}{(}\PYG{n}{fname}\PYG{o}{=}\PYG{l+s+s1}{\PYGZsq{}}\PYG{l+s+s1}{L001.tiff}\PYG{l+s+s1}{\PYGZsq{}}\PYG{p}{)}
\PYG{g+gp}{\PYGZgt{}\PYGZgt{}\PYGZgt{} }\PYG{n}{tape}\PYG{o}{.}\PYG{n}{plot}\PYG{p}{(}\PYG{p}{)}
\PYG{g+gp}{\PYGZgt{}\PYGZgt{}\PYGZgt{} }\PYG{n}{tape}\PYG{o}{.}\PYG{n}{plot}\PYG{p}{(}\PYG{n}{savefig}\PYG{o}{=}\PYG{l+s+s1}{\PYGZsq{}}\PYG{l+s+s1}{L001.png}\PYG{l+s+s1}{\PYGZsq{}}\PYG{p}{)}
\end{sphinxVerbatim}

\noindent{\hspace*{\fill}\sphinxincludegraphics{{1.show}.png}\hspace*{\fill}}

\sphinxAtStartPar
\sphinxcode{\sphinxupquote{gaussian\_blur}} is an important filter applied at the beginning of this class.
This filter is used to reduce image noise and facilitate boundary detection.
The optimal value typically depends on the amount of noise the scanner introduces to the image.
This parameter defines the window of pixels where the filter is applied. The default value for this parameter is \sphinxcode{\sphinxupquote{gaussian\_blur=(15,15)}}.
To illustrate the effects of this filter, we choose a large window to exaggerate the impact. The window must always be constructed using odd numbers.
Usage:

\begin{sphinxVerbatim}[commandchars=\\\{\}]
\PYG{n}{tape\PYGZus{}image} \PYG{o}{=} \PYG{n}{forensicfit}\PYG{o}{.}\PYG{n}{preprocess}\PYG{o}{.}\PYG{n}{TapeImage}\PYG{p}{(}\PYG{l+s+s1}{\PYGZsq{}}\PYG{l+s+s1}{LQ\PYGZus{}775.tif}\PYG{l+s+s1}{\PYGZsq{}}\PYG{p}{,}
                            \PYG{n}{gaussian\PYGZus{}blur}\PYG{o}{=}\PYG{p}{(}\PYG{l+m+mi}{101}\PYG{p}{,}\PYG{l+m+mi}{101}\PYG{p}{)}\PYG{p}{)}
\end{sphinxVerbatim}

\noindent{\hspace*{\fill}\sphinxincludegraphics{{1.gaussian_blur}.png}\hspace*{\fill}}


\paragraph{1.2. Splitting the image vertically}
\label{\detokenize{tutorials:splitting-the-image-vertically}}
\sphinxAtStartPar
Sometimes one does not need one side of the image. To address this issue \sphinxcode{\sphinxupquote{split}}
parameter is implemented.One has to turn on the split paramter by \sphinxcode{\sphinxupquote{split=True}},
Then select the side of the image that is important to us (\sphinxcode{\sphinxupquote{\textquotesingle{}L\textquotesingle{}}} for left or \sphinxcode{\sphinxupquote{\textquotesingle{}R\textquotesingle{}}} for right)
by \sphinxcode{\sphinxupquote{split\_side=\textquotesingle{}L\textquotesingle{}}} (important: do not forget that this parameter has to a python string), and
finally one has to choose the \sphinxcode{\sphinxupquote{split\_position}}. split\_position can only be a number between 0
and 1. for example if \sphinxcode{\sphinxupquote{split\_position=0.5}}, the image will be divided in the half way line. The defaults are
\sphinxcode{\sphinxupquote{split=False}}, \sphinxcode{\sphinxupquote{split\_position=0.5}} and \sphinxcode{\sphinxupquote{split\_side=\textquotesingle{}L\textquotesingle{}}}.
Let’s also view the split image by using the \sphinxcode{\sphinxupquote{show()}} method.

\sphinxAtStartPar
Usage:

\begin{sphinxVerbatim}[commandchars=\\\{\}]
\PYG{n}{tape\PYGZus{}image} \PYG{o}{=} \PYG{n}{forensicfir}\PYG{o}{.}\PYG{n}{TapeImage}\PYG{p}{(}\PYG{n}{fname}\PYG{o}{=}\PYG{l+s+s1}{\PYGZsq{}}\PYG{l+s+s1}{LQ\PYGZus{}775.tiff}\PYG{l+s+s1}{\PYGZsq{}}\PYG{p}{,}
                                    \PYG{n}{split}\PYG{o}{=}\PYG{k+kc}{True}\PYG{p}{,}
                                    \PYG{n}{split\PYGZus{}side}\PYG{o}{=}\PYG{l+s+s1}{\PYGZsq{}}\PYG{l+s+s1}{L}\PYG{l+s+s1}{\PYGZsq{}}\PYG{p}{,}
                                    \PYG{n}{split\PYGZus{}position}\PYG{o}{=}\PYG{l+m+mf}{0.5}\PYG{p}{)}
\PYG{n}{tape\PYGZus{}image}\PYG{o}{.}\PYG{n}{show}\PYG{p}{(}\PYG{n}{cmap}\PYG{o}{=}\PYG{l+s+s1}{\PYGZsq{}}\PYG{l+s+s1}{gray}\PYG{l+s+s1}{\PYGZsq{}}\PYG{p}{)}
\end{sphinxVerbatim}

\noindent{\hspace*{\fill}\sphinxincludegraphics{{2.split_L}.png}\hspace*{\fill}}


\paragraph{1.3. Finding the tilt of the image}
\label{\detokenize{tutorials:finding-the-tilt-of-the-image}}
\sphinxAtStartPar
During the scanning process one might not position the image exactly parallel to the
horizontal line. The property of \sphinxcode{\sphinxupquote{image\_tilt}} can calculate the tilt of the tape. To calculate
this number the algorithm finds the boundaries at the top and the bottom of the image.
Then each line is divided in 6 segments(our experience showed that 6 is segments usually works good).
The first and the last segments are discarted to avoid noise close to the edge. The the slop and standard
deviation(in y direction) is calculated with a linear fit. The two
segments with the least are selected from the top and bottom. Furthur more the average angle
by each line is reported in degrees.

\sphinxAtStartPar
Usage:

\begin{sphinxVerbatim}[commandchars=\\\{\}]
\PYG{n+nb}{print}\PYG{p}{(}\PYG{n}{tape\PYGZus{}image}\PYG{o}{.}\PYG{n}{image\PYGZus{}tilt}\PYG{p}{)}
\PYG{l+m+mf}{0.12844069008595374}
\end{sphinxVerbatim}

\sphinxAtStartPar
If one wants to monitor all of the steps of the selection of the best segment for angle calculation,
One can use the \sphinxcode{\sphinxupquote{get\_image\_tilt(plot=True)}}, with the plot parameter turn on(True).

\sphinxAtStartPar
Usage:

\begin{sphinxVerbatim}[commandchars=\\\{\}]
\PYG{n}{tape\PYGZus{}image}\PYG{o}{.}\PYG{n}{get\PYGZus{}image\PYGZus{}tilt}\PYG{p}{(}\PYG{n}{plot}\PYG{o}{=}\PYG{k+kc}{True}\PYG{p}{)}
\end{sphinxVerbatim}

\sphinxAtStartPar
This method will produce two plots one with all of the segments shown in diferent colors,

\noindent{\hspace*{\fill}\sphinxincludegraphics{{3.tilt_1_all}.png}\hspace*{\fill}}

\sphinxAtStartPar
and one with the two segments with the least standard deviation, plotted over the detected boundary.

\noindent{\hspace*{\fill}\sphinxincludegraphics{{3.tilt_2_best}.png}\hspace*{\fill}}


\paragraph{1.4. Plot Boundaries}
\label{\detokenize{tutorials:plot-boundaries}}
\sphinxAtStartPar
This class automatically(using opencv) detects the boundaries. To plot this boundary
one has to use, \sphinxcode{\sphinxupquote{plot\_boundary(color=\textquotesingle{}red\textquotesingle{})}}. As it’s self explanatory the color
parameter changes the color of the boundary. To plot this boundary one has to plot
the image first then use the plot boundary similar to the following example.

\sphinxAtStartPar
Usage:

\begin{sphinxVerbatim}[commandchars=\\\{\}]
\PYG{n}{tape\PYGZus{}image}\PYG{o}{.}\PYG{n}{show}\PYG{p}{(}\PYG{n}{cmap}\PYG{o}{=}\PYG{l+s+s1}{\PYGZsq{}}\PYG{l+s+s1}{gray}\PYG{l+s+s1}{\PYGZsq{}}\PYG{p}{)}
\PYG{n}{tape\PYGZus{}image}\PYG{o}{.}\PYG{n}{plot\PYGZus{}boundary}\PYG{p}{(}\PYG{n}{color}\PYG{o}{=}\PYG{l+s+s1}{\PYGZsq{}}\PYG{l+s+s1}{red}\PYG{l+s+s1}{\PYGZsq{}}\PYG{p}{)}
\end{sphinxVerbatim}


\paragraph{1.5. Auto Crop in Y direction}
\label{\detokenize{tutorials:auto-crop-in-y-direction}}
\sphinxAtStartPar
\sphinxcode{\sphinxupquote{auto\_crop\_y()}} will automatically crop the image based on the boundaries that it found.
To make this example more interesting we add a plot boundary function as well.

\sphinxAtStartPar
Usage:

\begin{sphinxVerbatim}[commandchars=\\\{\}]
\PYG{n}{tape\PYGZus{}image}\PYG{o}{.}\PYG{n}{auto\PYGZus{}crop\PYGZus{}y}\PYG{p}{(}\PYG{p}{)}
\PYG{n}{tape\PYGZus{}image}\PYG{o}{.}\PYG{n}{show}\PYG{p}{(}\PYG{n}{cmap}\PYG{o}{=}\PYG{l+s+s1}{\PYGZsq{}}\PYG{l+s+s1}{gray}\PYG{l+s+s1}{\PYGZsq{}}\PYG{p}{)}
\PYG{n}{tape\PYGZus{}image}\PYG{o}{.}\PYG{n}{plot\PYGZus{}boundary}\PYG{p}{(}\PYG{n}{color}\PYG{o}{=}\PYG{l+s+s1}{\PYGZsq{}}\PYG{l+s+s1}{red}\PYG{l+s+s1}{\PYGZsq{}}\PYG{p}{)}
\end{sphinxVerbatim}

\noindent{\hspace*{\fill}\sphinxincludegraphics{{5.auto_crop_y}.png}\hspace*{\fill}}


\paragraph{1.6. Rotate Image}
\label{\detokenize{tutorials:rotate-image}}
\sphinxAtStartPar
As the name suggests This method will perform an rotation around the center of the image.

\sphinxAtStartPar
Usage:

\begin{sphinxVerbatim}[commandchars=\\\{\}]
\PYG{n}{tape\PYGZus{}image}\PYG{o}{.}\PYG{n}{rotate\PYGZus{}image}\PYG{p}{(}\PYG{l+m+mi}{90}\PYG{p}{)}
\PYG{n}{tape\PYGZus{}image}\PYG{o}{.}\PYG{n}{show}\PYG{p}{(}\PYG{n}{cmap}\PYG{o}{=}\PYG{l+s+s1}{\PYGZsq{}}\PYG{l+s+s1}{gray}\PYG{l+s+s1}{\PYGZsq{}}\PYG{p}{)}
\end{sphinxVerbatim}


\paragraph{1.7. Coordinate Based}
\label{\detokenize{tutorials:coordinate-based}}
\sphinxAtStartPar
This method will return a 2 dimentional array of coordinates of points on the edge.
The most important parameter for this method is \sphinxcode{\sphinxupquote{npoints}} representing the number of
points in the returned array. This method divides the edge into the small sections and
returns the average of each section as one point. if the parameter \sphinxcode{\sphinxupquote{plot}} is set to true
the plots will be plotted on the main image. The following example contains 1000 points for
the resulting array.

\sphinxAtStartPar
Usage:

\begin{sphinxVerbatim}[commandchars=\\\{\}]
\PYG{n}{tape\PYGZus{}image}\PYG{o}{.}\PYG{n}{coordinate\PYGZus{}based}\PYG{p}{(}\PYG{n}{plot}\PYG{o}{=}\PYG{k+kc}{True}\PYG{p}{,}\PYG{n}{x\PYGZus{}trim\PYGZus{}param}\PYG{o}{=}\PYG{l+m+mi}{6}\PYG{p}{,}\PYG{n}{npoints}\PYG{o}{=}\PYG{l+m+mi}{500}\PYG{p}{)}
\end{sphinxVerbatim}

\noindent\sphinxincludegraphics{{7.coordinate_based_zoomed}.png}

\noindent\sphinxincludegraphics{{7.coordinate_based}.png}


\paragraph{1.8. Weft based}
\label{\detokenize{tutorials:weft-based}}
\sphinxAtStartPar
In order to get as close as possible to a regular examination, this method was
added. This method will divide the edge of the image by the number of segments
defined by \sphinxcode{\sphinxupquote{nsegments}}. If this value is chose to be as close as the number of
wefts in a specific tape, the segments will be close to separating the segments
by the wefts. There are three important paramters that can be passed on to this
method. \sphinxcode{\sphinxupquote{window\_backround}} and \sphinxcode{\sphinxupquote{window\_tape}} define the number of pixels
that are going to be considered from the edge towards the background and from the
edge towards the tape respectively. There two different approaches that one can
define the window, either the whole window is fixed for the whole image or the window
can moves to adjust the same amount of background and tape to be involve in the image.
This can be defined by \sphinxcode{\sphinxupquote{dynamic\_window}} equal to \sphinxcode{\sphinxupquote{True}} or \sphinxcode{\sphinxupquote{False}}. The
following example can illustrate the dynamic window better. The image on the left
represents \sphinxcode{\sphinxupquote{dynamic=True}} and the image on the right represent \sphinxcode{\sphinxupquote{dynamic=False}}.
Similar to the coordinate based, if one choose \sphinxcode{\sphinxupquote{plot=True}}, one can oversee the
boundary and window selection by plotting the results.

\sphinxAtStartPar
Usage:

\begin{sphinxVerbatim}[commandchars=\\\{\}]
\PYG{n}{tape\PYGZus{}image}\PYG{o}{.}\PYG{n}{weft\PYGZus{}based}\PYG{p}{(}\PYG{n}{plot}\PYG{o}{=}\PYG{k+kc}{True}\PYG{p}{,}\PYG{n}{dynamic\PYGZus{}window}\PYG{o}{=}\PYG{k+kc}{True}\PYG{p}{,}\PYG{n}{nsegments}\PYG{o}{=}\PYG{l+m+mi}{39}\PYG{p}{,}
                  \PYG{n}{window\PYGZus{}background}\PYG{o}{=}\PYG{l+m+mi}{70}\PYG{p}{,}\PYG{n}{window\PYGZus{}tape}\PYG{o}{=}\PYG{l+m+mi}{300}\PYG{p}{)}
\PYG{n}{tape\PYGZus{}image}\PYG{o}{.}\PYG{n}{weft\PYGZus{}based}\PYG{p}{(}\PYG{n}{plot}\PYG{o}{=}\PYG{k+kc}{True}\PYG{p}{,}\PYG{n}{dynamic\PYGZus{}window}\PYG{o}{=}\PYG{k+kc}{False}\PYG{p}{,}\PYG{n}{nsegments}\PYG{o}{=}\PYG{l+m+mi}{39}\PYG{p}{,}
                  \PYG{n}{window\PYGZus{}background}\PYG{o}{=}\PYG{l+m+mi}{70}\PYG{p}{,}\PYG{n}{window\PYGZus{}tape}\PYG{o}{=}\PYG{l+m+mi}{300}\PYG{p}{)}
\end{sphinxVerbatim}

\noindent\sphinxincludegraphics[width=0.495\linewidth]{{6.weft_based_dynamic1}.png}

\noindent\sphinxincludegraphics[width=0.495\linewidth]{{6.weft_based}.png}

\sphinxAtStartPar
If \sphinxcode{\sphinxupquote{nsegments}} is set to a small number, this approach can be looked at as a
big picture approach that can help improve the weft base approach.


\paragraph{1.9. Max Contrast}
\label{\detokenize{tutorials:max-contrast}}
\sphinxAtStartPar
This method makes a maximum contrast between the edge and the rest of the image.
The rest of the image is represneted by a white color(255) and the rest of the
image is represented by a black color(0). Similar to the weft based one can choose
the number of pixels included in the \sphinxcode{\sphinxupquote{window\_background}} and the \sphinxcode{\sphinxupquote{wondow\_tape}}.
\sphinxcode{\sphinxupquote{window\_backround}} and \sphinxcode{\sphinxupquote{window\_tape}} define the number of pixels
that are going to be considered from the edge towards the background and from the
edge towards the tape respectively. There two different approaches that one can
define the window, either the whole window is fixed for the whole image or the window
can moves to adjust the same amount of background and tape to be involve in the image.
Similar to the coordinate based, if one choose \sphinxcode{\sphinxupquote{plot=True}}, one can oversee the
boundary and window selection by plotting the results. In the following example
we choose the default for the \sphinxcode{\sphinxupquote{window\_background}} and \sphinxcode{\sphinxupquote{window\_tape}}.

\sphinxAtStartPar
Usage:

\begin{sphinxVerbatim}[commandchars=\\\{\}]
\PYG{n}{tape\PYGZus{}image}\PYG{o}{.}\PYG{n}{max\PYGZus{}contrast}\PYG{p}{(}\PYG{n}{plot}\PYG{o}{=}\PYG{k+kc}{True}\PYG{p}{)}
\end{sphinxVerbatim}

\noindent{\hspace*{\fill}\sphinxincludegraphics[width=0.200\linewidth]{{9.max_contrast}.png}\hspace*{\fill}}

\sphinxstepscope


\section{forensicfit}
\label{\detokenize{modules:forensicfit}}\label{\detokenize{modules::doc}}
\sphinxstepscope


\subsection{forensicfit package}
\label{\detokenize{forensicfit:forensicfit-package}}\label{\detokenize{forensicfit::doc}}

\subsubsection{Subpackages}
\label{\detokenize{forensicfit:subpackages}}
\sphinxstepscope


\paragraph{forensicfit.core package}
\label{\detokenize{forensicfit.core:forensicfit-core-package}}\label{\detokenize{forensicfit.core::doc}}

\subparagraph{Submodules}
\label{\detokenize{forensicfit.core:submodules}}
\sphinxstepscope


\subparagraph{forensicfit.core.analyzer module}
\label{\detokenize{forensicfit.core.analyzer:module-forensicfit.core.analyzer}}\label{\detokenize{forensicfit.core.analyzer:forensicfit-core-analyzer-module}}\label{\detokenize{forensicfit.core.analyzer::doc}}\index{module@\spxentry{module}!forensicfit.core.analyzer@\spxentry{forensicfit.core.analyzer}}\index{forensicfit.core.analyzer@\spxentry{forensicfit.core.analyzer}!module@\spxentry{module}}
\sphinxAtStartPar
analyzer.py

\sphinxAtStartPar
This module contains the Analyzer class, which is responsible for
handling and analyzing images in the context of the ForensicFit
application. It uses computer vision techniques for image analysis
and provides utilities for plotting the results and converting
images to and from byte buffers.

\sphinxAtStartPar
The module includes the following classes:
\sphinxhyphen{} Analyzer: An abstract base class that defines the necessary
interface for image analysis in the ForensicFit application.

\sphinxAtStartPar
Author: Pedram Tavadze
Email: \sphinxhref{mailto:petavazohi@gmail.com}{petavazohi@gmail.com}
\index{Analyzer (class in forensicfit.core.analyzer)@\spxentry{Analyzer}\spxextra{class in forensicfit.core.analyzer}}

\begin{fulllineitems}
\phantomsection\label{\detokenize{forensicfit.core.analyzer:forensicfit.core.analyzer.Analyzer}}
\pysigstartsignatures
\pysiglinewithargsret{\sphinxbfcode{\sphinxupquote{class\DUrole{w,w}{  }}}\sphinxcode{\sphinxupquote{forensicfit.core.analyzer.}}\sphinxbfcode{\sphinxupquote{Analyzer}}}{\sphinxparam{\DUrole{o,o}{**}\DUrole{n,n}{kwargs}}}{}
\pysigstopsignatures
\sphinxAtStartPar
Bases: \sphinxcode{\sphinxupquote{object}}

\sphinxAtStartPar
Abstract base class that represents an analyzer in the system.

\sphinxAtStartPar
The Analyzer class is designed to be subclassed by concrete analyzer classes.
Each subclass should implement the methods that make sense for that specific
type of analyzer.

\sphinxAtStartPar
This class uses the Abstract Base Classes (ABC) module which enables the
creation of a blueprint for other classes. This means you can’t create an
instance of this class, it is intended to be subclassed. All methods marked
with @abstractmethod must be implemented in any concrete (i.e.,
non\sphinxhyphen{}abstract) subclass.
\index{image (forensicfit.core.analyzer.Analyzer attribute)@\spxentry{image}\spxextra{forensicfit.core.analyzer.Analyzer attribute}}

\begin{fulllineitems}
\phantomsection\label{\detokenize{forensicfit.core.analyzer:forensicfit.core.analyzer.Analyzer.image}}
\pysigstartsignatures
\pysigline{\sphinxbfcode{\sphinxupquote{image}}}
\pysigstopsignatures
\sphinxAtStartPar
The image to be analyzed. This attribute is expected to be a numpy
array representing the image, but it’s initially set to None.
\begin{quote}\begin{description}
\sphinxlineitem{Type}
\sphinxAtStartPar
np.ndarray

\end{description}\end{quote}

\end{fulllineitems}

\index{values (forensicfit.core.analyzer.Analyzer attribute)@\spxentry{values}\spxextra{forensicfit.core.analyzer.Analyzer attribute}}

\begin{fulllineitems}
\phantomsection\label{\detokenize{forensicfit.core.analyzer:forensicfit.core.analyzer.Analyzer.values}}
\pysigstartsignatures
\pysigline{\sphinxbfcode{\sphinxupquote{values}}}
\pysigstopsignatures
\sphinxAtStartPar
A dictionary that contains the results of the analysis. The keys are
strings describing what each value represents.
\begin{quote}\begin{description}
\sphinxlineitem{Type}
\sphinxAtStartPar
dict

\end{description}\end{quote}

\end{fulllineitems}

\index{metadata (forensicfit.core.analyzer.Analyzer attribute)@\spxentry{metadata}\spxextra{forensicfit.core.analyzer.Analyzer attribute}}

\begin{fulllineitems}
\phantomsection\label{\detokenize{forensicfit.core.analyzer:forensicfit.core.analyzer.Analyzer.metadata}}
\pysigstartsignatures
\pysigline{\sphinxbfcode{\sphinxupquote{metadata}}}
\pysigstopsignatures
\sphinxAtStartPar
An instance of the Metadata class, containing metadata related to the
analysis.
\begin{quote}\begin{description}
\sphinxlineitem{Type}
\sphinxAtStartPar
Metadata

\end{description}\end{quote}

\end{fulllineitems}

\subsubsection*{Notes}

\sphinxAtStartPar
This class is part of a module called “analyzer.py”. It serves as the
parent class for all future analyzers in the system.
\begin{quote}\begin{description}
\sphinxlineitem{Attributes}\begin{description}
\sphinxlineitem{{\hyperref[\detokenize{forensicfit.core.analyzer:forensicfit.core.analyzer.Analyzer.shape}]{\sphinxcrossref{\sphinxcode{\sphinxupquote{shape}}}}}}
\sphinxAtStartPar
A property that provides the shape of the image contained in the Analyzer instance.

\end{description}

\end{description}\end{quote}
\subsubsection*{Methods}


\begin{savenotes}\sphinxattablestart
\sphinxthistablewithglobalstyle
\sphinxthistablewithnovlinesstyle
\centering
\begin{tabulary}{\linewidth}[t]{\X{1}{2}\X{1}{2}}
\sphinxtoprule
\sphinxtableatstartofbodyhook
\sphinxAtStartPar
{\hyperref[\detokenize{forensicfit.core.analyzer:forensicfit.core.analyzer.Analyzer.apply_filter}]{\sphinxcrossref{\sphinxcode{\sphinxupquote{apply\_filter}}}}}(mode, **kwargs)
&
\sphinxAtStartPar

\\
\sphinxhline
\sphinxAtStartPar
{\hyperref[\detokenize{forensicfit.core.analyzer:forensicfit.core.analyzer.Analyzer.exposure_control}]{\sphinxcrossref{\sphinxcode{\sphinxupquote{exposure\_control}}}}}({[}mode{]})
&
\sphinxAtStartPar

\\
\sphinxhline
\sphinxAtStartPar
{\hyperref[\detokenize{forensicfit.core.analyzer:forensicfit.core.analyzer.Analyzer.from_buffer}]{\sphinxcrossref{\sphinxcode{\sphinxupquote{from\_buffer}}}}}(buffer, metadata{[}, ext{]})
&
\sphinxAtStartPar
Receives an io byte buffer with the corresponding metadata and creates an instance of the class.
\\
\sphinxhline
\sphinxAtStartPar
{\hyperref[\detokenize{forensicfit.core.analyzer:forensicfit.core.analyzer.Analyzer.from_dict}]{\sphinxcrossref{\sphinxcode{\sphinxupquote{from\_dict}}}}}()
&
\sphinxAtStartPar
Abstract method for setting the state of an object from a dictionary.
\\
\sphinxhline
\sphinxAtStartPar
{\hyperref[\detokenize{forensicfit.core.analyzer:forensicfit.core.analyzer.Analyzer.load_dict}]{\sphinxcrossref{\sphinxcode{\sphinxupquote{load\_dict}}}}}()
&
\sphinxAtStartPar
Abstract method for loading a dictionary.
\\
\sphinxhline
\sphinxAtStartPar
{\hyperref[\detokenize{forensicfit.core.analyzer:forensicfit.core.analyzer.Analyzer.plot}]{\sphinxcrossref{\sphinxcode{\sphinxupquote{plot}}}}}(which{[}, cmap, zoom, savefig, ax, show, ...{]})
&
\sphinxAtStartPar
Plots different kinds of data based on the given parameters.
\\
\sphinxhline
\sphinxAtStartPar
{\hyperref[\detokenize{forensicfit.core.analyzer:forensicfit.core.analyzer.Analyzer.plot_boundary}]{\sphinxcrossref{\sphinxcode{\sphinxupquote{plot\_boundary}}}}}({[}savefig, color, ax, show{]})
&
\sphinxAtStartPar
Plots the detected boundary of the image.
\\
\sphinxhline
\sphinxAtStartPar
{\hyperref[\detokenize{forensicfit.core.analyzer:forensicfit.core.analyzer.Analyzer.resize}]{\sphinxcrossref{\sphinxcode{\sphinxupquote{resize}}}}}({[}size, dpi{]})
&
\sphinxAtStartPar
Resize the image associated with this analyzer.
\\
\sphinxhline
\sphinxAtStartPar
{\hyperref[\detokenize{forensicfit.core.analyzer:forensicfit.core.analyzer.Analyzer.to_buffer}]{\sphinxcrossref{\sphinxcode{\sphinxupquote{to\_buffer}}}}}({[}ext{]})
&
\sphinxAtStartPar
Converts the current instance of the Analyzer class to a byte buffer, which can be useful for serialization or for writing to a file.
\\
\sphinxbottomrule
\end{tabulary}
\sphinxtableafterendhook\par
\sphinxattableend\end{savenotes}
\index{\_\_init\_\_() (forensicfit.core.analyzer.Analyzer method)@\spxentry{\_\_init\_\_()}\spxextra{forensicfit.core.analyzer.Analyzer method}}

\begin{fulllineitems}
\phantomsection\label{\detokenize{forensicfit.core.analyzer:forensicfit.core.analyzer.Analyzer.__init__}}
\pysigstartsignatures
\pysiglinewithargsret{\sphinxbfcode{\sphinxupquote{\_\_init\_\_}}}{\sphinxparam{\DUrole{o,o}{**}\DUrole{n,n}{kwargs}}}{}
\pysigstopsignatures
\end{fulllineitems}

\index{exposure\_control() (forensicfit.core.analyzer.Analyzer method)@\spxentry{exposure\_control()}\spxextra{forensicfit.core.analyzer.Analyzer method}}

\begin{fulllineitems}
\phantomsection\label{\detokenize{forensicfit.core.analyzer:forensicfit.core.analyzer.Analyzer.exposure_control}}
\pysigstartsignatures
\pysiglinewithargsret{\sphinxbfcode{\sphinxupquote{exposure\_control}}}{\sphinxparam{\DUrole{n,n}{mode}\DUrole{o,o}{=}\DUrole{default_value}{\textquotesingle{}equalize\_hist\textquotesingle{}}}, \sphinxparam{\DUrole{o,o}{**}\DUrole{n,n}{kwargs}}}{}
\pysigstopsignatures
\end{fulllineitems}

\index{apply\_filter() (forensicfit.core.analyzer.Analyzer method)@\spxentry{apply\_filter()}\spxextra{forensicfit.core.analyzer.Analyzer method}}

\begin{fulllineitems}
\phantomsection\label{\detokenize{forensicfit.core.analyzer:forensicfit.core.analyzer.Analyzer.apply_filter}}
\pysigstartsignatures
\pysiglinewithargsret{\sphinxbfcode{\sphinxupquote{apply\_filter}}}{\sphinxparam{\DUrole{n,n}{mode}}, \sphinxparam{\DUrole{o,o}{**}\DUrole{n,n}{kwargs}}}{}
\pysigstopsignatures
\end{fulllineitems}

\index{resize() (forensicfit.core.analyzer.Analyzer method)@\spxentry{resize()}\spxextra{forensicfit.core.analyzer.Analyzer method}}

\begin{fulllineitems}
\phantomsection\label{\detokenize{forensicfit.core.analyzer:forensicfit.core.analyzer.Analyzer.resize}}
\pysigstartsignatures
\pysiglinewithargsret{\sphinxbfcode{\sphinxupquote{resize}}}{\sphinxparam{\DUrole{n,n}{size}\DUrole{o,o}{=}\DUrole{default_value}{None}}, \sphinxparam{\DUrole{n,n}{dpi}\DUrole{o,o}{=}\DUrole{default_value}{None}}}{}
\pysigstopsignatures
\sphinxAtStartPar
Resize the image associated with this analyzer.

\sphinxAtStartPar
The method allows to resize the image either by providing the desired
output size, or by providing the dots per inch (dpi). If both parameters
are None, the image will not be modified.
\begin{quote}\begin{description}
\sphinxlineitem{Parameters}\begin{itemize}
\item {} 
\sphinxAtStartPar
\sphinxstyleliteralstrong{\sphinxupquote{size}} (\sphinxstyleliteralemphasis{\sphinxupquote{tuple}}\sphinxstyleliteralemphasis{\sphinxupquote{, }}\sphinxstyleliteralemphasis{\sphinxupquote{optional}}) \textendash{} Desired output size in pixels as (height, width). If provided, this
value will be used to resize the image. Default is None.

\item {} 
\sphinxAtStartPar
\sphinxstyleliteralstrong{\sphinxupquote{dpi}} (\sphinxstyleliteralemphasis{\sphinxupquote{tuple}}\sphinxstyleliteralemphasis{\sphinxupquote{, }}\sphinxstyleliteralemphasis{\sphinxupquote{optional}}) \textendash{} Desired dots per inch (dpi) as (horizontal, vertical). If provided and
‘dpi’ key exists in the metadata, this value will be used to compute
the new size and then resize the image. Default is None.

\end{itemize}

\sphinxlineitem{Return type}
\sphinxAtStartPar
None

\sphinxlineitem{Raises}
\sphinxAtStartPar
\sphinxstyleliteralstrong{\sphinxupquote{ValueError}} \textendash{} If both ‘size’ and ‘dpi’ are None.

\end{description}\end{quote}

\end{fulllineitems}

\index{plot\_boundary() (forensicfit.core.analyzer.Analyzer method)@\spxentry{plot\_boundary()}\spxextra{forensicfit.core.analyzer.Analyzer method}}

\begin{fulllineitems}
\phantomsection\label{\detokenize{forensicfit.core.analyzer:forensicfit.core.analyzer.Analyzer.plot_boundary}}
\pysigstartsignatures
\pysiglinewithargsret{\sphinxbfcode{\sphinxupquote{plot\_boundary}}}{\sphinxparam{\DUrole{n,n}{savefig}\DUrole{o,o}{=}\DUrole{default_value}{None}}, \sphinxparam{\DUrole{n,n}{color}\DUrole{o,o}{=}\DUrole{default_value}{\textquotesingle{}red\textquotesingle{}}}, \sphinxparam{\DUrole{n,n}{ax}\DUrole{o,o}{=}\DUrole{default_value}{None}}, \sphinxparam{\DUrole{n,n}{show}\DUrole{o,o}{=}\DUrole{default_value}{False}}}{}
\pysigstopsignatures
\sphinxAtStartPar
Plots the detected boundary of the image.
\begin{quote}\begin{description}
\sphinxlineitem{Parameters}\begin{itemize}
\item {} 
\sphinxAtStartPar
\sphinxstyleliteralstrong{\sphinxupquote{savefig}} (\sphinxstyleliteralemphasis{\sphinxupquote{Union}}\sphinxstyleliteralemphasis{\sphinxupquote{{[}}}\sphinxstyleliteralemphasis{\sphinxupquote{str}}\sphinxstyleliteralemphasis{\sphinxupquote{, }}\sphinxstyleliteralemphasis{\sphinxupquote{Path}}\sphinxstyleliteralemphasis{\sphinxupquote{{]}}}\sphinxstyleliteralemphasis{\sphinxupquote{, }}\sphinxstyleliteralemphasis{\sphinxupquote{optional}}) \textendash{} Path to save the plot. If provided, the plot will be saved at the
specified location. If None, the plot will not be saved.
Default is None.

\item {} 
\sphinxAtStartPar
\sphinxstyleliteralstrong{\sphinxupquote{color}} (\sphinxstyleliteralemphasis{\sphinxupquote{str}}\sphinxstyleliteralemphasis{\sphinxupquote{, }}\sphinxstyleliteralemphasis{\sphinxupquote{optional}}) \textendash{} Color of the boundary line in the plot. Default is ‘red’.

\item {} 
\sphinxAtStartPar
\sphinxstyleliteralstrong{\sphinxupquote{ax}} (\sphinxstyleliteralemphasis{\sphinxupquote{matplotlib.axes.Axes}}\sphinxstyleliteralemphasis{\sphinxupquote{, }}\sphinxstyleliteralemphasis{\sphinxupquote{optional}}) \textendash{} An instance of Axes in which to draw the plot. If None, a new
Axes instance will be created. Default is None.

\item {} 
\sphinxAtStartPar
\sphinxstyleliteralstrong{\sphinxupquote{show}} (\sphinxstyleliteralemphasis{\sphinxupquote{bool}}\sphinxstyleliteralemphasis{\sphinxupquote{, }}\sphinxstyleliteralemphasis{\sphinxupquote{optional}}) \textendash{} Controls whether to show the image using matplotlib.pyplot.show()
after it is drawn. Default is False.

\end{itemize}

\sphinxlineitem{Returns}
\sphinxAtStartPar
\sphinxstylestrong{ax} \textendash{} The Axes instance in which the plot was drawn.

\sphinxlineitem{Return type}
\sphinxAtStartPar
matplotlib.axes.Axes

\end{description}\end{quote}

\end{fulllineitems}

\index{plot() (forensicfit.core.analyzer.Analyzer method)@\spxentry{plot()}\spxextra{forensicfit.core.analyzer.Analyzer method}}

\begin{fulllineitems}
\phantomsection\label{\detokenize{forensicfit.core.analyzer:forensicfit.core.analyzer.Analyzer.plot}}
\pysigstartsignatures
\pysiglinewithargsret{\sphinxbfcode{\sphinxupquote{plot}}}{\sphinxparam{\DUrole{n,n}{which}}, \sphinxparam{\DUrole{n,n}{cmap}\DUrole{o,o}{=}\DUrole{default_value}{\textquotesingle{}gray\textquotesingle{}}}, \sphinxparam{\DUrole{n,n}{zoom}\DUrole{o,o}{=}\DUrole{default_value}{4}}, \sphinxparam{\DUrole{n,n}{savefig}\DUrole{o,o}{=}\DUrole{default_value}{None}}, \sphinxparam{\DUrole{n,n}{ax}\DUrole{o,o}{=}\DUrole{default_value}{None}}, \sphinxparam{\DUrole{n,n}{show}\DUrole{o,o}{=}\DUrole{default_value}{False}}, \sphinxparam{\DUrole{n,n}{mode}\DUrole{o,o}{=}\DUrole{default_value}{None}}, \sphinxparam{\DUrole{o,o}{**}\DUrole{n,n}{kwargs}}}{}
\pysigstopsignatures
\sphinxAtStartPar
Plots different kinds of data based on the given parameters.
\begin{quote}\begin{description}
\sphinxlineitem{Parameters}\begin{itemize}
\item {} 
\sphinxAtStartPar
\sphinxstyleliteralstrong{\sphinxupquote{which}} (\sphinxstyleliteralemphasis{\sphinxupquote{str}}) \textendash{} Determines the kind of plot to be created. Possible values include
“coordinate\_based”, “boundary”, “bin\_based+coordinate\_based”,
“coordinate\_based+bin\_based”, “bin\_based”,
“bin\_based+max\_contrast”, “max\_contrast+bin\_based” and others.

\item {} 
\sphinxAtStartPar
\sphinxstyleliteralstrong{\sphinxupquote{cmap}} (\sphinxstyleliteralemphasis{\sphinxupquote{str}}\sphinxstyleliteralemphasis{\sphinxupquote{, }}\sphinxstyleliteralemphasis{\sphinxupquote{optional}}) \textendash{} The Colormap instance or registered colormap name. Default is ‘gray’.

\item {} 
\sphinxAtStartPar
\sphinxstyleliteralstrong{\sphinxupquote{zoom}} (\sphinxstyleliteralemphasis{\sphinxupquote{int}}\sphinxstyleliteralemphasis{\sphinxupquote{, }}\sphinxstyleliteralemphasis{\sphinxupquote{optional}}) \textendash{} The zoom factor for the plot. Default is 4.

\item {} 
\sphinxAtStartPar
\sphinxstyleliteralstrong{\sphinxupquote{savefig}} (\sphinxstyleliteralemphasis{\sphinxupquote{str}}\sphinxstyleliteralemphasis{\sphinxupquote{, }}\sphinxstyleliteralemphasis{\sphinxupquote{optional}}) \textendash{} Path and name to save the image. If None, the plot will not be saved.
Default is None.

\item {} 
\sphinxAtStartPar
\sphinxstyleliteralstrong{\sphinxupquote{ax}} (\sphinxstyleliteralemphasis{\sphinxupquote{matplotlib.axes.Axes}}\sphinxstyleliteralemphasis{\sphinxupquote{ or }}\sphinxstyleliteralemphasis{\sphinxupquote{List}}\sphinxstyleliteralemphasis{\sphinxupquote{{[}}}\sphinxstyleliteralemphasis{\sphinxupquote{matplotlib.axes.Axes}}\sphinxstyleliteralemphasis{\sphinxupquote{{]}}}\sphinxstyleliteralemphasis{\sphinxupquote{, }}\sphinxstyleliteralemphasis{\sphinxupquote{optional}}) \textendash{} An instance of Axes or list of Axes in which to draw the plot. If None,
a new Axes instance will be created. Default is None.

\item {} 
\sphinxAtStartPar
\sphinxstyleliteralstrong{\sphinxupquote{show}} (\sphinxstyleliteralemphasis{\sphinxupquote{bool}}\sphinxstyleliteralemphasis{\sphinxupquote{, }}\sphinxstyleliteralemphasis{\sphinxupquote{optional}}) \textendash{} If True, displays the image. Default is False.

\item {} 
\sphinxAtStartPar
\sphinxstyleliteralstrong{\sphinxupquote{mode}} (\sphinxstyleliteralemphasis{\sphinxupquote{str}}\sphinxstyleliteralemphasis{\sphinxupquote{, }}\sphinxstyleliteralemphasis{\sphinxupquote{optional}}) \textendash{} Determines the mode of operation, which affects how the plot is
generated. The effect depends on the value of \sphinxtitleref{which}.

\item {} 
\sphinxAtStartPar
\sphinxstyleliteralstrong{\sphinxupquote{**kwargs}} \textendash{} Arbitrary keyword arguments.

\end{itemize}

\sphinxlineitem{Returns}
\sphinxAtStartPar
\sphinxstylestrong{ax} \textendash{} The Axes instance(s) in which the plot was drawn.

\sphinxlineitem{Return type}
\sphinxAtStartPar
matplotlib.axes.Axes or List{[}matplotlib.axes.Axes{]}

\end{description}\end{quote}

\end{fulllineitems}

\index{load\_dict() (forensicfit.core.analyzer.Analyzer method)@\spxentry{load\_dict()}\spxextra{forensicfit.core.analyzer.Analyzer method}}

\begin{fulllineitems}
\phantomsection\label{\detokenize{forensicfit.core.analyzer:forensicfit.core.analyzer.Analyzer.load_dict}}
\pysigstartsignatures
\pysiglinewithargsret{\sphinxbfcode{\sphinxupquote{abstract\DUrole{w,w}{  }}}\sphinxbfcode{\sphinxupquote{load\_dict}}}{}{}
\pysigstopsignatures
\sphinxAtStartPar
Abstract method for loading a dictionary.

\sphinxAtStartPar
This method should be implemented by any non\sphinxhyphen{}abstract subclass of
Analyzer. The implementation should handle the loading of some kind
of dictionary data specific to that subclass.
\begin{quote}\begin{description}
\sphinxlineitem{Returns}
\sphinxAtStartPar
\begin{itemize}
\item {} 
\sphinxAtStartPar
\sphinxstyleemphasis{Typically, this method would return the loaded dictionary, but the}

\item {} 
\sphinxAtStartPar
\sphinxstyleemphasis{exact return type and value will depend on the specific implementation}

\item {} 
\sphinxAtStartPar
\sphinxstyleemphasis{in the subclass.}

\end{itemize}


\end{description}\end{quote}

\end{fulllineitems}

\index{from\_dict() (forensicfit.core.analyzer.Analyzer method)@\spxentry{from\_dict()}\spxextra{forensicfit.core.analyzer.Analyzer method}}

\begin{fulllineitems}
\phantomsection\label{\detokenize{forensicfit.core.analyzer:forensicfit.core.analyzer.Analyzer.from_dict}}
\pysigstartsignatures
\pysiglinewithargsret{\sphinxbfcode{\sphinxupquote{abstract\DUrole{w,w}{  }}}\sphinxbfcode{\sphinxupquote{from\_dict}}}{}{}
\pysigstopsignatures
\sphinxAtStartPar
Abstract method for setting the state of an object from a dictionary.

\sphinxAtStartPar
This method should be implemented by any non\sphinxhyphen{}abstract subclass of
Analyzer. The implementation should set the state of an object based
on data provided in a dictionary.
\begin{quote}\begin{description}
\sphinxlineitem{Parameters}\begin{itemize}
\item {} 
\sphinxAtStartPar
\sphinxstyleliteralstrong{\sphinxupquote{implementation}} (\sphinxstyleliteralemphasis{\sphinxupquote{This will vary depending on the subclass}}) \textendash{} 

\item {} 
\sphinxAtStartPar
\sphinxstyleliteralstrong{\sphinxupquote{typically}} (\sphinxstyleliteralemphasis{\sphinxupquote{but}}) \textendash{} 

\item {} 
\sphinxAtStartPar
\sphinxstyleliteralstrong{\sphinxupquote{argument}} (\sphinxstyleliteralemphasis{\sphinxupquote{this method would accept a single}}) \textendash{} 

\item {} 
\sphinxAtStartPar
\sphinxstyleliteralstrong{\sphinxupquote{state.}} (\sphinxstyleliteralemphasis{\sphinxupquote{the data to use when setting the object\textquotesingle{}s}}) \textendash{} 

\end{itemize}

\sphinxlineitem{Returns}
\sphinxAtStartPar
\begin{itemize}
\item {} 
\sphinxAtStartPar
\sphinxstyleemphasis{Typically, this method would not return a value, but this will depend}

\item {} 
\sphinxAtStartPar
\sphinxstyleemphasis{on the specific implementation in the subclass.}

\end{itemize}


\end{description}\end{quote}

\end{fulllineitems}

\index{from\_buffer() (forensicfit.core.analyzer.Analyzer class method)@\spxentry{from\_buffer()}\spxextra{forensicfit.core.analyzer.Analyzer class method}}

\begin{fulllineitems}
\phantomsection\label{\detokenize{forensicfit.core.analyzer:forensicfit.core.analyzer.Analyzer.from_buffer}}
\pysigstartsignatures
\pysiglinewithargsret{\sphinxbfcode{\sphinxupquote{classmethod\DUrole{w,w}{  }}}\sphinxbfcode{\sphinxupquote{from\_buffer}}}{\sphinxparam{\DUrole{n,n}{buffer}}, \sphinxparam{\DUrole{n,n}{metadata}}, \sphinxparam{\DUrole{n,n}{ext}\DUrole{o,o}{=}\DUrole{default_value}{\textquotesingle{}.png\textquotesingle{}}}}{}
\pysigstopsignatures
\sphinxAtStartPar
Receives an io byte buffer with the corresponding metadata and
creates an instance of the class. This class method is helpful in
situations where you have raw image data along with associated metadata
and need to create an Analyzer object.
\begin{quote}\begin{description}
\sphinxlineitem{Parameters}\begin{itemize}
\item {} 
\sphinxAtStartPar
\sphinxstyleliteralstrong{\sphinxupquote{buffer}} (\sphinxstyleliteralemphasis{\sphinxupquote{bytes}}) \textendash{} A buffer containing raw image data, typically in the form of bytes.

\item {} 
\sphinxAtStartPar
\sphinxstyleliteralstrong{\sphinxupquote{metadata}} (\sphinxstyleliteralemphasis{\sphinxupquote{dict}}) \textendash{} A dictionary containing metadata related to the image. The specific
contents will depend on your application, but might include things
like the image’s origin, resolution, or creation date.

\item {} 
\sphinxAtStartPar
\sphinxstyleliteralstrong{\sphinxupquote{ext}} (\sphinxstyleliteralemphasis{\sphinxupquote{str}}\sphinxstyleliteralemphasis{\sphinxupquote{, }}\sphinxstyleliteralemphasis{\sphinxupquote{optional}}) \textendash{} The file extension of the image being loaded. Used to determine
the decoding method. Default is ‘.png’. If the ‘ext’ key is present
in the metadata dict, it will override this parameter.

\end{itemize}

\sphinxlineitem{Returns}
\sphinxAtStartPar
\begin{itemize}
\item {} 
\sphinxAtStartPar
\sphinxstyleemphasis{An instance of the Analyzer class, initialized with the image and}

\item {} 
\sphinxAtStartPar
\sphinxstyleemphasis{metadata provided.}

\end{itemize}


\end{description}\end{quote}

\end{fulllineitems}

\index{to\_buffer() (forensicfit.core.analyzer.Analyzer method)@\spxentry{to\_buffer()}\spxextra{forensicfit.core.analyzer.Analyzer method}}

\begin{fulllineitems}
\phantomsection\label{\detokenize{forensicfit.core.analyzer:forensicfit.core.analyzer.Analyzer.to_buffer}}
\pysigstartsignatures
\pysiglinewithargsret{\sphinxbfcode{\sphinxupquote{to\_buffer}}}{\sphinxparam{\DUrole{n,n}{ext}\DUrole{o,o}{=}\DUrole{default_value}{\textquotesingle{}.png\textquotesingle{}}}}{}
\pysigstopsignatures
\sphinxAtStartPar
Converts the current instance of the Analyzer class to a byte buffer,
which can be useful for serialization or for writing to a file. This
method supports various image formats determined by the extension provided.
\begin{quote}\begin{description}
\sphinxlineitem{Return type}
\sphinxAtStartPar
\sphinxcode{\sphinxupquote{bytes}}

\sphinxlineitem{Parameters}
\sphinxAtStartPar
\sphinxstyleliteralstrong{\sphinxupquote{ext}} (\sphinxstyleliteralemphasis{\sphinxupquote{str}}\sphinxstyleliteralemphasis{\sphinxupquote{, }}\sphinxstyleliteralemphasis{\sphinxupquote{optional}}) \textendash{} The file extension for the output buffer. This will determine the
format of the output image. Default is ‘.png’. This method supports
any image format that is recognized by the OpenCV library.

\sphinxlineitem{Returns}
\sphinxAtStartPar
A byte string representing the image data in the format specified
by ‘ext’. This can be directly written to a file or transmitted
over a network, among other things.

\sphinxlineitem{Return type}
\sphinxAtStartPar
bytes

\sphinxlineitem{Raises}
\sphinxAtStartPar
\sphinxstyleliteralstrong{\sphinxupquote{ValueError}} \textendash{} If the provided extension is not supported, a ValueError will be raised.

\end{description}\end{quote}

\end{fulllineitems}

\index{shape (forensicfit.core.analyzer.Analyzer property)@\spxentry{shape}\spxextra{forensicfit.core.analyzer.Analyzer property}}

\begin{fulllineitems}
\phantomsection\label{\detokenize{forensicfit.core.analyzer:forensicfit.core.analyzer.Analyzer.shape}}
\pysigstartsignatures
\pysigline{\sphinxbfcode{\sphinxupquote{property\DUrole{w,w}{  }}}\sphinxbfcode{\sphinxupquote{shape}}\sphinxbfcode{\sphinxupquote{\DUrole{p,p}{:}\DUrole{w,w}{  }tuple}}}
\pysigstopsignatures
\sphinxAtStartPar
A property that provides the shape of the image contained in the Analyzer instance.
\begin{quote}\begin{description}
\sphinxlineitem{Returns}
\sphinxAtStartPar
A tuple representing the shape of the image. For grayscale images,
this will be a 2\sphinxhyphen{}tuple (height, width). For color images, this
will be a 3\sphinxhyphen{}tuple (height, width, channels), where ‘channels’
is typically 3 for an RGB image or 4 for an RGBA image.

\sphinxlineitem{Return type}
\sphinxAtStartPar
tuple

\end{description}\end{quote}

\end{fulllineitems}


\end{fulllineitems}


\sphinxstepscope


\subparagraph{forensicfit.core.image module}
\label{\detokenize{forensicfit.core.image:module-forensicfit.core.image}}\label{\detokenize{forensicfit.core.image:forensicfit-core-image-module}}\label{\detokenize{forensicfit.core.image::doc}}\index{module@\spxentry{module}!forensicfit.core.image@\spxentry{forensicfit.core.image}}\index{forensicfit.core.image@\spxentry{forensicfit.core.image}!module@\spxentry{module}}
\sphinxstepscope


\subparagraph{forensicfit.core.metadata module}
\label{\detokenize{forensicfit.core.metadata:module-forensicfit.core.metadata}}\label{\detokenize{forensicfit.core.metadata:forensicfit-core-metadata-module}}\label{\detokenize{forensicfit.core.metadata::doc}}\index{module@\spxentry{module}!forensicfit.core.metadata@\spxentry{forensicfit.core.metadata}}\index{forensicfit.core.metadata@\spxentry{forensicfit.core.metadata}!module@\spxentry{module}}

\subparagraph{metadata.py}
\label{\detokenize{forensicfit.core.metadata:metadata-py}}
\sphinxAtStartPar
This module contains the Metadata class which is used to manage the metadata of images for forensic analysis.

\sphinxAtStartPar
The Metadata class implements a MutableMapping, which provides a dict\sphinxhyphen{}like interface. This class offers methods
for setting and retrieving metadata values, as well as special conversion methods that adapt the metadata to various
needs, such as MongoDB filter creation or serialization for storage in MongoDB.

\sphinxAtStartPar
The Metadata class makes use of the serializer function from the \sphinxtitleref{array\_tools} utility module, which converts nested
dictionaries and numpy arrays to a suitable form for storage in MongoDB.

\sphinxAtStartPar
This module is part of the ‘forensicfit’ package which aims to provide tools for forensic analysis of images.

\sphinxAtStartPar
Author: Pedram Tavadze
Email: \sphinxhref{mailto:petavazohi@gmail.com}{petavazohi@gmail.com}

\sphinxstepscope


\subparagraph{forensicfit.core.tape module}
\label{\detokenize{forensicfit.core.tape:module-forensicfit.core.tape}}\label{\detokenize{forensicfit.core.tape:forensicfit-core-tape-module}}\label{\detokenize{forensicfit.core.tape::doc}}\index{module@\spxentry{module}!forensicfit.core.tape@\spxentry{forensicfit.core.tape}}\index{forensicfit.core.tape@\spxentry{forensicfit.core.tape}!module@\spxentry{module}}
\sphinxAtStartPar
tape.py

\sphinxAtStartPar
This module contains classes for handling and analyzing measurements
from tape in the context of the ForensicFit application. These
measurements are represented as images which are processed and
analyzed using a variety of computer vision techniques.

\sphinxAtStartPar
The module includes the following classes:
\begin{itemize}
\item {} 
\sphinxAtStartPar
Tape: A class that represents an image of a tape measurement. It

\end{itemize}

\sphinxAtStartPar
provides functionalities to load and process the image, extract
relevant information, and perform various operations such as
binarization and smearing.
\begin{itemize}
\item {} 
\sphinxAtStartPar
TapeAnalyzer: A class that inherits from the Tape class, adding

\end{itemize}

\sphinxAtStartPar
analysis functionalities. It can calculate the tape’s boundary,
plot it, and compute and store analysis metadata.

\sphinxAtStartPar
The Tape class represents a single tape measurement and provides
basic image processing operations. The TapeAnalyzer class extends
this functionality by adding methods to analyze the tape’s boundary
and other features, which can be useful in forensic applications.

\sphinxAtStartPar
Author: Pedram Tavadze
Email: \sphinxhref{mailto:petavazohi@gmail.com}{petavazohi@gmail.com}
\index{Tape (class in forensicfit.core.tape)@\spxentry{Tape}\spxextra{class in forensicfit.core.tape}}

\begin{fulllineitems}
\phantomsection\label{\detokenize{forensicfit.core.tape:forensicfit.core.tape.Tape}}
\pysigstartsignatures
\pysiglinewithargsret{\sphinxbfcode{\sphinxupquote{class\DUrole{w,w}{  }}}\sphinxcode{\sphinxupquote{forensicfit.core.tape.}}\sphinxbfcode{\sphinxupquote{Tape}}}{\sphinxparam{\DUrole{n,n}{image}}, \sphinxparam{\DUrole{n,n}{label}\DUrole{o,o}{=}\DUrole{default_value}{None}}, \sphinxparam{\DUrole{n,n}{surface}\DUrole{o,o}{=}\DUrole{default_value}{None}}, \sphinxparam{\DUrole{n,n}{stretched}\DUrole{o,o}{=}\DUrole{default_value}{False}}, \sphinxparam{\DUrole{o,o}{**}\DUrole{n,n}{kwargs}}}{}
\pysigstopsignatures
\sphinxAtStartPar
Bases: \sphinxcode{\sphinxupquote{Image}}

\sphinxAtStartPar
Tape class is used for preprocessing tape images for machine learning.

\sphinxAtStartPar
This class takes in a tape image, detects the edges, auto crops the image and
returns the results in 3 different methods: coordinate\_based, bin\_based,
and max\_contrast.
\begin{quote}\begin{description}
\sphinxlineitem{Parameters}\begin{itemize}
\item {} 
\sphinxAtStartPar
\sphinxstyleliteralstrong{\sphinxupquote{image}} (\sphinxstyleliteralemphasis{\sphinxupquote{np.ndarray}}) \textendash{} The image to be processed. It must be a 2D numpy array.

\item {} 
\sphinxAtStartPar
\sphinxstyleliteralstrong{\sphinxupquote{label}} (\sphinxstyleliteralemphasis{\sphinxupquote{str}}\sphinxstyleliteralemphasis{\sphinxupquote{, }}\sphinxstyleliteralemphasis{\sphinxupquote{optional}}) \textendash{} The label associated with the tape, by default None.

\item {} 
\sphinxAtStartPar
\sphinxstyleliteralstrong{\sphinxupquote{surface}} (\sphinxstyleliteralemphasis{\sphinxupquote{str}}\sphinxstyleliteralemphasis{\sphinxupquote{, }}\sphinxstyleliteralemphasis{\sphinxupquote{optional}}) \textendash{} The surface the tape is on, by default None.

\item {} 
\sphinxAtStartPar
\sphinxstyleliteralstrong{\sphinxupquote{stretched}} (\sphinxstyleliteralemphasis{\sphinxupquote{bool}}\sphinxstyleliteralemphasis{\sphinxupquote{, }}\sphinxstyleliteralemphasis{\sphinxupquote{optional}}) \textendash{} Flag indicating whether the tape is stretched or not, by default False.

\item {} 
\sphinxAtStartPar
\sphinxstyleliteralstrong{\sphinxupquote{**kwargs}} \textendash{} Arbitrary keyword arguments.

\end{itemize}

\end{description}\end{quote}
\index{metadata (forensicfit.core.tape.Tape attribute)@\spxentry{metadata}\spxextra{forensicfit.core.tape.Tape attribute}}

\begin{fulllineitems}
\phantomsection\label{\detokenize{forensicfit.core.tape:forensicfit.core.tape.Tape.metadata}}
\pysigstartsignatures
\pysigline{\sphinxbfcode{\sphinxupquote{metadata}}}
\pysigstopsignatures
\sphinxAtStartPar
A dictionary storing metadata of the image such as flipping information,
splitting information, label, material, surface, stretched status, and mode.
\begin{quote}\begin{description}
\sphinxlineitem{Type}
\sphinxAtStartPar
dict

\end{description}\end{quote}

\end{fulllineitems}

\begin{quote}\begin{description}
\sphinxlineitem{Raises}
\sphinxAtStartPar
\sphinxstyleliteralstrong{\sphinxupquote{AssertionError}} \textendash{} If \sphinxtitleref{image} is not a numpy array or not a 2D array.

\sphinxlineitem{Attributes}\begin{description}
\sphinxlineitem{\sphinxcode{\sphinxupquote{shape}}}
\sphinxAtStartPar
Returns the shape of the image.

\end{description}

\end{description}\end{quote}
\subsubsection*{Methods}


\begin{savenotes}\sphinxattablestart
\sphinxthistablewithglobalstyle
\sphinxthistablewithnovlinesstyle
\centering
\begin{tabulary}{\linewidth}[t]{\X{1}{2}\X{1}{2}}
\sphinxtoprule
\sphinxtableatstartofbodyhook
\sphinxAtStartPar
\sphinxcode{\sphinxupquote{apply\_filter}}(mode, **kwargs)
&
\sphinxAtStartPar
Applies different types of filters to the image
\\
\sphinxhline
\sphinxAtStartPar
\sphinxcode{\sphinxupquote{convert\_to\_gray}}()
&
\sphinxAtStartPar
Converts the image to grayscale.
\\
\sphinxhline
\sphinxAtStartPar
\sphinxcode{\sphinxupquote{convert\_to\_rgb}}()
&
\sphinxAtStartPar
Converts the image to RGB color space.
\\
\sphinxhline
\sphinxAtStartPar
\sphinxcode{\sphinxupquote{copy}}()
&
\sphinxAtStartPar
Creates a copy of the current image instance.
\\
\sphinxhline
\sphinxAtStartPar
\sphinxcode{\sphinxupquote{crop}}(x\_start, x\_end, y\_start, y\_end)
&
\sphinxAtStartPar
Crops the image using the specified x and y coordinates.
\\
\sphinxhline
\sphinxAtStartPar
\sphinxcode{\sphinxupquote{exposure\_control}}({[}mode{]})
&
\sphinxAtStartPar
modifies the exposure
\\
\sphinxhline
\sphinxAtStartPar
\sphinxcode{\sphinxupquote{flip\_h}}()
&
\sphinxAtStartPar
Flips the image horizontally.
\\
\sphinxhline
\sphinxAtStartPar
\sphinxcode{\sphinxupquote{flip\_v}}()
&
\sphinxAtStartPar
Flips the image vertically.
\\
\sphinxhline
\sphinxAtStartPar
\sphinxcode{\sphinxupquote{from\_buffer}}(buffer, metadata{[}, ext, ...{]})
&
\sphinxAtStartPar
Create an Image object from a given buffer and its corresponding metadata.
\\
\sphinxhline
\sphinxAtStartPar
\sphinxcode{\sphinxupquote{from\_dict}}(values, metadata)
&
\sphinxAtStartPar
Create an Image object from a dictionary of values and metadata.
\\
\sphinxhline
\sphinxAtStartPar
\sphinxcode{\sphinxupquote{from\_file}}(filepath)
&
\sphinxAtStartPar
Create an Image object from a given file path.
\\
\sphinxhline
\sphinxAtStartPar
\sphinxcode{\sphinxupquote{get}}(k{[},d{]})
&
\sphinxAtStartPar

\\
\sphinxhline
\sphinxAtStartPar
\sphinxcode{\sphinxupquote{isolate}}({[}x\_start, x\_end, y\_start, y\_end{]})
&
\sphinxAtStartPar
Isolates a rectangular section from the image.
\\
\sphinxhline
\sphinxAtStartPar
\sphinxcode{\sphinxupquote{items}}()
&
\sphinxAtStartPar

\\
\sphinxhline
\sphinxAtStartPar
\sphinxcode{\sphinxupquote{keys}}()
&
\sphinxAtStartPar

\\
\sphinxhline
\sphinxAtStartPar
\sphinxcode{\sphinxupquote{plot}}({[}savefig, cmap, ax, show, zoom{]})
&
\sphinxAtStartPar
Plots the current image.
\\
\sphinxhline
\sphinxAtStartPar
\sphinxcode{\sphinxupquote{resize}}({[}size, dpi{]})
&
\sphinxAtStartPar
Resizes the image according to the specified dimensions or dpi.
\\
\sphinxhline
\sphinxAtStartPar
\sphinxcode{\sphinxupquote{rotate}}(angle)
&
\sphinxAtStartPar
Rotates the image by a specified angle.
\\
\sphinxhline
\sphinxAtStartPar
\sphinxcode{\sphinxupquote{show}}({[}wait, savefig{]})
&
\sphinxAtStartPar
\begin{quote}\begin{description}
\sphinxlineitem{param wait}
\sphinxAtStartPar
DESCRIPTION. The default is 0.

\end{description}\end{quote}

\\
\sphinxhline
\sphinxAtStartPar
{\hyperref[\detokenize{forensicfit.core.tape:forensicfit.core.tape.Tape.split_v}]{\sphinxcrossref{\sphinxcode{\sphinxupquote{split\_v}}}}}(side{[}, correct\_tilt, pixel\_index{]})
&
\sphinxAtStartPar
Splits the tape image vertically.
\\
\sphinxhline
\sphinxAtStartPar
\sphinxcode{\sphinxupquote{to\_buffer}}({[}ext{]})
&
\sphinxAtStartPar
Converts the Image instance into a bytes object.
\\
\sphinxhline
\sphinxAtStartPar
\sphinxcode{\sphinxupquote{to\_file}}(filepath)
&
\sphinxAtStartPar
Writes the Image instance to a file.
\\
\sphinxhline
\sphinxAtStartPar
\sphinxcode{\sphinxupquote{values}}()
&
\sphinxAtStartPar

\\
\sphinxbottomrule
\end{tabulary}
\sphinxtableafterendhook\par
\sphinxattableend\end{savenotes}
\index{\_\_init\_\_() (forensicfit.core.tape.Tape method)@\spxentry{\_\_init\_\_()}\spxextra{forensicfit.core.tape.Tape method}}

\begin{fulllineitems}
\phantomsection\label{\detokenize{forensicfit.core.tape:forensicfit.core.tape.Tape.__init__}}
\pysigstartsignatures
\pysiglinewithargsret{\sphinxbfcode{\sphinxupquote{\_\_init\_\_}}}{\sphinxparam{\DUrole{n,n}{image}}, \sphinxparam{\DUrole{n,n}{label}\DUrole{o,o}{=}\DUrole{default_value}{None}}, \sphinxparam{\DUrole{n,n}{surface}\DUrole{o,o}{=}\DUrole{default_value}{None}}, \sphinxparam{\DUrole{n,n}{stretched}\DUrole{o,o}{=}\DUrole{default_value}{False}}, \sphinxparam{\DUrole{o,o}{**}\DUrole{n,n}{kwargs}}}{}
\pysigstopsignatures
\end{fulllineitems}

\index{split\_v() (forensicfit.core.tape.Tape method)@\spxentry{split\_v()}\spxextra{forensicfit.core.tape.Tape method}}

\begin{fulllineitems}
\phantomsection\label{\detokenize{forensicfit.core.tape:forensicfit.core.tape.Tape.split_v}}
\pysigstartsignatures
\pysiglinewithargsret{\sphinxbfcode{\sphinxupquote{split\_v}}}{\sphinxparam{\DUrole{n,n}{side}}, \sphinxparam{\DUrole{n,n}{correct\_tilt}\DUrole{o,o}{=}\DUrole{default_value}{True}}, \sphinxparam{\DUrole{n,n}{pixel\_index}\DUrole{o,o}{=}\DUrole{default_value}{None}}}{}
\pysigstopsignatures
\sphinxAtStartPar
Splits the tape image vertically.
\begin{quote}\begin{description}
\sphinxlineitem{Parameters}\begin{itemize}
\item {} 
\sphinxAtStartPar
\sphinxstyleliteralstrong{\sphinxupquote{side}} (\sphinxstyleliteralemphasis{\sphinxupquote{str}}) \textendash{} The side of the image to be processed.

\item {} 
\sphinxAtStartPar
\sphinxstyleliteralstrong{\sphinxupquote{correct\_tilt}} (\sphinxstyleliteralemphasis{\sphinxupquote{bool}}\sphinxstyleliteralemphasis{\sphinxupquote{, }}\sphinxstyleliteralemphasis{\sphinxupquote{optional}}) \textendash{} If set to True, tilt correction is applied on the image, by default True.

\item {} 
\sphinxAtStartPar
\sphinxstyleliteralstrong{\sphinxupquote{pixel\_index}} (\sphinxstyleliteralemphasis{\sphinxupquote{int}}\sphinxstyleliteralemphasis{\sphinxupquote{, }}\sphinxstyleliteralemphasis{\sphinxupquote{optional}}) \textendash{} The index of the pixel where the image is to be split, by default None.

\end{itemize}

\sphinxlineitem{Return type}
\sphinxAtStartPar
None

\end{description}\end{quote}
\subsubsection*{Notes}

\sphinxAtStartPar
Changes the ‘image’, ‘split\_v’, ‘image\_tilt’, ‘tilt\_corrected’, ‘resolution’ fields of the metadata attribute.

\end{fulllineitems}


\end{fulllineitems}

\index{TapeAnalyzer (class in forensicfit.core.tape)@\spxentry{TapeAnalyzer}\spxextra{class in forensicfit.core.tape}}

\begin{fulllineitems}
\phantomsection\label{\detokenize{forensicfit.core.tape:forensicfit.core.tape.TapeAnalyzer}}
\pysigstartsignatures
\pysiglinewithargsret{\sphinxbfcode{\sphinxupquote{class\DUrole{w,w}{  }}}\sphinxcode{\sphinxupquote{forensicfit.core.tape.}}\sphinxbfcode{\sphinxupquote{TapeAnalyzer}}}{\sphinxparam{\DUrole{n,n}{tape}}, \sphinxparam{\DUrole{n,n}{mask\_threshold}\DUrole{o,o}{=}\DUrole{default_value}{60}}, \sphinxparam{\DUrole{n,n}{gaussian\_blur}\DUrole{o,o}{=}\DUrole{default_value}{(15, 15)}}, \sphinxparam{\DUrole{n,n}{n\_divisions}\DUrole{o,o}{=}\DUrole{default_value}{6}}, \sphinxparam{\DUrole{n,n}{auto\_crop}\DUrole{o,o}{=}\DUrole{default_value}{False}}, \sphinxparam{\DUrole{n,n}{correct\_tilt}\DUrole{o,o}{=}\DUrole{default_value}{False}}, \sphinxparam{\DUrole{n,n}{padding}\DUrole{o,o}{=}\DUrole{default_value}{\textquotesingle{}tape\textquotesingle{}}}, \sphinxparam{\DUrole{n,n}{remove\_background}\DUrole{o,o}{=}\DUrole{default_value}{True}}}{}
\pysigstopsignatures
\sphinxAtStartPar
Bases: {\hyperref[\detokenize{forensicfit.core.analyzer:forensicfit.core.analyzer.Analyzer}]{\sphinxcrossref{\sphinxcode{\sphinxupquote{Analyzer}}}}}

\sphinxAtStartPar
The TapeAnalyzer class is a specialized Analyzer used to preprocess duct tape images.

\sphinxAtStartPar
This class is used to process images of duct tape to prepare them for machine learning tasks.
It includes functionality for Gaussian blur, auto\sphinxhyphen{}cropping, tilt correction, and background removal.
\begin{quote}\begin{description}
\sphinxlineitem{Parameters}\begin{itemize}
\item {} 
\sphinxAtStartPar
\sphinxstyleliteralstrong{\sphinxupquote{tape}} ({\hyperref[\detokenize{forensicfit.core.tape:forensicfit.core.tape.Tape}]{\sphinxcrossref{\sphinxstyleliteralemphasis{\sphinxupquote{Tape}}}}}\sphinxstyleliteralemphasis{\sphinxupquote{, }}\sphinxstyleliteralemphasis{\sphinxupquote{optional}}) \textendash{} An instance of the Tape class representing the tape image to be analyzed.

\item {} 
\sphinxAtStartPar
\sphinxstyleliteralstrong{\sphinxupquote{mask\_threshold}} (\sphinxstyleliteralemphasis{\sphinxupquote{int}}\sphinxstyleliteralemphasis{\sphinxupquote{, }}\sphinxstyleliteralemphasis{\sphinxupquote{optional}}) \textendash{} The threshold used for masking during the image preprocessing. The default is 60.

\item {} 
\sphinxAtStartPar
\sphinxstyleliteralstrong{\sphinxupquote{gaussian\_blur}} (\sphinxstyleliteralemphasis{\sphinxupquote{tuple}}\sphinxstyleliteralemphasis{\sphinxupquote{, }}\sphinxstyleliteralemphasis{\sphinxupquote{optional}}) \textendash{} The kernel size for the Gaussian blur applied during the preprocessing. The default is (15, 15).

\item {} 
\sphinxAtStartPar
\sphinxstyleliteralstrong{\sphinxupquote{n\_divisions}} (\sphinxstyleliteralemphasis{\sphinxupquote{int}}\sphinxstyleliteralemphasis{\sphinxupquote{, }}\sphinxstyleliteralemphasis{\sphinxupquote{optional}}) \textendash{} The number of divisions to be used in the analysis. The default is 6.

\item {} 
\sphinxAtStartPar
\sphinxstyleliteralstrong{\sphinxupquote{auto\_crop}} (\sphinxstyleliteralemphasis{\sphinxupquote{bool}}\sphinxstyleliteralemphasis{\sphinxupquote{, }}\sphinxstyleliteralemphasis{\sphinxupquote{optional}}) \textendash{} A flag indicating whether the image should be auto\sphinxhyphen{}cropped. The default is False.

\item {} 
\sphinxAtStartPar
\sphinxstyleliteralstrong{\sphinxupquote{correct\_tilt}} (\sphinxstyleliteralemphasis{\sphinxupquote{bool}}\sphinxstyleliteralemphasis{\sphinxupquote{, }}\sphinxstyleliteralemphasis{\sphinxupquote{optional}}) \textendash{} A flag indicating whether the image tilt should be corrected. The default is False.

\item {} 
\sphinxAtStartPar
\sphinxstyleliteralstrong{\sphinxupquote{padding}} (\sphinxstyleliteralemphasis{\sphinxupquote{str}}\sphinxstyleliteralemphasis{\sphinxupquote{, }}\sphinxstyleliteralemphasis{\sphinxupquote{optional}}) \textendash{} The padding method used in the analysis. The default is ‘tape’.

\item {} 
\sphinxAtStartPar
\sphinxstyleliteralstrong{\sphinxupquote{remove\_background}} (\sphinxstyleliteralemphasis{\sphinxupquote{bool}}\sphinxstyleliteralemphasis{\sphinxupquote{, }}\sphinxstyleliteralemphasis{\sphinxupquote{optional}}) \textendash{} A flag indicating whether the background should be removed from the image. The default is True.

\end{itemize}

\sphinxlineitem{Attributes}\begin{description}
\sphinxlineitem{\sphinxcode{\sphinxupquote{shape}}}
\sphinxAtStartPar
A property that provides the shape of the image contained in the Analyzer instance.

\sphinxlineitem{{\hyperref[\detokenize{forensicfit.core.tape:forensicfit.core.tape.TapeAnalyzer.x_interval}]{\sphinxcrossref{\sphinxcode{\sphinxupquote{x\_interval}}}}}}
\sphinxAtStartPar
Calculate the maximum x\sphinxhyphen{}coordinate value among the boundary points.

\sphinxlineitem{{\hyperref[\detokenize{forensicfit.core.tape:forensicfit.core.tape.TapeAnalyzer.xmax}]{\sphinxcrossref{\sphinxcode{\sphinxupquote{xmax}}}}}}
\sphinxAtStartPar
X coordinate of minimum pixel of the boundary

\sphinxlineitem{{\hyperref[\detokenize{forensicfit.core.tape:forensicfit.core.tape.TapeAnalyzer.xmin}]{\sphinxcrossref{\sphinxcode{\sphinxupquote{xmin}}}}}}
\sphinxAtStartPar
Calculate the minimum x\sphinxhyphen{}coordinate value among the boundary points.

\sphinxlineitem{{\hyperref[\detokenize{forensicfit.core.tape:forensicfit.core.tape.TapeAnalyzer.ymax}]{\sphinxcrossref{\sphinxcode{\sphinxupquote{ymax}}}}}}
\sphinxAtStartPar
Calculate the maximum y\sphinxhyphen{}coordinate value among the boundary points.

\sphinxlineitem{{\hyperref[\detokenize{forensicfit.core.tape:forensicfit.core.tape.TapeAnalyzer.ymin}]{\sphinxcrossref{\sphinxcode{\sphinxupquote{ymin}}}}}}
\sphinxAtStartPar
Calculate the minimum y\sphinxhyphen{}coordinate value among the boundary points.

\end{description}

\end{description}\end{quote}
\subsubsection*{Methods}


\begin{savenotes}\sphinxattablestart
\sphinxthistablewithglobalstyle
\sphinxthistablewithnovlinesstyle
\centering
\begin{tabulary}{\linewidth}[t]{\X{1}{2}\X{1}{2}}
\sphinxtoprule
\sphinxtableatstartofbodyhook
\sphinxAtStartPar
\sphinxcode{\sphinxupquote{apply\_filter}}(mode, **kwargs)
&
\sphinxAtStartPar

\\
\sphinxhline
\sphinxAtStartPar
{\hyperref[\detokenize{forensicfit.core.tape:forensicfit.core.tape.TapeAnalyzer.auto_crop_y}]{\sphinxcrossref{\sphinxcode{\sphinxupquote{auto\_crop\_y}}}}}()
&
\sphinxAtStartPar
Automatically crop the image in the y\sphinxhyphen{}direction.
\\
\sphinxhline
\sphinxAtStartPar
\sphinxcode{\sphinxupquote{exposure\_control}}({[}mode{]})
&
\sphinxAtStartPar

\\
\sphinxhline
\sphinxAtStartPar
{\hyperref[\detokenize{forensicfit.core.tape:forensicfit.core.tape.TapeAnalyzer.flip_h}]{\sphinxcrossref{\sphinxcode{\sphinxupquote{flip\_h}}}}}()
&
\sphinxAtStartPar
Flips the image horizontally and updates the relevant metadata.
\\
\sphinxhline
\sphinxAtStartPar
{\hyperref[\detokenize{forensicfit.core.tape:forensicfit.core.tape.TapeAnalyzer.flip_v}]{\sphinxcrossref{\sphinxcode{\sphinxupquote{flip\_v}}}}}()
&
\sphinxAtStartPar
Flips the image vertically and updates the relevant metadata.
\\
\sphinxhline
\sphinxAtStartPar
\sphinxcode{\sphinxupquote{from\_buffer}}(buffer, metadata{[}, ext{]})
&
\sphinxAtStartPar
Receives an io byte buffer with the corresponding metadata and creates an instance of the class.
\\
\sphinxhline
\sphinxAtStartPar
{\hyperref[\detokenize{forensicfit.core.tape:forensicfit.core.tape.TapeAnalyzer.from_dict}]{\sphinxcrossref{\sphinxcode{\sphinxupquote{from\_dict}}}}}(image, metadata)
&
\sphinxAtStartPar
Class method to create an instance of the TapeAnalyzer class from provided image data and metadata.
\\
\sphinxhline
\sphinxAtStartPar
{\hyperref[\detokenize{forensicfit.core.tape:forensicfit.core.tape.TapeAnalyzer.get_bin_based}]{\sphinxcrossref{\sphinxcode{\sphinxupquote{get\_bin\_based}}}}}({[}window\_background, ...{]})
&
\sphinxAtStartPar
This method returns the edges detected in the image segmented into a given number of bins.
\\
\sphinxhline
\sphinxAtStartPar
{\hyperref[\detokenize{forensicfit.core.tape:forensicfit.core.tape.TapeAnalyzer.get_coordinate_based}]{\sphinxcrossref{\sphinxcode{\sphinxupquote{get\_coordinate\_based}}}}}({[}n\_points, x\_trim\_param{]})
&
\sphinxAtStartPar
This method returns the coordinate\sphinxhyphen{}based representation of the detected edge in the image.
\\
\sphinxhline
\sphinxAtStartPar
{\hyperref[\detokenize{forensicfit.core.tape:forensicfit.core.tape.TapeAnalyzer.get_image_tilt}]{\sphinxcrossref{\sphinxcode{\sphinxupquote{get\_image\_tilt}}}}}({[}plot{]})
&
\sphinxAtStartPar
Calculate the tilt angle of the image.
\\
\sphinxhline
\sphinxAtStartPar
{\hyperref[\detokenize{forensicfit.core.tape:forensicfit.core.tape.TapeAnalyzer.get_max_contrast}]{\sphinxcrossref{\sphinxcode{\sphinxupquote{get\_max\_contrast}}}}}({[}window\_background, ...{]})
&
\sphinxAtStartPar
This method generates a binary image representation of the detected edge in the image.
\\
\sphinxhline
\sphinxAtStartPar
\sphinxcode{\sphinxupquote{load\_dict}}()
&
\sphinxAtStartPar
Abstract method for loading a dictionary.
\\
\sphinxhline
\sphinxAtStartPar
{\hyperref[\detokenize{forensicfit.core.tape:forensicfit.core.tape.TapeAnalyzer.load_metadata}]{\sphinxcrossref{\sphinxcode{\sphinxupquote{load\_metadata}}}}}()
&
\sphinxAtStartPar
Loads additional metadata about the image.
\\
\sphinxhline
\sphinxAtStartPar
\sphinxcode{\sphinxupquote{plot}}(which{[}, cmap, zoom, savefig, ax, show, ...{]})
&
\sphinxAtStartPar
Plots different kinds of data based on the given parameters.
\\
\sphinxhline
\sphinxAtStartPar
\sphinxcode{\sphinxupquote{plot\_boundary}}({[}savefig, color, ax, show{]})
&
\sphinxAtStartPar
Plots the detected boundary of the image.
\\
\sphinxhline
\sphinxAtStartPar
{\hyperref[\detokenize{forensicfit.core.tape:forensicfit.core.tape.TapeAnalyzer.preprocess}]{\sphinxcrossref{\sphinxcode{\sphinxupquote{preprocess}}}}}()
&
\sphinxAtStartPar
The preprocess method applies various image processing techniques to prepare the image for analysis.
\\
\sphinxhline
\sphinxAtStartPar
\sphinxcode{\sphinxupquote{resize}}({[}size, dpi{]})
&
\sphinxAtStartPar
Resize the image associated with this analyzer.
\\
\sphinxhline
\sphinxAtStartPar
\sphinxcode{\sphinxupquote{to\_buffer}}({[}ext{]})
&
\sphinxAtStartPar
Converts the current instance of the Analyzer class to a byte buffer, which can be useful for serialization or for writing to a file.
\\
\sphinxbottomrule
\end{tabulary}
\sphinxtableafterendhook\par
\sphinxattableend\end{savenotes}
\index{\_\_init\_\_() (forensicfit.core.tape.TapeAnalyzer method)@\spxentry{\_\_init\_\_()}\spxextra{forensicfit.core.tape.TapeAnalyzer method}}

\begin{fulllineitems}
\phantomsection\label{\detokenize{forensicfit.core.tape:forensicfit.core.tape.TapeAnalyzer.__init__}}
\pysigstartsignatures
\pysiglinewithargsret{\sphinxbfcode{\sphinxupquote{\_\_init\_\_}}}{\sphinxparam{\DUrole{n,n}{tape}}, \sphinxparam{\DUrole{n,n}{mask\_threshold}\DUrole{o,o}{=}\DUrole{default_value}{60}}, \sphinxparam{\DUrole{n,n}{gaussian\_blur}\DUrole{o,o}{=}\DUrole{default_value}{(15, 15)}}, \sphinxparam{\DUrole{n,n}{n\_divisions}\DUrole{o,o}{=}\DUrole{default_value}{6}}, \sphinxparam{\DUrole{n,n}{auto\_crop}\DUrole{o,o}{=}\DUrole{default_value}{False}}, \sphinxparam{\DUrole{n,n}{correct\_tilt}\DUrole{o,o}{=}\DUrole{default_value}{False}}, \sphinxparam{\DUrole{n,n}{padding}\DUrole{o,o}{=}\DUrole{default_value}{\textquotesingle{}tape\textquotesingle{}}}, \sphinxparam{\DUrole{n,n}{remove\_background}\DUrole{o,o}{=}\DUrole{default_value}{True}}}{}
\pysigstopsignatures
\end{fulllineitems}

\index{preprocess() (forensicfit.core.tape.TapeAnalyzer method)@\spxentry{preprocess()}\spxextra{forensicfit.core.tape.TapeAnalyzer method}}

\begin{fulllineitems}
\phantomsection\label{\detokenize{forensicfit.core.tape:forensicfit.core.tape.TapeAnalyzer.preprocess}}
\pysigstartsignatures
\pysiglinewithargsret{\sphinxbfcode{\sphinxupquote{preprocess}}}{}{}
\pysigstopsignatures
\sphinxAtStartPar
The preprocess method applies various image processing techniques to prepare the image for analysis.

\sphinxAtStartPar
This method performs several image preprocessing tasks including color inversion if necessary, conversion to grayscale, Gaussian blur, contour detection, and retrieval of the largest contour.
\begin{quote}\begin{description}
\sphinxlineitem{Parameters}\begin{itemize}
\item {} 
\sphinxAtStartPar
\sphinxstyleliteralstrong{\sphinxupquote{calculate\_tilt}} (\sphinxstyleliteralemphasis{\sphinxupquote{bool}}\sphinxstyleliteralemphasis{\sphinxupquote{, }}\sphinxstyleliteralemphasis{\sphinxupquote{optional}}) \textendash{} A flag indicating whether the image tilt should be calculated. The default is True.

\item {} 
\sphinxAtStartPar
\sphinxstyleliteralstrong{\sphinxupquote{auto\_crop}} (\sphinxstyleliteralemphasis{\sphinxupquote{bool}}\sphinxstyleliteralemphasis{\sphinxupquote{, }}\sphinxstyleliteralemphasis{\sphinxupquote{optional}}) \textendash{} A flag indicating whether the image should be auto\sphinxhyphen{}cropped. The default is True.

\end{itemize}

\sphinxlineitem{Return type}
\sphinxAtStartPar
None.

\end{description}\end{quote}
\subsubsection*{Notes}

\sphinxAtStartPar
The original image is processed and the largest contour of the processed image is stored in the metadata under the ‘boundary’ key.

\end{fulllineitems}

\index{flip\_v() (forensicfit.core.tape.TapeAnalyzer method)@\spxentry{flip\_v()}\spxextra{forensicfit.core.tape.TapeAnalyzer method}}

\begin{fulllineitems}
\phantomsection\label{\detokenize{forensicfit.core.tape:forensicfit.core.tape.TapeAnalyzer.flip_v}}
\pysigstartsignatures
\pysiglinewithargsret{\sphinxbfcode{\sphinxupquote{flip\_v}}}{}{}
\pysigstopsignatures
\sphinxAtStartPar
Flips the image vertically and updates the relevant metadata.

\sphinxAtStartPar
The flip\_v method flips the image vertically, i.e., around the y\sphinxhyphen{}axis. It also updates the associated
metadata such as the boundary, coordinates, slopes and dynamic positions if they are present in the metadata.
The ‘flip\_v’ metadata attribute is also toggled.
\begin{quote}\begin{description}
\sphinxlineitem{Parameters}
\sphinxAtStartPar
\sphinxstyleliteralstrong{\sphinxupquote{None}} \textendash{} 

\sphinxlineitem{Return type}
\sphinxAtStartPar
None

\end{description}\end{quote}
\subsubsection*{Notes}

\sphinxAtStartPar
The original image is flipped vertically, and the corresponding changes are reflected in the metadata. The ‘flip\_v’
metadata attribute is also toggled to reflect whether a vertical flip has been performed.

\end{fulllineitems}

\index{flip\_h() (forensicfit.core.tape.TapeAnalyzer method)@\spxentry{flip\_h()}\spxextra{forensicfit.core.tape.TapeAnalyzer method}}

\begin{fulllineitems}
\phantomsection\label{\detokenize{forensicfit.core.tape:forensicfit.core.tape.TapeAnalyzer.flip_h}}
\pysigstartsignatures
\pysiglinewithargsret{\sphinxbfcode{\sphinxupquote{flip\_h}}}{}{}
\pysigstopsignatures
\sphinxAtStartPar
Flips the image horizontally and updates the relevant metadata.

\sphinxAtStartPar
The flip\_h method flips the image horizontally, i.e., around the x\sphinxhyphen{}axis. It also updates the associated
metadata such as the boundary, coordinates, slopes and dynamic positions if they are present in the metadata.
The ‘flip\_h’ metadata attribute is also toggled.
\begin{quote}\begin{description}
\sphinxlineitem{Parameters}
\sphinxAtStartPar
\sphinxstyleliteralstrong{\sphinxupquote{None}} \textendash{} 

\sphinxlineitem{Return type}
\sphinxAtStartPar
None

\end{description}\end{quote}
\subsubsection*{Notes}

\sphinxAtStartPar
The original image is flipped horizontally, and the corresponding changes are reflected in the metadata. The ‘flip\_h’
metadata attribute is also toggled to reflect whether a horizontal flip has been performed.

\end{fulllineitems}

\index{load\_metadata() (forensicfit.core.tape.TapeAnalyzer method)@\spxentry{load\_metadata()}\spxextra{forensicfit.core.tape.TapeAnalyzer method}}

\begin{fulllineitems}
\phantomsection\label{\detokenize{forensicfit.core.tape:forensicfit.core.tape.TapeAnalyzer.load_metadata}}
\pysigstartsignatures
\pysiglinewithargsret{\sphinxbfcode{\sphinxupquote{load\_metadata}}}{}{}
\pysigstopsignatures
\sphinxAtStartPar
Loads additional metadata about the image.

\sphinxAtStartPar
This method retrieves the minimum and maximum values of the x and y coordinates
and their intervals from the TapeAnalyzer object, casts them to integers and stores them
in the metadata attribute.
\begin{quote}\begin{description}
\sphinxlineitem{Parameters}
\sphinxAtStartPar
\sphinxstyleliteralstrong{\sphinxupquote{None}} \textendash{} 

\sphinxlineitem{Return type}
\sphinxAtStartPar
None

\end{description}\end{quote}
\subsubsection*{Notes}

\sphinxAtStartPar
The original metadata of the TapeAnalyzer object is updated with the minimum and maximum x and y values
and their intervals, and these values are cast to integers.

\end{fulllineitems}

\index{from\_dict() (forensicfit.core.tape.TapeAnalyzer class method)@\spxentry{from\_dict()}\spxextra{forensicfit.core.tape.TapeAnalyzer class method}}

\begin{fulllineitems}
\phantomsection\label{\detokenize{forensicfit.core.tape:forensicfit.core.tape.TapeAnalyzer.from_dict}}
\pysigstartsignatures
\pysiglinewithargsret{\sphinxbfcode{\sphinxupquote{classmethod\DUrole{w,w}{  }}}\sphinxbfcode{\sphinxupquote{from\_dict}}}{\sphinxparam{\DUrole{n,n}{image}}, \sphinxparam{\DUrole{n,n}{metadata}}}{}
\pysigstopsignatures
\sphinxAtStartPar
Class method to create an instance of the TapeAnalyzer class from provided image data and metadata.
\begin{quote}\begin{description}
\sphinxlineitem{Parameters}\begin{itemize}
\item {} 
\sphinxAtStartPar
\sphinxstyleliteralstrong{\sphinxupquote{image}} (\sphinxstyleliteralemphasis{\sphinxupquote{np.ndarray}}) \textendash{} The image data to initialize the TapeAnalyzer instance.

\item {} 
\sphinxAtStartPar
\sphinxstyleliteralstrong{\sphinxupquote{metadata}} (\sphinxstyleliteralemphasis{\sphinxupquote{dict}}) \textendash{} Dictionary containing metadata for the TapeAnalyzer instance.

\end{itemize}

\sphinxlineitem{Raises}
\sphinxAtStartPar
\sphinxstyleliteralstrong{\sphinxupquote{Exception}} \textendash{} If the provided image is empty or None.

\sphinxlineitem{Returns}
\sphinxAtStartPar
An instance of TapeAnalyzer initialized with the provided image data and metadata.

\sphinxlineitem{Return type}
\sphinxAtStartPar
{\hyperref[\detokenize{forensicfit.core.tape:forensicfit.core.tape.TapeAnalyzer}]{\sphinxcrossref{TapeAnalyzer}}}

\end{description}\end{quote}
\subsubsection*{Notes}

\sphinxAtStartPar
This class method provides an alternative way to create an instance of the TapeAnalyzer class, particularly
when the necessary image data and metadata are available in advance.

\end{fulllineitems}

\index{get\_image\_tilt() (forensicfit.core.tape.TapeAnalyzer method)@\spxentry{get\_image\_tilt()}\spxextra{forensicfit.core.tape.TapeAnalyzer method}}

\begin{fulllineitems}
\phantomsection\label{\detokenize{forensicfit.core.tape:forensicfit.core.tape.TapeAnalyzer.get_image_tilt}}
\pysigstartsignatures
\pysiglinewithargsret{\sphinxbfcode{\sphinxupquote{get\_image\_tilt}}}{\sphinxparam{\DUrole{n,n}{plot}\DUrole{o,o}{=}\DUrole{default_value}{False}}}{}
\pysigstopsignatures
\sphinxAtStartPar
Calculate the tilt angle of the image.

\sphinxAtStartPar
This function calculates the tilt angle of the image by applying a linear fit to the upper and lower boundaries of the image.
It first divides the x\sphinxhyphen{}axis into ‘n\_divisions’ segments, then it finds the top and bottom boundaries by searching for points
within each segment that are within the top and bottom y\sphinxhyphen{}intervals respectively. For each segment, a linear fit is applied to
the found boundary points, resulting in a set of slopes. The process is done separately for the top and bottom boundaries.

\sphinxAtStartPar
For each set of slopes, the one with the smallest standard deviation of the y\sphinxhyphen{}coordinates is selected.
If no points are found in a segment, no slope is added for that segment.

\sphinxAtStartPar
The final tilt angle is the arctan of the average of the selected top and bottom slopes, converted to degrees.
\begin{quote}\begin{description}
\sphinxlineitem{Return type}
\sphinxAtStartPar
\sphinxcode{\sphinxupquote{float}}

\sphinxlineitem{Parameters}
\sphinxAtStartPar
\sphinxstyleliteralstrong{\sphinxupquote{plot}} (\sphinxstyleliteralemphasis{\sphinxupquote{bool}}\sphinxstyleliteralemphasis{\sphinxupquote{, }}\sphinxstyleliteralemphasis{\sphinxupquote{optional}}) \textendash{} If True, the function will plot the boundary conditions that were used for the fit.
The top boundary conditions are plotted in blue, and the bottom ones in red. Default is False.

\sphinxlineitem{Returns}
\sphinxAtStartPar
The tilt angle of the image in degrees.

\sphinxlineitem{Return type}
\sphinxAtStartPar
float

\end{description}\end{quote}
\subsubsection*{Notes}

\sphinxAtStartPar
If the standard deviation of the y\sphinxhyphen{}coordinates of the boundaries used for the fit exceeds 10,
the respective y\sphinxhyphen{}coordinate is set to the corresponding image boundary (y\_max for the top, y\_min for the bottom).
The function also updates the ‘crop\_y\_top’, ‘crop\_y\_bottom’ and ‘image\_tilt’ keys in the metadata of the TapeAnalyzer instance.

\end{fulllineitems}

\index{xmin (forensicfit.core.tape.TapeAnalyzer property)@\spxentry{xmin}\spxextra{forensicfit.core.tape.TapeAnalyzer property}}

\begin{fulllineitems}
\phantomsection\label{\detokenize{forensicfit.core.tape:forensicfit.core.tape.TapeAnalyzer.xmin}}
\pysigstartsignatures
\pysigline{\sphinxbfcode{\sphinxupquote{property\DUrole{w,w}{  }}}\sphinxbfcode{\sphinxupquote{xmin}}\sphinxbfcode{\sphinxupquote{\DUrole{p,p}{:}\DUrole{w,w}{  }int}}}
\pysigstopsignatures
\sphinxAtStartPar
Calculate the minimum x\sphinxhyphen{}coordinate value among the boundary points.

\sphinxAtStartPar
This function computes the minimum value along the x\sphinxhyphen{}axis (horizontal direction in image) from the set of
coordinates that form the boundary of the image. These boundaries have been identified
and stored in the ‘boundary’ attribute of the object, which is a numpy array of shape (N, 2),
where N is the number of boundary points and the 2 columns represent the x and y coordinates, respectively.
\begin{quote}\begin{description}
\sphinxlineitem{Returns}
\sphinxAtStartPar
The minimum x\sphinxhyphen{}coordinate value of the boundary points in the image.

\sphinxlineitem{Return type}
\sphinxAtStartPar
int

\end{description}\end{quote}
\subsubsection*{Notes}

\sphinxAtStartPar
The boundary points should be pre\sphinxhyphen{}calculated and stored in the ‘boundary’ attribute before
calling this function. If the ‘boundary’ attribute is not set, or if it is empty,
this function may raise an error or return an unexpected result.

\end{fulllineitems}

\index{xmax (forensicfit.core.tape.TapeAnalyzer property)@\spxentry{xmax}\spxextra{forensicfit.core.tape.TapeAnalyzer property}}

\begin{fulllineitems}
\phantomsection\label{\detokenize{forensicfit.core.tape:forensicfit.core.tape.TapeAnalyzer.xmax}}
\pysigstartsignatures
\pysigline{\sphinxbfcode{\sphinxupquote{property\DUrole{w,w}{  }}}\sphinxbfcode{\sphinxupquote{xmax}}\sphinxbfcode{\sphinxupquote{\DUrole{p,p}{:}\DUrole{w,w}{  }int}}}
\pysigstopsignatures
\sphinxAtStartPar
X coordinate of minimum pixel of the boundary
\begin{quote}\begin{description}
\sphinxlineitem{Returns}
\sphinxAtStartPar
\sphinxstylestrong{xmax} \textendash{} X coordinate of minimum pixel of the boundary

\sphinxlineitem{Return type}
\sphinxAtStartPar
int

\end{description}\end{quote}

\end{fulllineitems}

\index{x\_interval (forensicfit.core.tape.TapeAnalyzer property)@\spxentry{x\_interval}\spxextra{forensicfit.core.tape.TapeAnalyzer property}}

\begin{fulllineitems}
\phantomsection\label{\detokenize{forensicfit.core.tape:forensicfit.core.tape.TapeAnalyzer.x_interval}}
\pysigstartsignatures
\pysigline{\sphinxbfcode{\sphinxupquote{property\DUrole{w,w}{  }}}\sphinxbfcode{\sphinxupquote{x\_interval}}\sphinxbfcode{\sphinxupquote{\DUrole{p,p}{:}\DUrole{w,w}{  }int}}}
\pysigstopsignatures
\sphinxAtStartPar
Calculate the maximum x\sphinxhyphen{}coordinate value among the boundary points.

\sphinxAtStartPar
This function computes the maximum value along the x\sphinxhyphen{}axis (horizontal direction in image)
from the set of coordinates that form the boundary of the image. These boundaries have been
identified and stored in the ‘boundary’ attribute of the object, which is a numpy array of
shape (N, 2), where N is the number of boundary points and the 2 columns represent the x
and y coordinates, respectively.
\begin{quote}\begin{description}
\sphinxlineitem{Returns}
\sphinxAtStartPar
The maximum x\sphinxhyphen{}coordinate value of the boundary points in the image.

\sphinxlineitem{Return type}
\sphinxAtStartPar
int

\end{description}\end{quote}
\subsubsection*{Notes}

\sphinxAtStartPar
The boundary points should be pre\sphinxhyphen{}calculated and stored in the ‘boundary’ attribute before
calling this function. If the ‘boundary’ attribute is not set, or if it is empty,
this function may raise an error or return an unexpected result.

\end{fulllineitems}

\index{ymin (forensicfit.core.tape.TapeAnalyzer property)@\spxentry{ymin}\spxextra{forensicfit.core.tape.TapeAnalyzer property}}

\begin{fulllineitems}
\phantomsection\label{\detokenize{forensicfit.core.tape:forensicfit.core.tape.TapeAnalyzer.ymin}}
\pysigstartsignatures
\pysigline{\sphinxbfcode{\sphinxupquote{property\DUrole{w,w}{  }}}\sphinxbfcode{\sphinxupquote{ymin}}\sphinxbfcode{\sphinxupquote{\DUrole{p,p}{:}\DUrole{w,w}{  }int}}}
\pysigstopsignatures
\sphinxAtStartPar
Calculate the minimum y\sphinxhyphen{}coordinate value among the boundary points.

\sphinxAtStartPar
This function computes the minimum value along the y\sphinxhyphen{}axis (vertical direction in image)
from the set of coordinates that form the boundary of the image. These boundaries have been
identified and stored in the ‘boundary’ attribute of the object, which is a numpy array of
shape (N, 2), where N is the number of boundary points and the 2 columns represent the x
and y coordinates, respectively.
\begin{quote}\begin{description}
\sphinxlineitem{Returns}
\sphinxAtStartPar
The minimum y\sphinxhyphen{}coordinate value of the boundary points in the image.

\sphinxlineitem{Return type}
\sphinxAtStartPar
int

\end{description}\end{quote}
\subsubsection*{Notes}

\sphinxAtStartPar
The boundary points should be pre\sphinxhyphen{}calculated and stored in the ‘boundary’ attribute before
calling this function. If the ‘boundary’ attribute is not set, or if it is empty,
this function may raise an error or return an unexpected result.

\end{fulllineitems}

\index{ymax (forensicfit.core.tape.TapeAnalyzer property)@\spxentry{ymax}\spxextra{forensicfit.core.tape.TapeAnalyzer property}}

\begin{fulllineitems}
\phantomsection\label{\detokenize{forensicfit.core.tape:forensicfit.core.tape.TapeAnalyzer.ymax}}
\pysigstartsignatures
\pysigline{\sphinxbfcode{\sphinxupquote{property\DUrole{w,w}{  }}}\sphinxbfcode{\sphinxupquote{ymax}}\sphinxbfcode{\sphinxupquote{\DUrole{p,p}{:}\DUrole{w,w}{  }int}}}
\pysigstopsignatures
\sphinxAtStartPar
Calculate the maximum y\sphinxhyphen{}coordinate value among the boundary points.

\sphinxAtStartPar
This function computes the maximum value along the y\sphinxhyphen{}axis (vertical direction in image)
from the set of coordinates that form the boundary of the image. These boundaries have been
identified and stored in the ‘boundary’ attribute of the object, which is a numpy array of
shape (N, 2), where N is the number of boundary points and the 2 columns represent the x
and y coordinates, respectively.
\begin{quote}\begin{description}
\sphinxlineitem{Returns}
\sphinxAtStartPar
The maximum y\sphinxhyphen{}coordinate value of the boundary points in the image.

\sphinxlineitem{Return type}
\sphinxAtStartPar
int

\end{description}\end{quote}
\subsubsection*{Notes}

\sphinxAtStartPar
The boundary points should be pre\sphinxhyphen{}calculated and stored in the ‘boundary’ attribute before
calling this function. If the ‘boundary’ attribute is not set, or if it is empty,
this function may raise an error or return an unexpected result.

\end{fulllineitems}

\index{auto\_crop\_y() (forensicfit.core.tape.TapeAnalyzer method)@\spxentry{auto\_crop\_y()}\spxextra{forensicfit.core.tape.TapeAnalyzer method}}

\begin{fulllineitems}
\phantomsection\label{\detokenize{forensicfit.core.tape:forensicfit.core.tape.TapeAnalyzer.auto_crop_y}}
\pysigstartsignatures
\pysiglinewithargsret{\sphinxbfcode{\sphinxupquote{auto\_crop\_y}}}{}{}
\pysigstopsignatures
\sphinxAtStartPar
Automatically crop the image in the y\sphinxhyphen{}direction.

\sphinxAtStartPar
This function removes pixels from the top and bottom of the image based on the values stored
in \sphinxtitleref{metadata.crop\_y\_bottom} and \sphinxtitleref{metadata.crop\_y\_top}, respectively. This can be useful
for focusing on specific regions of interest in the image and removing unnecessary or
distracting parts of the image.
\subsubsection*{Notes}

\sphinxAtStartPar
The cropping limits are determined by the \sphinxtitleref{metadata.crop\_y\_bottom} and \sphinxtitleref{metadata.crop\_y\_top}
attributes. Before calling this method, these attributes should be calculated or set. If these
attributes are not set, the function may throw an error or return an unexpected result.

\sphinxAtStartPar
After cropping, the original image stored in the ‘image’ attribute is replaced by the cropped
image. If you need to keep the original image as well, you should create a copy before calling
this function.
\begin{quote}\begin{description}
\sphinxlineitem{Return type}
\sphinxAtStartPar
None

\end{description}\end{quote}

\end{fulllineitems}

\index{get\_coordinate\_based() (forensicfit.core.tape.TapeAnalyzer method)@\spxentry{get\_coordinate\_based()}\spxextra{forensicfit.core.tape.TapeAnalyzer method}}

\begin{fulllineitems}
\phantomsection\label{\detokenize{forensicfit.core.tape:forensicfit.core.tape.TapeAnalyzer.get_coordinate_based}}
\pysigstartsignatures
\pysiglinewithargsret{\sphinxbfcode{\sphinxupquote{get\_coordinate\_based}}}{\sphinxparam{\DUrole{n,n}{n\_points}\DUrole{o,o}{=}\DUrole{default_value}{64}}, \sphinxparam{\DUrole{n,n}{x\_trim\_param}\DUrole{o,o}{=}\DUrole{default_value}{6}}}{}
\pysigstopsignatures
\sphinxAtStartPar
This method returns the coordinate\sphinxhyphen{}based representation of the detected edge in the image.
The edge is segmented into ‘n\_points’ horizontal slices, and for each slice, the average
x and y coordinates of the edge points within that slice are calculated. The method can also
account for vertically flipped images.
\begin{quote}\begin{description}
\sphinxlineitem{Return type}
\sphinxAtStartPar
\sphinxcode{\sphinxupquote{dict}}

\sphinxlineitem{Parameters}\begin{itemize}
\item {} 
\sphinxAtStartPar
\sphinxstyleliteralstrong{\sphinxupquote{n\_points}} (\sphinxstyleliteralemphasis{\sphinxupquote{int}}\sphinxstyleliteralemphasis{\sphinxupquote{, }}\sphinxstyleliteralemphasis{\sphinxupquote{optional}}) \textendash{} The number of slices into which the edge is divided. The default is 64.

\item {} 
\sphinxAtStartPar
\sphinxstyleliteralstrong{\sphinxupquote{x\_trim\_param}} (\sphinxstyleliteralemphasis{\sphinxupquote{int}}\sphinxstyleliteralemphasis{\sphinxupquote{, }}\sphinxstyleliteralemphasis{\sphinxupquote{optional}}) \textendash{} A parameter to determine the range in the x\sphinxhyphen{}direction for edge point consideration.
The x\sphinxhyphen{}range is divided by this parameter to define the x\sphinxhyphen{}range for edge detection.
A smaller value of x\_trim\_param results in a larger x\sphinxhyphen{}range. The default is 6.

\end{itemize}

\sphinxlineitem{Returns}
\sphinxAtStartPar
A dictionary containing the calculated coordinate\sphinxhyphen{}based representation. The dictionary
includes the number of points, the x\_trim\_param used, and three numpy arrays: ‘coordinates’
with the average (x, y) coordinates for each slice, ‘stds’ with the standard deviations
of the x\sphinxhyphen{}coordinates within each slice, and ‘slopes’ with the slope and intercept of
the least\sphinxhyphen{}square linear fit to the edge points within each slice.

\sphinxlineitem{Return type}
\sphinxAtStartPar
dict

\end{description}\end{quote}

\end{fulllineitems}

\index{get\_bin\_based() (forensicfit.core.tape.TapeAnalyzer method)@\spxentry{get\_bin\_based()}\spxextra{forensicfit.core.tape.TapeAnalyzer method}}

\begin{fulllineitems}
\phantomsection\label{\detokenize{forensicfit.core.tape:forensicfit.core.tape.TapeAnalyzer.get_bin_based}}
\pysigstartsignatures
\pysiglinewithargsret{\sphinxbfcode{\sphinxupquote{get\_bin\_based}}}{\sphinxparam{\DUrole{n,n}{window\_background}\DUrole{o,o}{=}\DUrole{default_value}{50}}, \sphinxparam{\DUrole{n,n}{window\_tape}\DUrole{o,o}{=}\DUrole{default_value}{1000}}, \sphinxparam{\DUrole{n,n}{dynamic\_window}\DUrole{o,o}{=}\DUrole{default_value}{True}}, \sphinxparam{\DUrole{n,n}{n\_bins}\DUrole{o,o}{=}\DUrole{default_value}{10}}, \sphinxparam{\DUrole{n,n}{overlap}\DUrole{o,o}{=}\DUrole{default_value}{0}}, \sphinxparam{\DUrole{n,n}{border}\DUrole{o,o}{=}\DUrole{default_value}{\textquotesingle{}avg\textquotesingle{}}}}{}
\pysigstopsignatures
\sphinxAtStartPar
This method returns the edges detected in the image segmented into a given number of bins. Each bin is a slice
of the image, vertically defined and extending horizontally to include a given number of pixels on both sides
of the edge. The slices can have overlaps and their horizontal positions can be adjusted dynamically based on
the edge location within the slice.
\begin{quote}\begin{description}
\sphinxlineitem{Return type}
\sphinxAtStartPar
\sphinxcode{\sphinxupquote{List}}{[}\sphinxcode{\sphinxupquote{Tuple}}{[}\sphinxcode{\sphinxupquote{int}}, \sphinxcode{\sphinxupquote{int}}{]}{]}

\sphinxlineitem{Parameters}\begin{itemize}
\item {} 
\sphinxAtStartPar
\sphinxstyleliteralstrong{\sphinxupquote{window\_background}} (\sphinxstyleliteralemphasis{\sphinxupquote{int}}\sphinxstyleliteralemphasis{\sphinxupquote{, }}\sphinxstyleliteralemphasis{\sphinxupquote{optional}}) \textendash{} Number of pixels to be included in each slice on the background side of the edge. The default is 50.

\item {} 
\sphinxAtStartPar
\sphinxstyleliteralstrong{\sphinxupquote{window\_tape}} (\sphinxstyleliteralemphasis{\sphinxupquote{int}}\sphinxstyleliteralemphasis{\sphinxupquote{, }}\sphinxstyleliteralemphasis{\sphinxupquote{optional}}) \textendash{} Number of pixels to be included in each slice on the tape side of the edge. The default is 1000.

\item {} 
\sphinxAtStartPar
\sphinxstyleliteralstrong{\sphinxupquote{dynamic\_window}} (\sphinxstyleliteralemphasis{\sphinxupquote{bool}}\sphinxstyleliteralemphasis{\sphinxupquote{, }}\sphinxstyleliteralemphasis{\sphinxupquote{optional}}) \textendash{} Whether to adjust the horizontal position of the slices based on the edge location within the slice.
The default is True.

\item {} 
\sphinxAtStartPar
\sphinxstyleliteralstrong{\sphinxupquote{n\_bins}} (\sphinxstyleliteralemphasis{\sphinxupquote{int}}\sphinxstyleliteralemphasis{\sphinxupquote{, }}\sphinxstyleliteralemphasis{\sphinxupquote{optional}}) \textendash{} Number of slices into which the edge is divided. The default is 10.

\item {} 
\sphinxAtStartPar
\sphinxstyleliteralstrong{\sphinxupquote{overlap}} (\sphinxstyleliteralemphasis{\sphinxupquote{int}}\sphinxstyleliteralemphasis{\sphinxupquote{, }}\sphinxstyleliteralemphasis{\sphinxupquote{optional}}) \textendash{} The number of rows of overlap between consecutive slices. If positive, slices will overlap, if negative,
there will be gaps between them. The default is 0.

\item {} 
\sphinxAtStartPar
\sphinxstyleliteralstrong{\sphinxupquote{border}} (\sphinxstyleliteralemphasis{\sphinxupquote{str}}\sphinxstyleliteralemphasis{\sphinxupquote{, }}\sphinxstyleliteralemphasis{\sphinxupquote{optional}}) \textendash{} Determines the method of edge detection within each slice. Options are ‘avg’ for the average position
or ‘min’ for the minimum position of edge pixels in the slice. The default is ‘avg’.

\end{itemize}

\sphinxlineitem{Returns}
\sphinxAtStartPar
A list of tuples specifying the x\sphinxhyphen{}range and y\sphinxhyphen{}range of each slice in the format:
{[}(x\_start, x\_end), (y\_start, y\_end){]}

\sphinxlineitem{Return type}
\sphinxAtStartPar
List{[}Tuple{[}int, int{]}{]}

\end{description}\end{quote}

\end{fulllineitems}

\index{get\_max\_contrast() (forensicfit.core.tape.TapeAnalyzer method)@\spxentry{get\_max\_contrast()}\spxextra{forensicfit.core.tape.TapeAnalyzer method}}

\begin{fulllineitems}
\phantomsection\label{\detokenize{forensicfit.core.tape:forensicfit.core.tape.TapeAnalyzer.get_max_contrast}}
\pysigstartsignatures
\pysiglinewithargsret{\sphinxbfcode{\sphinxupquote{get\_max\_contrast}}}{\sphinxparam{\DUrole{n,n}{window\_background}\DUrole{o,o}{=}\DUrole{default_value}{100}}, \sphinxparam{\DUrole{n,n}{window\_tape}\DUrole{o,o}{=}\DUrole{default_value}{600}}}{}
\pysigstopsignatures
\sphinxAtStartPar
This method generates a binary image representation of the detected edge in the image.
It generates a new image where all pixels are set to 0 (black), except for the ones
at the detected edge location, which are set to 1 (white).

\sphinxAtStartPar
This can help in visualizing the edge or in performing further analysis, as the edge
is distinctly highlighted against a uniform background.
\begin{quote}\begin{description}
\sphinxlineitem{Return type}
\sphinxAtStartPar
\sphinxcode{\sphinxupquote{ndarray}}

\sphinxlineitem{Parameters}\begin{itemize}
\item {} 
\sphinxAtStartPar
\sphinxstyleliteralstrong{\sphinxupquote{window\_background}} (\sphinxstyleliteralemphasis{\sphinxupquote{int}}\sphinxstyleliteralemphasis{\sphinxupquote{, }}\sphinxstyleliteralemphasis{\sphinxupquote{optional}}) \textendash{} Number of pixels to be included in each slice on the background side of the edge.
These pixels will be set to black in the resulting image. The default is 100.

\item {} 
\sphinxAtStartPar
\sphinxstyleliteralstrong{\sphinxupquote{window\_tape}} (\sphinxstyleliteralemphasis{\sphinxupquote{int}}\sphinxstyleliteralemphasis{\sphinxupquote{, }}\sphinxstyleliteralemphasis{\sphinxupquote{optional}}) \textendash{} Number of pixels to be included in each slice on the tape side of the edge.
These pixels will be set to black in the resulting image. The default is 600.

\end{itemize}

\sphinxlineitem{Returns}
\sphinxAtStartPar
\sphinxstylestrong{edge\_bw} \textendash{} A 2D numpy array representing the image. All pixels are black (0), except the ones
at the detected edge location, which are white (1).

\sphinxlineitem{Return type}
\sphinxAtStartPar
np.ndarray

\end{description}\end{quote}

\end{fulllineitems}


\end{fulllineitems}



\subparagraph{Module contents}
\label{\detokenize{forensicfit.core:module-forensicfit.core}}\label{\detokenize{forensicfit.core:module-contents}}\index{module@\spxentry{module}!forensicfit.core@\spxentry{forensicfit.core}}\index{forensicfit.core@\spxentry{forensicfit.core}!module@\spxentry{module}}

\subparagraph{Core Module}
\label{\detokenize{forensicfit.core:core-module}}\label{\detokenize{forensicfit.core:id1}}
\sphinxAtStartPar
The core subpackage provides fundamental classes and functions needed for the operation of the ForensicFit package.


\subparagraph{Dependencies}
\label{\detokenize{forensicfit.core:dependencies}}
\sphinxAtStartPar
The core subpackage requires the following external libraries:
\begin{itemize}
\item {} 
\sphinxAtStartPar
OpenCV (\sphinxcode{\sphinxupquote{HAS\_OPENCV}})

\item {} 
\sphinxAtStartPar
PyMongo (\sphinxcode{\sphinxupquote{HAS\_PYMONGO}})

\end{itemize}

\sphinxAtStartPar
If these libraries are not installed, relevant functionalities might be disabled.


\subparagraph{Main Classes}
\label{\detokenize{forensicfit.core:main-classes}}
\sphinxAtStartPar
The core subpackage defines the following primary classes:
\begin{itemize}
\item {} 
\sphinxAtStartPar
\sphinxcode{\sphinxupquote{Image}} : Class for handling and manipulating images.

\item {} 
\sphinxAtStartPar
\sphinxcode{\sphinxupquote{Metadata}} : Class for handling metadata associated with images.

\item {} 
\sphinxAtStartPar
{\hyperref[\detokenize{forensicfit.core.analyzer:forensicfit.core.analyzer.Analyzer}]{\sphinxcrossref{\sphinxcode{\sphinxupquote{Analyzer}}}}} : Base class for implementing different types of image analysis.

\item {} 
\sphinxAtStartPar
{\hyperref[\detokenize{forensicfit.core.tape:forensicfit.core.tape.Tape}]{\sphinxcrossref{\sphinxcode{\sphinxupquote{Tape}}}}} : Class for creating and manipulating tape objects.

\item {} 
\sphinxAtStartPar
{\hyperref[\detokenize{forensicfit.core.tape:forensicfit.core.tape.TapeAnalyzer}]{\sphinxcrossref{\sphinxcode{\sphinxupquote{TapeAnalyzer}}}}} : Class that extends Analyzer, specialized in analyzing tape images.

\end{itemize}

\begin{sphinxadmonition}{note}{Note:}
\sphinxAtStartPar
These classes form the backbone of the ForensicFit package and are used throughout its various functions and methods.
\end{sphinxadmonition}

\sphinxstepscope


\paragraph{forensicfit.database package}
\label{\detokenize{forensicfit.database:forensicfit-database-package}}\label{\detokenize{forensicfit.database::doc}}

\subparagraph{Submodules}
\label{\detokenize{forensicfit.database:submodules}}
\sphinxstepscope


\subparagraph{forensicfit.database.database module}
\label{\detokenize{forensicfit.database.database:module-forensicfit.database.database}}\label{\detokenize{forensicfit.database.database:forensicfit-database-database-module}}\label{\detokenize{forensicfit.database.database::doc}}\index{module@\spxentry{module}!forensicfit.database.database@\spxentry{forensicfit.database.database}}\index{forensicfit.database.database@\spxentry{forensicfit.database.database}!module@\spxentry{module}}
\sphinxAtStartPar
Created on Mon Apr 12 09:36:25 2021

\sphinxAtStartPar
@author: Pedram Tavadze
used PyChemia Database class as a guide
\sphinxurl{https://github.com/MaterialsDiscovery/PyChemia/blob/master/pychemia/db/db.py}
\index{ClassMap (class in forensicfit.database.database)@\spxentry{ClassMap}\spxextra{class in forensicfit.database.database}}

\begin{fulllineitems}
\phantomsection\label{\detokenize{forensicfit.database.database:forensicfit.database.database.ClassMap}}
\pysigstartsignatures
\pysigline{\sphinxbfcode{\sphinxupquote{class\DUrole{w,w}{  }}}\sphinxcode{\sphinxupquote{forensicfit.database.database.}}\sphinxbfcode{\sphinxupquote{ClassMap}}}
\pysigstopsignatures
\sphinxAtStartPar
Bases: \sphinxcode{\sphinxupquote{Mapping}}

\sphinxAtStartPar
A custom mapping class that maps string keys to specific classes.

\sphinxAtStartPar
This class is a subclass of \sphinxtitleref{collections.abc.Mapping} and provides a custom mapping
between string keys and classes. The keys ‘material’, ‘analysis’, and ‘any’ are mapped
to the classes \sphinxtitleref{Tape}, \sphinxtitleref{TapeAnalyzer}, and \sphinxtitleref{Image} respectively.
\index{mapping (forensicfit.database.database.ClassMap attribute)@\spxentry{mapping}\spxextra{forensicfit.database.database.ClassMap attribute}}

\begin{fulllineitems}
\phantomsection\label{\detokenize{forensicfit.database.database:forensicfit.database.database.ClassMap.mapping}}
\pysigstartsignatures
\pysigline{\sphinxbfcode{\sphinxupquote{mapping}}}
\pysigstopsignatures
\sphinxAtStartPar
The internal dictionary that stores the mapping between keys and classes.
\begin{quote}\begin{description}
\sphinxlineitem{Type}
\sphinxAtStartPar
dict

\end{description}\end{quote}

\end{fulllineitems}

\index{\_\_contains\_\_() (forensicfit.database.database.ClassMap method)@\spxentry{\_\_contains\_\_()}\spxextra{forensicfit.database.database.ClassMap method}}

\begin{fulllineitems}
\phantomsection\label{\detokenize{forensicfit.database.database:forensicfit.database.database.ClassMap.__contains__}}
\pysigstartsignatures
\pysiglinewithargsret{\sphinxbfcode{\sphinxupquote{\_\_contains\_\_}}}{\sphinxparam{\DUrole{n,n}{x}\DUrole{p,p}{:}\DUrole{w,w}{  }\DUrole{n,n}{str}}}{{ $\rightarrow$ bool}}
\pysigstopsignatures
\sphinxAtStartPar
Check if \sphinxtitleref{x} is a key in the mapping.

\end{fulllineitems}

\index{\_\_getitem\_\_() (forensicfit.database.database.ClassMap method)@\spxentry{\_\_getitem\_\_()}\spxextra{forensicfit.database.database.ClassMap method}}

\begin{fulllineitems}
\phantomsection\label{\detokenize{forensicfit.database.database:forensicfit.database.database.ClassMap.__getitem__}}
\pysigstartsignatures
\pysiglinewithargsret{\sphinxbfcode{\sphinxupquote{\_\_getitem\_\_}}}{\sphinxparam{\DUrole{n,n}{x}\DUrole{p,p}{:}\DUrole{w,w}{  }\DUrole{n,n}{str}}}{{ $\rightarrow$ {\hyperref[\detokenize{forensicfit.core.tape:forensicfit.core.tape.Tape}]{\sphinxcrossref{Tape}}}\DUrole{w,w}{  }\DUrole{p,p}{|}\DUrole{w,w}{  }{\hyperref[\detokenize{forensicfit.core.tape:forensicfit.core.tape.TapeAnalyzer}]{\sphinxcrossref{TapeAnalyzer}}}\DUrole{w,w}{  }\DUrole{p,p}{|}\DUrole{w,w}{  }Image}}
\pysigstopsignatures
\sphinxAtStartPar
Get the class associated with the key \sphinxtitleref{x}. If \sphinxtitleref{x} is not a key in the mapping, return the class associated with the key ‘any’.

\end{fulllineitems}

\index{\_\_iter\_\_() (forensicfit.database.database.ClassMap method)@\spxentry{\_\_iter\_\_()}\spxextra{forensicfit.database.database.ClassMap method}}

\begin{fulllineitems}
\phantomsection\label{\detokenize{forensicfit.database.database:forensicfit.database.database.ClassMap.__iter__}}
\pysigstartsignatures
\pysiglinewithargsret{\sphinxbfcode{\sphinxupquote{\_\_iter\_\_}}}{}{{ $\rightarrow$ Iterator}}
\pysigstopsignatures
\sphinxAtStartPar
Return an iterator over the keys in the mapping.

\end{fulllineitems}

\index{\_\_len\_\_() (forensicfit.database.database.ClassMap method)@\spxentry{\_\_len\_\_()}\spxextra{forensicfit.database.database.ClassMap method}}

\begin{fulllineitems}
\phantomsection\label{\detokenize{forensicfit.database.database:forensicfit.database.database.ClassMap.__len__}}
\pysigstartsignatures
\pysiglinewithargsret{\sphinxbfcode{\sphinxupquote{\_\_len\_\_}}}{}{{ $\rightarrow$ int}}
\pysigstopsignatures
\sphinxAtStartPar
Return the number of key\sphinxhyphen{}value pairs in the mapping.

\end{fulllineitems}

\subsubsection*{Methods}


\begin{savenotes}\sphinxattablestart
\sphinxthistablewithglobalstyle
\sphinxthistablewithnovlinesstyle
\centering
\begin{tabulary}{\linewidth}[t]{\X{1}{2}\X{1}{2}}
\sphinxtoprule
\sphinxtableatstartofbodyhook
\sphinxAtStartPar
\sphinxcode{\sphinxupquote{get}}(k{[},d{]})
&
\sphinxAtStartPar

\\
\sphinxhline
\sphinxAtStartPar
\sphinxcode{\sphinxupquote{items}}()
&
\sphinxAtStartPar

\\
\sphinxhline
\sphinxAtStartPar
\sphinxcode{\sphinxupquote{keys}}()
&
\sphinxAtStartPar

\\
\sphinxhline
\sphinxAtStartPar
\sphinxcode{\sphinxupquote{values}}()
&
\sphinxAtStartPar

\\
\sphinxbottomrule
\end{tabulary}
\sphinxtableafterendhook\par
\sphinxattableend\end{savenotes}

\sphinxAtStartPar
Initialize a new instance of ClassMap.
\subsubsection*{Methods}


\begin{savenotes}\sphinxattablestart
\sphinxthistablewithglobalstyle
\sphinxthistablewithnovlinesstyle
\centering
\begin{tabulary}{\linewidth}[t]{\X{1}{2}\X{1}{2}}
\sphinxtoprule
\sphinxtableatstartofbodyhook
\sphinxAtStartPar
\sphinxcode{\sphinxupquote{get}}(k{[},d{]})
&
\sphinxAtStartPar

\\
\sphinxhline
\sphinxAtStartPar
\sphinxcode{\sphinxupquote{items}}()
&
\sphinxAtStartPar

\\
\sphinxhline
\sphinxAtStartPar
\sphinxcode{\sphinxupquote{keys}}()
&
\sphinxAtStartPar

\\
\sphinxhline
\sphinxAtStartPar
\sphinxcode{\sphinxupquote{values}}()
&
\sphinxAtStartPar

\\
\sphinxbottomrule
\end{tabulary}
\sphinxtableafterendhook\par
\sphinxattableend\end{savenotes}
\index{\_\_init\_\_() (forensicfit.database.database.ClassMap method)@\spxentry{\_\_init\_\_()}\spxextra{forensicfit.database.database.ClassMap method}}

\begin{fulllineitems}
\phantomsection\label{\detokenize{forensicfit.database.database:forensicfit.database.database.ClassMap.__init__}}
\pysigstartsignatures
\pysiglinewithargsret{\sphinxbfcode{\sphinxupquote{\_\_init\_\_}}}{}{}
\pysigstopsignatures
\sphinxAtStartPar
Initialize a new instance of ClassMap.

\end{fulllineitems}


\end{fulllineitems}

\index{Database (class in forensicfit.database.database)@\spxentry{Database}\spxextra{class in forensicfit.database.database}}

\begin{fulllineitems}
\phantomsection\label{\detokenize{forensicfit.database.database:forensicfit.database.database.Database}}
\pysigstartsignatures
\pysiglinewithargsret{\sphinxbfcode{\sphinxupquote{class\DUrole{w,w}{  }}}\sphinxcode{\sphinxupquote{forensicfit.database.database.}}\sphinxbfcode{\sphinxupquote{Database}}}{\sphinxparam{\DUrole{n,n}{name}\DUrole{o,o}{=}\DUrole{default_value}{\textquotesingle{}forensicfit\textquotesingle{}}}, \sphinxparam{\DUrole{n,n}{host}\DUrole{o,o}{=}\DUrole{default_value}{\textquotesingle{}localhost\textquotesingle{}}}, \sphinxparam{\DUrole{n,n}{port}\DUrole{o,o}{=}\DUrole{default_value}{27017}}, \sphinxparam{\DUrole{n,n}{username}\DUrole{o,o}{=}\DUrole{default_value}{\textquotesingle{}\textquotesingle{}}}, \sphinxparam{\DUrole{n,n}{password}\DUrole{o,o}{=}\DUrole{default_value}{\textquotesingle{}\textquotesingle{}}}, \sphinxparam{\DUrole{n,n}{verbose}\DUrole{o,o}{=}\DUrole{default_value}{False}}, \sphinxparam{\DUrole{o,o}{**}\DUrole{n,n}{kwargs}}}{}
\pysigstopsignatures
\sphinxAtStartPar
Bases: \sphinxcode{\sphinxupquote{object}}

\phantomsection\label{\detokenize{forensicfit.database.database:database}}\begin{quote}\begin{description}
\sphinxlineitem{Attributes}\begin{description}
\sphinxlineitem{{\hyperref[\detokenize{forensicfit.database.database:forensicfit.database.database.Database.collection_names}]{\sphinxcrossref{\sphinxcode{\sphinxupquote{collection\_names}}}}}}
\sphinxAtStartPar
Get the names of the collections in the MongoDB database.

\sphinxlineitem{{\hyperref[\detokenize{forensicfit.database.database:forensicfit.database.database.Database.connected}]{\sphinxcrossref{\sphinxcode{\sphinxupquote{connected}}}}}}
\sphinxAtStartPar
Check if the Database instance is currently connected to the MongoDB database.

\sphinxlineitem{{\hyperref[\detokenize{forensicfit.database.database:forensicfit.database.database.Database.server_info}]{\sphinxcrossref{\sphinxcode{\sphinxupquote{server\_info}}}}}}
\sphinxAtStartPar
Get the server information for the MongoDB database.

\end{description}

\end{description}\end{quote}
\subsubsection*{Methods}


\begin{savenotes}\sphinxattablestart
\sphinxthistablewithglobalstyle
\sphinxthistablewithnovlinesstyle
\centering
\begin{tabulary}{\linewidth}[t]{\X{1}{2}\X{1}{2}}
\sphinxtoprule
\sphinxtableatstartofbodyhook
\sphinxAtStartPar
{\hyperref[\detokenize{forensicfit.database.database:forensicfit.database.database.Database.add_collection}]{\sphinxcrossref{\sphinxcode{\sphinxupquote{add\_collection}}}}}(collection)
&
\sphinxAtStartPar
Add a new collection to the database.
\\
\sphinxhline
\sphinxAtStartPar
{\hyperref[\detokenize{forensicfit.database.database:forensicfit.database.database.Database.count_documents}]{\sphinxcrossref{\sphinxcode{\sphinxupquote{count\_documents}}}}}(filter, collection)
&
\sphinxAtStartPar
Count the number of documents in the specified collection that match the provided filter.
\\
\sphinxhline
\sphinxAtStartPar
{\hyperref[\detokenize{forensicfit.database.database:forensicfit.database.database.Database.delete}]{\sphinxcrossref{\sphinxcode{\sphinxupquote{delete}}}}}(filter, collection)
&
\sphinxAtStartPar
Delete documents from the specified collection that match the provided filter.
\\
\sphinxhline
\sphinxAtStartPar
{\hyperref[\detokenize{forensicfit.database.database:forensicfit.database.database.Database.delete_database}]{\sphinxcrossref{\sphinxcode{\sphinxupquote{delete\_database}}}}}()
&
\sphinxAtStartPar
Delete the MongoDB database associated with this Database instance.
\\
\sphinxhline
\sphinxAtStartPar
{\hyperref[\detokenize{forensicfit.database.database:forensicfit.database.database.Database.disconnect}]{\sphinxcrossref{\sphinxcode{\sphinxupquote{disconnect}}}}}()
&
\sphinxAtStartPar
\phantomsection\label{\detokenize{forensicfit.database.database:disconnect}}
\\
\sphinxhline
\sphinxAtStartPar
{\hyperref[\detokenize{forensicfit.database.database:forensicfit.database.database.Database.drop_collection}]{\sphinxcrossref{\sphinxcode{\sphinxupquote{drop\_collection}}}}}(collection)
&
\sphinxAtStartPar
Drop the specified collection from the MongoDB database.
\\
\sphinxhline
\sphinxAtStartPar
{\hyperref[\detokenize{forensicfit.database.database:forensicfit.database.database.Database.exists}]{\sphinxcrossref{\sphinxcode{\sphinxupquote{exists}}}}}({[}filter, collection, metadata{]})
&
\sphinxAtStartPar
Check if a document exists in the specified collection based on the provided filter or metadata.
\\
\sphinxhline
\sphinxAtStartPar
{\hyperref[\detokenize{forensicfit.database.database:forensicfit.database.database.Database.export_to_files}]{\sphinxcrossref{\sphinxcode{\sphinxupquote{export\_to\_files}}}}}(destination, filter, collection)
&
\sphinxAtStartPar
Export objects from the specified collection that match the provided filter to files.
\\
\sphinxhline
\sphinxAtStartPar
{\hyperref[\detokenize{forensicfit.database.database:forensicfit.database.database.Database.filter_with_metadata}]{\sphinxcrossref{\sphinxcode{\sphinxupquote{filter\_with\_metadata}}}}}(inp, filter, collection)
&
\sphinxAtStartPar
Filter a list of filenames based on metadata and return the indices of the matching filenames.
\\
\sphinxhline
\sphinxAtStartPar
{\hyperref[\detokenize{forensicfit.database.database:forensicfit.database.database.Database.find}]{\sphinxcrossref{\sphinxcode{\sphinxupquote{find}}}}}(filter{[}, collection, ext, version, ...{]})
&
\sphinxAtStartPar
Find and return objects from the specified collection that match the provided filter.
\\
\sphinxhline
\sphinxAtStartPar
{\hyperref[\detokenize{forensicfit.database.database:forensicfit.database.database.Database.find_one}]{\sphinxcrossref{\sphinxcode{\sphinxupquote{find\_one}}}}}({[}filter, collection{]})
&
\sphinxAtStartPar
Find and return one object from the specified collection that matches the provided filter.
\\
\sphinxhline
\sphinxAtStartPar
{\hyperref[\detokenize{forensicfit.database.database:forensicfit.database.database.Database.find_with_id}]{\sphinxcrossref{\sphinxcode{\sphinxupquote{find\_with\_id}}}}}(\_id, collection)
&
\sphinxAtStartPar
Find and return an object from the specified collection based on its MongoDB \_id.
\\
\sphinxhline
\sphinxAtStartPar
{\hyperref[\detokenize{forensicfit.database.database:forensicfit.database.database.Database.insert}]{\sphinxcrossref{\sphinxcode{\sphinxupquote{insert}}}}}(obj{[}, ext, overwrite, skip, collection{]})
&
\sphinxAtStartPar
Insert an object into the specified collection in the MongoDB database.
\\
\sphinxhline
\sphinxAtStartPar
{\hyperref[\detokenize{forensicfit.database.database:forensicfit.database.database.Database.map_to}]{\sphinxcrossref{\sphinxcode{\sphinxupquote{map\_to}}}}}(func, filter, collection\_source, ...)
&
\sphinxAtStartPar
Apply a function to each object in the source collection that matches the provided filter and insert the results into the target collection.
\\
\sphinxbottomrule
\end{tabulary}
\sphinxtableafterendhook\par
\sphinxattableend\end{savenotes}
\index{\_\_init\_\_() (forensicfit.database.database.Database method)@\spxentry{\_\_init\_\_()}\spxextra{forensicfit.database.database.Database method}}

\begin{fulllineitems}
\phantomsection\label{\detokenize{forensicfit.database.database:forensicfit.database.database.Database.__init__}}
\pysigstartsignatures
\pysiglinewithargsret{\sphinxbfcode{\sphinxupquote{\_\_init\_\_}}}{\sphinxparam{\DUrole{n,n}{name}\DUrole{o,o}{=}\DUrole{default_value}{\textquotesingle{}forensicfit\textquotesingle{}}}, \sphinxparam{\DUrole{n,n}{host}\DUrole{o,o}{=}\DUrole{default_value}{\textquotesingle{}localhost\textquotesingle{}}}, \sphinxparam{\DUrole{n,n}{port}\DUrole{o,o}{=}\DUrole{default_value}{27017}}, \sphinxparam{\DUrole{n,n}{username}\DUrole{o,o}{=}\DUrole{default_value}{\textquotesingle{}\textquotesingle{}}}, \sphinxparam{\DUrole{n,n}{password}\DUrole{o,o}{=}\DUrole{default_value}{\textquotesingle{}\textquotesingle{}}}, \sphinxparam{\DUrole{n,n}{verbose}\DUrole{o,o}{=}\DUrole{default_value}{False}}, \sphinxparam{\DUrole{o,o}{**}\DUrole{n,n}{kwargs}}}{}
\pysigstopsignatures
\end{fulllineitems}

\index{disconnect() (forensicfit.database.database.Database method)@\spxentry{disconnect()}\spxextra{forensicfit.database.database.Database method}}

\begin{fulllineitems}
\phantomsection\label{\detokenize{forensicfit.database.database:forensicfit.database.database.Database.disconnect}}
\pysigstartsignatures
\pysiglinewithargsret{\sphinxbfcode{\sphinxupquote{disconnect}}}{}{}
\pysigstopsignatures\phantomsection\label{\detokenize{forensicfit.database.database:id1}}
\end{fulllineitems}

\index{add\_collection() (forensicfit.database.database.Database method)@\spxentry{add\_collection()}\spxextra{forensicfit.database.database.Database method}}

\begin{fulllineitems}
\phantomsection\label{\detokenize{forensicfit.database.database:forensicfit.database.database.Database.add_collection}}
\pysigstartsignatures
\pysiglinewithargsret{\sphinxbfcode{\sphinxupquote{add\_collection}}}{\sphinxparam{\DUrole{n,n}{collection}}}{}
\pysigstopsignatures
\sphinxAtStartPar
Add a new collection to the database.

\sphinxAtStartPar
This method creates a new GridFS instance for the specified collection
and adds it to the \sphinxtitleref{fs} attribute.
\begin{quote}\begin{description}
\sphinxlineitem{Parameters}
\sphinxAtStartPar
\sphinxstyleliteralstrong{\sphinxupquote{collection}} (\sphinxstyleliteralemphasis{\sphinxupquote{str}}) \textendash{} The name of the collection to add.

\sphinxlineitem{Return type}
\sphinxAtStartPar
None

\end{description}\end{quote}

\end{fulllineitems}

\index{exists() (forensicfit.database.database.Database method)@\spxentry{exists()}\spxextra{forensicfit.database.database.Database method}}

\begin{fulllineitems}
\phantomsection\label{\detokenize{forensicfit.database.database:forensicfit.database.database.Database.exists}}
\pysigstartsignatures
\pysiglinewithargsret{\sphinxbfcode{\sphinxupquote{exists}}}{\sphinxparam{\DUrole{n,n}{filter}\DUrole{o,o}{=}\DUrole{default_value}{None}}, \sphinxparam{\DUrole{n,n}{collection}\DUrole{o,o}{=}\DUrole{default_value}{None}}, \sphinxparam{\DUrole{n,n}{metadata}\DUrole{o,o}{=}\DUrole{default_value}{None}}}{}
\pysigstopsignatures
\sphinxAtStartPar
Check if a document exists in the specified collection based on the provided filter or metadata.

\sphinxAtStartPar
This method checks if a document exists in the specified collection of the MongoDB database
that matches the provided filter or metadata. If a document is found, it returns the ObjectId
of the document. If no document is found, it returns False.
\begin{quote}\begin{description}
\sphinxlineitem{Return type}
\sphinxAtStartPar
\sphinxcode{\sphinxupquote{Union}}{[}\sphinxcode{\sphinxupquote{ObjectId}}, \sphinxcode{\sphinxupquote{bool}}{]}

\sphinxlineitem{Parameters}\begin{itemize}
\item {} 
\sphinxAtStartPar
\sphinxstyleliteralstrong{\sphinxupquote{filter}} (\sphinxstyleliteralemphasis{\sphinxupquote{dict}}\sphinxstyleliteralemphasis{\sphinxupquote{, }}\sphinxstyleliteralemphasis{\sphinxupquote{optional}}) \textendash{} A dictionary specifying the filter criteria to use when searching for the document.

\item {} 
\sphinxAtStartPar
\sphinxstyleliteralstrong{\sphinxupquote{collection}} (\sphinxstyleliteralemphasis{\sphinxupquote{str}}\sphinxstyleliteralemphasis{\sphinxupquote{, }}\sphinxstyleliteralemphasis{\sphinxupquote{optional}}) \textendash{} The name of the collection to search in. If not provided, the collection name is
determined from the ‘mode’ field of the metadata.

\item {} 
\sphinxAtStartPar
\sphinxstyleliteralstrong{\sphinxupquote{metadata}} (\sphinxstyleliteralemphasis{\sphinxupquote{Metadata}}\sphinxstyleliteralemphasis{\sphinxupquote{, }}\sphinxstyleliteralemphasis{\sphinxupquote{optional}}) \textendash{} A Metadata object specifying the metadata to use when searching for the document.
If provided, the ‘mode’ field of the metadata is used as the collection name and
the metadata is converted to a MongoDB filter.

\end{itemize}

\sphinxlineitem{Returns}
\sphinxAtStartPar
The ObjectId of the found document if a matching document is found, False otherwise.

\sphinxlineitem{Return type}
\sphinxAtStartPar
ObjectId or bool

\sphinxlineitem{Raises}
\sphinxAtStartPar
\sphinxstyleliteralstrong{\sphinxupquote{Exception}} \textendash{} If neither metadata nor filter and collection are provided.

\end{description}\end{quote}

\end{fulllineitems}

\index{insert() (forensicfit.database.database.Database method)@\spxentry{insert()}\spxextra{forensicfit.database.database.Database method}}

\begin{fulllineitems}
\phantomsection\label{\detokenize{forensicfit.database.database:forensicfit.database.database.Database.insert}}
\pysigstartsignatures
\pysiglinewithargsret{\sphinxbfcode{\sphinxupquote{insert}}}{\sphinxparam{\DUrole{n,n}{obj}}, \sphinxparam{\DUrole{n,n}{ext}\DUrole{o,o}{=}\DUrole{default_value}{\textquotesingle{}.png\textquotesingle{}}}, \sphinxparam{\DUrole{n,n}{overwrite}\DUrole{o,o}{=}\DUrole{default_value}{False}}, \sphinxparam{\DUrole{n,n}{skip}\DUrole{o,o}{=}\DUrole{default_value}{False}}, \sphinxparam{\DUrole{n,n}{collection}\DUrole{o,o}{=}\DUrole{default_value}{None}}}{}
\pysigstopsignatures
\sphinxAtStartPar
Insert an object into the specified collection in the MongoDB database.

\sphinxAtStartPar
This method inserts an object into the specified collection of the MongoDB database.
If the object already exists in the database, the behavior depends on the \sphinxtitleref{overwrite}
and \sphinxtitleref{skip} parameters. If \sphinxtitleref{overwrite} is True, the existing object is deleted and the
new object is inserted. If \sphinxtitleref{skip} is True, the insertion is skipped and the ObjectId
of the existing object is returned. If neither \sphinxtitleref{overwrite} nor \sphinxtitleref{skip} is True and the
object already exists, an exception is raised.
\begin{quote}\begin{description}
\sphinxlineitem{Return type}
\sphinxAtStartPar
\sphinxcode{\sphinxupquote{ObjectId}}

\sphinxlineitem{Parameters}\begin{itemize}
\item {} 
\sphinxAtStartPar
\sphinxstyleliteralstrong{\sphinxupquote{obj}} (\sphinxstyleliteralemphasis{\sphinxupquote{Union}}\sphinxstyleliteralemphasis{\sphinxupquote{{[}}}\sphinxstyleliteralemphasis{\sphinxupquote{Image}}\sphinxstyleliteralemphasis{\sphinxupquote{, }}{\hyperref[\detokenize{forensicfit.core.tape:forensicfit.core.tape.Tape}]{\sphinxcrossref{\sphinxstyleliteralemphasis{\sphinxupquote{Tape}}}}}\sphinxstyleliteralemphasis{\sphinxupquote{, }}{\hyperref[\detokenize{forensicfit.core.tape:forensicfit.core.tape.TapeAnalyzer}]{\sphinxcrossref{\sphinxstyleliteralemphasis{\sphinxupquote{TapeAnalyzer}}}}}\sphinxstyleliteralemphasis{\sphinxupquote{{]}}}) \textendash{} The object to insert into the database.

\item {} 
\sphinxAtStartPar
\sphinxstyleliteralstrong{\sphinxupquote{ext}} (\sphinxstyleliteralemphasis{\sphinxupquote{str}}\sphinxstyleliteralemphasis{\sphinxupquote{, }}\sphinxstyleliteralemphasis{\sphinxupquote{optional}}) \textendash{} The file extension to use when converting the object to a buffer, defaults to ‘.png’.

\item {} 
\sphinxAtStartPar
\sphinxstyleliteralstrong{\sphinxupquote{overwrite}} (\sphinxstyleliteralemphasis{\sphinxupquote{bool}}\sphinxstyleliteralemphasis{\sphinxupquote{, }}\sphinxstyleliteralemphasis{\sphinxupquote{optional}}) \textendash{} Whether to overwrite the existing object if it already exists, defaults to False.

\item {} 
\sphinxAtStartPar
\sphinxstyleliteralstrong{\sphinxupquote{skip}} (\sphinxstyleliteralemphasis{\sphinxupquote{bool}}\sphinxstyleliteralemphasis{\sphinxupquote{, }}\sphinxstyleliteralemphasis{\sphinxupquote{optional}}) \textendash{} Whether to skip the insertion if the object already exists, defaults to False.

\item {} 
\sphinxAtStartPar
\sphinxstyleliteralstrong{\sphinxupquote{collection}} (\sphinxstyleliteralemphasis{\sphinxupquote{str}}\sphinxstyleliteralemphasis{\sphinxupquote{, }}\sphinxstyleliteralemphasis{\sphinxupquote{optional}}) \textendash{} The name of the collection to insert the object into. If not provided, the collection
name is determined from the ‘mode’ field of the object’s metadata.

\end{itemize}

\sphinxlineitem{Returns}
\sphinxAtStartPar
The ObjectId of the inserted object.

\sphinxlineitem{Return type}
\sphinxAtStartPar
ObjectId

\sphinxlineitem{Raises}
\sphinxAtStartPar
\sphinxstyleliteralstrong{\sphinxupquote{Exception}} \textendash{} If the object already exists in the database and neither \sphinxtitleref{overwrite} nor \sphinxtitleref{skip} is True.

\end{description}\end{quote}

\end{fulllineitems}

\index{find() (forensicfit.database.database.Database method)@\spxentry{find()}\spxextra{forensicfit.database.database.Database method}}

\begin{fulllineitems}
\phantomsection\label{\detokenize{forensicfit.database.database:forensicfit.database.database.Database.find}}
\pysigstartsignatures
\pysiglinewithargsret{\sphinxbfcode{\sphinxupquote{find}}}{\sphinxparam{\DUrole{n,n}{filter}}, \sphinxparam{\DUrole{n,n}{collection}\DUrole{o,o}{=}\DUrole{default_value}{\textquotesingle{}analysis\textquotesingle{}}}, \sphinxparam{\DUrole{n,n}{ext}\DUrole{o,o}{=}\DUrole{default_value}{\textquotesingle{}.png\textquotesingle{}}}, \sphinxparam{\DUrole{n,n}{version}\DUrole{o,o}{=}\DUrole{default_value}{\sphinxhyphen{}1}}, \sphinxparam{\DUrole{n,n}{no\_cursor\_timeout}\DUrole{o,o}{=}\DUrole{default_value}{False}}}{}
\pysigstopsignatures
\sphinxAtStartPar
Find and return objects from the specified collection that match the provided filter.

\sphinxAtStartPar
This method finds and returns objects from the specified collection of the MongoDB database
that match the provided filter. The objects are returned as instances of the class associated
with the collection in the class mapping. The objects are sorted by their upload date in the
order specified by the \sphinxtitleref{version} parameter.
\begin{quote}\begin{description}
\sphinxlineitem{Return type}
\sphinxAtStartPar
\sphinxcode{\sphinxupquote{list}}

\sphinxlineitem{Parameters}\begin{itemize}
\item {} 
\sphinxAtStartPar
\sphinxstyleliteralstrong{\sphinxupquote{filter}} (\sphinxstyleliteralemphasis{\sphinxupquote{dict}}) \textendash{} A dictionary specifying the filter criteria to use when searching for the objects.

\item {} 
\sphinxAtStartPar
\sphinxstyleliteralstrong{\sphinxupquote{collection}} (\sphinxstyleliteralemphasis{\sphinxupquote{str}}\sphinxstyleliteralemphasis{\sphinxupquote{, }}\sphinxstyleliteralemphasis{\sphinxupquote{optional}}) \textendash{} The name of the collection to search in, defaults to ‘analysis’.

\item {} 
\sphinxAtStartPar
\sphinxstyleliteralstrong{\sphinxupquote{ext}} (\sphinxstyleliteralemphasis{\sphinxupquote{str}}\sphinxstyleliteralemphasis{\sphinxupquote{, }}\sphinxstyleliteralemphasis{\sphinxupquote{optional}}) \textendash{} The file extension to use when converting the objects to buffers, defaults to ‘.png’.

\item {} 
\sphinxAtStartPar
\sphinxstyleliteralstrong{\sphinxupquote{version}} (\sphinxstyleliteralemphasis{\sphinxupquote{int}}\sphinxstyleliteralemphasis{\sphinxupquote{, }}\sphinxstyleliteralemphasis{\sphinxupquote{optional}}) \textendash{} The sort order for the objects based on their upload date. If \sphinxtitleref{version} is \sphinxhyphen{}1, the objects
are sorted in descending order. If \sphinxtitleref{version} is 1, the objects are sorted in ascending order,
defaults to \sphinxhyphen{}1.

\item {} 
\sphinxAtStartPar
\sphinxstyleliteralstrong{\sphinxupquote{no\_cursor\_timeout}} (\sphinxstyleliteralemphasis{\sphinxupquote{bool}}\sphinxstyleliteralemphasis{\sphinxupquote{, }}\sphinxstyleliteralemphasis{\sphinxupquote{optional}}) \textendash{} Whether to prevent the server\sphinxhyphen{}side cursor from timing out after an inactivity period,
defaults to False.

\end{itemize}

\sphinxlineitem{Returns}
\sphinxAtStartPar
A list of objects from the specified collection that match the provided filter. The objects
are returned as instances of the class associated with the collection in the class mapping.

\sphinxlineitem{Return type}
\sphinxAtStartPar
list

\end{description}\end{quote}

\end{fulllineitems}

\index{map\_to() (forensicfit.database.database.Database method)@\spxentry{map\_to()}\spxextra{forensicfit.database.database.Database method}}

\begin{fulllineitems}
\phantomsection\label{\detokenize{forensicfit.database.database:forensicfit.database.database.Database.map_to}}
\pysigstartsignatures
\pysiglinewithargsret{\sphinxbfcode{\sphinxupquote{map\_to}}}{\sphinxparam{\DUrole{n,n}{func}}, \sphinxparam{\DUrole{n,n}{filter}}, \sphinxparam{\DUrole{n,n}{collection\_source}}, \sphinxparam{\DUrole{n,n}{collection\_target}}, \sphinxparam{\DUrole{n,n}{verbose}\DUrole{o,o}{=}\DUrole{default_value}{True}}, \sphinxparam{\DUrole{n,n}{no\_cursor\_timeout}\DUrole{o,o}{=}\DUrole{default_value}{False}}}{}
\pysigstopsignatures
\sphinxAtStartPar
Apply a function to each object in the source collection that matches the provided filter
and insert the results into the target collection.

\sphinxAtStartPar
This method applies a function to each object in the source collection of the MongoDB database
that matches the provided filter. The results are inserted into the target collection. The
objects are retrieved as instances of the class associated with the source collection in the
class mapping.
\begin{quote}\begin{description}
\sphinxlineitem{Parameters}\begin{itemize}
\item {} 
\sphinxAtStartPar
\sphinxstyleliteralstrong{\sphinxupquote{func}} (\sphinxstyleliteralemphasis{\sphinxupquote{Callable}}) \textendash{} The function to apply to each object. The function should take an object as input and
return an object.

\item {} 
\sphinxAtStartPar
\sphinxstyleliteralstrong{\sphinxupquote{filter}} (\sphinxstyleliteralemphasis{\sphinxupquote{dict}}) \textendash{} A dictionary specifying the filter criteria to use when searching for the objects in
the source collection.

\item {} 
\sphinxAtStartPar
\sphinxstyleliteralstrong{\sphinxupquote{collection\_source}} (\sphinxstyleliteralemphasis{\sphinxupquote{str}}) \textendash{} The name of the source collection to search in.

\item {} 
\sphinxAtStartPar
\sphinxstyleliteralstrong{\sphinxupquote{collection\_target}} (\sphinxstyleliteralemphasis{\sphinxupquote{str}}) \textendash{} The name of the target collection to insert the results into.

\item {} 
\sphinxAtStartPar
\sphinxstyleliteralstrong{\sphinxupquote{verbose}} (\sphinxstyleliteralemphasis{\sphinxupquote{bool}}\sphinxstyleliteralemphasis{\sphinxupquote{, }}\sphinxstyleliteralemphasis{\sphinxupquote{optional}}) \textendash{} Whether to print the filename of each object being processed, defaults to True.

\item {} 
\sphinxAtStartPar
\sphinxstyleliteralstrong{\sphinxupquote{no\_cursor\_timeout}} (\sphinxstyleliteralemphasis{\sphinxupquote{bool}}\sphinxstyleliteralemphasis{\sphinxupquote{, }}\sphinxstyleliteralemphasis{\sphinxupquote{optional}}) \textendash{} Whether to prevent the server\sphinxhyphen{}side cursor from timing out after an inactivity period,
defaults to False.

\end{itemize}

\sphinxlineitem{Return type}
\sphinxAtStartPar
None

\end{description}\end{quote}

\end{fulllineitems}

\index{find\_one() (forensicfit.database.database.Database method)@\spxentry{find\_one()}\spxextra{forensicfit.database.database.Database method}}

\begin{fulllineitems}
\phantomsection\label{\detokenize{forensicfit.database.database:forensicfit.database.database.Database.find_one}}
\pysigstartsignatures
\pysiglinewithargsret{\sphinxbfcode{\sphinxupquote{find\_one}}}{\sphinxparam{\DUrole{n,n}{filter}\DUrole{o,o}{=}\DUrole{default_value}{None}}, \sphinxparam{\DUrole{n,n}{collection}\DUrole{o,o}{=}\DUrole{default_value}{None}}}{}
\pysigstopsignatures
\sphinxAtStartPar
Find and return one object from the specified collection that matches the provided filter.

\sphinxAtStartPar
This method finds and returns one object from the specified collection of the MongoDB database
that matches the provided filter. The object is returned as an instance of the class associated
with the collection in the class mapping. If no collection is specified, one is chosen randomly.
\begin{quote}\begin{description}
\sphinxlineitem{Return type}
\sphinxAtStartPar
\sphinxcode{\sphinxupquote{object}}

\sphinxlineitem{Parameters}\begin{itemize}
\item {} 
\sphinxAtStartPar
\sphinxstyleliteralstrong{\sphinxupquote{filter}} (\sphinxstyleliteralemphasis{\sphinxupquote{dict}}\sphinxstyleliteralemphasis{\sphinxupquote{, }}\sphinxstyleliteralemphasis{\sphinxupquote{optional}}) \textendash{} A dictionary specifying the filter criteria to use when searching for the object. If not
provided, the first object in the collection is returned.

\item {} 
\sphinxAtStartPar
\sphinxstyleliteralstrong{\sphinxupquote{collection}} (\sphinxstyleliteralemphasis{\sphinxupquote{str}}\sphinxstyleliteralemphasis{\sphinxupquote{, }}\sphinxstyleliteralemphasis{\sphinxupquote{optional}}) \textendash{} The name of the collection to search in. If not provided, a collection is chosen randomly.

\end{itemize}

\sphinxlineitem{Returns}
\sphinxAtStartPar
An object from the specified collection that matches the provided filter. The object is
returned as an instance of the class associated with the collection in the class mapping.

\sphinxlineitem{Return type}
\sphinxAtStartPar
object

\sphinxlineitem{Raises}
\sphinxAtStartPar
\sphinxstyleliteralstrong{\sphinxupquote{ValueError}} \textendash{} If no object is found that matches the provided filter.

\end{description}\end{quote}

\end{fulllineitems}

\index{find\_with\_id() (forensicfit.database.database.Database method)@\spxentry{find\_with\_id()}\spxextra{forensicfit.database.database.Database method}}

\begin{fulllineitems}
\phantomsection\label{\detokenize{forensicfit.database.database:forensicfit.database.database.Database.find_with_id}}
\pysigstartsignatures
\pysiglinewithargsret{\sphinxbfcode{\sphinxupquote{find\_with\_id}}}{\sphinxparam{\DUrole{n,n}{\_id}}, \sphinxparam{\DUrole{n,n}{collection}}}{}
\pysigstopsignatures
\sphinxAtStartPar
Find and return an object from the specified collection based on its MongoDB \_id.

\sphinxAtStartPar
This method finds and returns an object from the specified collection of the MongoDB database
based on its MongoDB \_id. The object is returned as an instance of the class associated with
the collection in the class mapping.
\begin{quote}\begin{description}
\sphinxlineitem{Return type}
\sphinxAtStartPar
\sphinxcode{\sphinxupquote{object}}

\sphinxlineitem{Parameters}\begin{itemize}
\item {} 
\sphinxAtStartPar
\sphinxstyleliteralstrong{\sphinxupquote{\_id}} (\sphinxstyleliteralemphasis{\sphinxupquote{str}}) \textendash{} The MongoDB \_id of the object to find.

\item {} 
\sphinxAtStartPar
\sphinxstyleliteralstrong{\sphinxupquote{collection}} (\sphinxstyleliteralemphasis{\sphinxupquote{str}}) \textendash{} The name of the collection to search in.

\end{itemize}

\sphinxlineitem{Returns}
\sphinxAtStartPar
An object from the specified collection with the provided MongoDB \_id. The object is
returned as an instance of the class associated with the collection in the class mapping.

\sphinxlineitem{Return type}
\sphinxAtStartPar
object

\end{description}\end{quote}

\end{fulllineitems}

\index{filter\_with\_metadata() (forensicfit.database.database.Database method)@\spxentry{filter\_with\_metadata()}\spxextra{forensicfit.database.database.Database method}}

\begin{fulllineitems}
\phantomsection\label{\detokenize{forensicfit.database.database:forensicfit.database.database.Database.filter_with_metadata}}
\pysigstartsignatures
\pysiglinewithargsret{\sphinxbfcode{\sphinxupquote{filter\_with\_metadata}}}{\sphinxparam{\DUrole{n,n}{inp}}, \sphinxparam{\DUrole{n,n}{filter}}, \sphinxparam{\DUrole{n,n}{collection}}}{}
\pysigstopsignatures
\sphinxAtStartPar
Filter a list of filenames based on metadata and return the indices of the matching filenames.

\sphinxAtStartPar
This method filters a list of filenames based on the provided filter and the metadata of the
files in the specified collection of the MongoDB database. It returns a list of indices of the
input list that correspond to the filenames that match the filter.
\begin{quote}\begin{description}
\sphinxlineitem{Return type}
\sphinxAtStartPar
\sphinxcode{\sphinxupquote{List}}{[}\sphinxcode{\sphinxupquote{int}}{]}

\sphinxlineitem{Parameters}\begin{itemize}
\item {} 
\sphinxAtStartPar
\sphinxstyleliteralstrong{\sphinxupquote{inp}} (\sphinxstyleliteralemphasis{\sphinxupquote{List}}\sphinxstyleliteralemphasis{\sphinxupquote{{[}}}\sphinxstyleliteralemphasis{\sphinxupquote{str}}\sphinxstyleliteralemphasis{\sphinxupquote{{]}}}) \textendash{} The list of filenames to filter.

\item {} 
\sphinxAtStartPar
\sphinxstyleliteralstrong{\sphinxupquote{filter}} (\sphinxstyleliteralemphasis{\sphinxupquote{dict}}) \textendash{} A dictionary specifying the filter criteria to use when filtering the filenames.

\item {} 
\sphinxAtStartPar
\sphinxstyleliteralstrong{\sphinxupquote{collection}} (\sphinxstyleliteralemphasis{\sphinxupquote{str}}) \textendash{} The name of the collection to search in.

\end{itemize}

\sphinxlineitem{Returns}
\sphinxAtStartPar
A list of indices of the input list that correspond to the filenames that match the filter.

\sphinxlineitem{Return type}
\sphinxAtStartPar
List{[}int{]}

\end{description}\end{quote}

\end{fulllineitems}

\index{count\_documents() (forensicfit.database.database.Database method)@\spxentry{count\_documents()}\spxextra{forensicfit.database.database.Database method}}

\begin{fulllineitems}
\phantomsection\label{\detokenize{forensicfit.database.database:forensicfit.database.database.Database.count_documents}}
\pysigstartsignatures
\pysiglinewithargsret{\sphinxbfcode{\sphinxupquote{count\_documents}}}{\sphinxparam{\DUrole{n,n}{filter}}, \sphinxparam{\DUrole{n,n}{collection}}}{}
\pysigstopsignatures
\sphinxAtStartPar
Count the number of documents in the specified collection that match the provided filter.

\sphinxAtStartPar
This method counts the number of documents in the specified collection of the MongoDB database
that match the provided filter. If the collection does not exist, it returns 0.
\begin{quote}\begin{description}
\sphinxlineitem{Return type}
\sphinxAtStartPar
\sphinxcode{\sphinxupquote{int}}

\sphinxlineitem{Parameters}\begin{itemize}
\item {} 
\sphinxAtStartPar
\sphinxstyleliteralstrong{\sphinxupquote{filter}} (\sphinxstyleliteralemphasis{\sphinxupquote{dict}}) \textendash{} A dictionary specifying the filter criteria to use when counting the documents.

\item {} 
\sphinxAtStartPar
\sphinxstyleliteralstrong{\sphinxupquote{collection}} (\sphinxstyleliteralemphasis{\sphinxupquote{str}}) \textendash{} The name of the collection to count the documents in.

\end{itemize}

\sphinxlineitem{Returns}
\sphinxAtStartPar
The number of documents in the specified collection that match the provided filter.

\sphinxlineitem{Return type}
\sphinxAtStartPar
int

\end{description}\end{quote}

\end{fulllineitems}

\index{export\_to\_files() (forensicfit.database.database.Database method)@\spxentry{export\_to\_files()}\spxextra{forensicfit.database.database.Database method}}

\begin{fulllineitems}
\phantomsection\label{\detokenize{forensicfit.database.database:forensicfit.database.database.Database.export_to_files}}
\pysigstartsignatures
\pysiglinewithargsret{\sphinxbfcode{\sphinxupquote{export\_to\_files}}}{\sphinxparam{\DUrole{n,n}{destination}}, \sphinxparam{\DUrole{n,n}{filter}}, \sphinxparam{\DUrole{n,n}{collection}}, \sphinxparam{\DUrole{n,n}{ext}\DUrole{o,o}{=}\DUrole{default_value}{\textquotesingle{}.png\textquotesingle{}}}, \sphinxparam{\DUrole{n,n}{verbose}\DUrole{o,o}{=}\DUrole{default_value}{True}}, \sphinxparam{\DUrole{n,n}{no\_cursor\_timeout}\DUrole{o,o}{=}\DUrole{default_value}{False}}}{}
\pysigstopsignatures
\sphinxAtStartPar
Export objects from the specified collection that match the provided filter to files.

\sphinxAtStartPar
This method exports objects from the specified collection of the MongoDB database that match
the provided filter to files. The objects are saved as files in the specified destination
directory. The objects are retrieved as instances of the class associated with the collection
in the class mapping.
\begin{quote}\begin{description}
\sphinxlineitem{Parameters}\begin{itemize}
\item {} 
\sphinxAtStartPar
\sphinxstyleliteralstrong{\sphinxupquote{destination}} (\sphinxstyleliteralemphasis{\sphinxupquote{str}}) \textendash{} The path to the directory where the files should be saved.

\item {} 
\sphinxAtStartPar
\sphinxstyleliteralstrong{\sphinxupquote{filter}} (\sphinxstyleliteralemphasis{\sphinxupquote{dict}}) \textendash{} A dictionary specifying the filter criteria to use when searching for the objects.

\item {} 
\sphinxAtStartPar
\sphinxstyleliteralstrong{\sphinxupquote{collection}} (\sphinxstyleliteralemphasis{\sphinxupquote{str}}) \textendash{} The name of the collection to search in.

\item {} 
\sphinxAtStartPar
\sphinxstyleliteralstrong{\sphinxupquote{ext}} (\sphinxstyleliteralemphasis{\sphinxupquote{str}}\sphinxstyleliteralemphasis{\sphinxupquote{, }}\sphinxstyleliteralemphasis{\sphinxupquote{optional}}) \textendash{} The file extension to use when saving the objects, defaults to ‘.png’.

\item {} 
\sphinxAtStartPar
\sphinxstyleliteralstrong{\sphinxupquote{verbose}} (\sphinxstyleliteralemphasis{\sphinxupquote{bool}}\sphinxstyleliteralemphasis{\sphinxupquote{, }}\sphinxstyleliteralemphasis{\sphinxupquote{optional}}) \textendash{} Whether to print the path of each file being saved, defaults to True.

\item {} 
\sphinxAtStartPar
\sphinxstyleliteralstrong{\sphinxupquote{no\_cursor\_timeout}} (\sphinxstyleliteralemphasis{\sphinxupquote{bool}}\sphinxstyleliteralemphasis{\sphinxupquote{, }}\sphinxstyleliteralemphasis{\sphinxupquote{optional}}) \textendash{} Whether to prevent the server\sphinxhyphen{}side cursor from timing out after an inactivity period,
defaults to False.

\end{itemize}

\sphinxlineitem{Return type}
\sphinxAtStartPar
None

\end{description}\end{quote}

\end{fulllineitems}

\index{drop\_collection() (forensicfit.database.database.Database method)@\spxentry{drop\_collection()}\spxextra{forensicfit.database.database.Database method}}

\begin{fulllineitems}
\phantomsection\label{\detokenize{forensicfit.database.database:forensicfit.database.database.Database.drop_collection}}
\pysigstartsignatures
\pysiglinewithargsret{\sphinxbfcode{\sphinxupquote{drop\_collection}}}{\sphinxparam{\DUrole{n,n}{collection}}}{}
\pysigstopsignatures
\sphinxAtStartPar
Drop the specified collection from the MongoDB database.

\sphinxAtStartPar
This method drops the specified collection and its associated ‘files’ and ‘chunks’ collections
from the MongoDB database.
\begin{quote}\begin{description}
\sphinxlineitem{Parameters}
\sphinxAtStartPar
\sphinxstyleliteralstrong{\sphinxupquote{collection}} (\sphinxstyleliteralemphasis{\sphinxupquote{str}}) \textendash{} The name of the collection to drop.

\sphinxlineitem{Return type}
\sphinxAtStartPar
None

\end{description}\end{quote}

\end{fulllineitems}

\index{delete() (forensicfit.database.database.Database method)@\spxentry{delete()}\spxextra{forensicfit.database.database.Database method}}

\begin{fulllineitems}
\phantomsection\label{\detokenize{forensicfit.database.database:forensicfit.database.database.Database.delete}}
\pysigstartsignatures
\pysiglinewithargsret{\sphinxbfcode{\sphinxupquote{delete}}}{\sphinxparam{\DUrole{n,n}{filter}}, \sphinxparam{\DUrole{n,n}{collection}}}{}
\pysigstopsignatures
\sphinxAtStartPar
Delete documents from the specified collection that match the provided filter.

\sphinxAtStartPar
This method deletes documents from the specified collection of the MongoDB database
that match the provided filter.
\begin{quote}\begin{description}
\sphinxlineitem{Parameters}\begin{itemize}
\item {} 
\sphinxAtStartPar
\sphinxstyleliteralstrong{\sphinxupquote{filter}} (\sphinxstyleliteralemphasis{\sphinxupquote{dict}}) \textendash{} A dictionary specifying the filter criteria to use when deleting the documents.

\item {} 
\sphinxAtStartPar
\sphinxstyleliteralstrong{\sphinxupquote{collection}} (\sphinxstyleliteralemphasis{\sphinxupquote{str}}) \textendash{} The name of the collection to delete the documents from.

\end{itemize}

\sphinxlineitem{Return type}
\sphinxAtStartPar
None

\end{description}\end{quote}

\end{fulllineitems}

\index{delete\_database() (forensicfit.database.database.Database method)@\spxentry{delete\_database()}\spxextra{forensicfit.database.database.Database method}}

\begin{fulllineitems}
\phantomsection\label{\detokenize{forensicfit.database.database:forensicfit.database.database.Database.delete_database}}
\pysigstartsignatures
\pysiglinewithargsret{\sphinxbfcode{\sphinxupquote{delete\_database}}}{}{}
\pysigstopsignatures
\sphinxAtStartPar
Delete the MongoDB database associated with this Database instance.

\sphinxAtStartPar
This method deletes the MongoDB database that this Database instance is connected to.
\begin{quote}\begin{description}
\sphinxlineitem{Return type}
\sphinxAtStartPar
None

\end{description}\end{quote}

\end{fulllineitems}

\index{collection\_names (forensicfit.database.database.Database property)@\spxentry{collection\_names}\spxextra{forensicfit.database.database.Database property}}

\begin{fulllineitems}
\phantomsection\label{\detokenize{forensicfit.database.database:forensicfit.database.database.Database.collection_names}}
\pysigstartsignatures
\pysigline{\sphinxbfcode{\sphinxupquote{property\DUrole{w,w}{  }}}\sphinxbfcode{\sphinxupquote{collection\_names}}\sphinxbfcode{\sphinxupquote{\DUrole{p,p}{:}\DUrole{w,w}{  }List\DUrole{p,p}{{[}}str\DUrole{p,p}{{]}}}}}
\pysigstopsignatures
\sphinxAtStartPar
Get the names of the collections in the MongoDB database.

\sphinxAtStartPar
This method returns the names of the collections in the MongoDB database that this Database
instance is connected to. The ‘.files’ suffix is removed from the collection names.
\begin{quote}\begin{description}
\sphinxlineitem{Returns}
\sphinxAtStartPar
A list of the names of the collections in the MongoDB database.

\sphinxlineitem{Return type}
\sphinxAtStartPar
List{[}str{]}

\end{description}\end{quote}

\end{fulllineitems}

\index{connected (forensicfit.database.database.Database property)@\spxentry{connected}\spxextra{forensicfit.database.database.Database property}}

\begin{fulllineitems}
\phantomsection\label{\detokenize{forensicfit.database.database:forensicfit.database.database.Database.connected}}
\pysigstartsignatures
\pysigline{\sphinxbfcode{\sphinxupquote{property\DUrole{w,w}{  }}}\sphinxbfcode{\sphinxupquote{connected}}\sphinxbfcode{\sphinxupquote{\DUrole{p,p}{:}\DUrole{w,w}{  }bool}}}
\pysigstopsignatures
\sphinxAtStartPar
Check if the Database instance is currently connected to the MongoDB database.

\sphinxAtStartPar
This method checks if the Database instance is currently connected to the MongoDB database
by attempting to retrieve the server information. If the server information is successfully
retrieved, the method returns True. If a ServerSelectionTimeoutError occurs, the method
prints the error and returns False.
\begin{quote}\begin{description}
\sphinxlineitem{Returns}
\sphinxAtStartPar
True if the Database instance is currently connected to the MongoDB database, False otherwise.

\sphinxlineitem{Return type}
\sphinxAtStartPar
bool

\end{description}\end{quote}

\end{fulllineitems}

\index{server\_info (forensicfit.database.database.Database property)@\spxentry{server\_info}\spxextra{forensicfit.database.database.Database property}}

\begin{fulllineitems}
\phantomsection\label{\detokenize{forensicfit.database.database:forensicfit.database.database.Database.server_info}}
\pysigstartsignatures
\pysigline{\sphinxbfcode{\sphinxupquote{property\DUrole{w,w}{  }}}\sphinxbfcode{\sphinxupquote{server\_info}}\sphinxbfcode{\sphinxupquote{\DUrole{p,p}{:}\DUrole{w,w}{  }dict}}}
\pysigstopsignatures
\sphinxAtStartPar
Get the server information for the MongoDB database.

\sphinxAtStartPar
This method retrieves and returns the server information for the MongoDB database that this
Database instance is connected to.
\begin{quote}\begin{description}
\sphinxlineitem{Returns}
\sphinxAtStartPar
A dictionary containing the server information for the MongoDB database.

\sphinxlineitem{Return type}
\sphinxAtStartPar
dict

\end{description}\end{quote}

\end{fulllineitems}


\end{fulllineitems}

\index{dict2mongo\_query() (in module forensicfit.database.database)@\spxentry{dict2mongo\_query()}\spxextra{in module forensicfit.database.database}}

\begin{fulllineitems}
\phantomsection\label{\detokenize{forensicfit.database.database:forensicfit.database.database.dict2mongo_query}}
\pysigstartsignatures
\pysiglinewithargsret{\sphinxcode{\sphinxupquote{forensicfit.database.database.}}\sphinxbfcode{\sphinxupquote{dict2mongo\_query}}}{\sphinxparam{\DUrole{n,n}{inp}}, \sphinxparam{\DUrole{n,n}{previous\_key}\DUrole{o,o}{=}\DUrole{default_value}{\textquotesingle{}\textquotesingle{}}}}{}
\pysigstopsignatures
\sphinxAtStartPar
Convert a dictionary into a MongoDB query.

\sphinxAtStartPar
This function takes a dictionary and converts it into a MongoDB query. The keys of the
dictionary are concatenated with the \sphinxtitleref{previous\_key} parameter to form the keys of the query.
The values of the dictionary are used as the values of the query.
\begin{quote}\begin{description}
\sphinxlineitem{Return type}
\sphinxAtStartPar
\sphinxcode{\sphinxupquote{dict}}

\sphinxlineitem{Parameters}\begin{itemize}
\item {} 
\sphinxAtStartPar
\sphinxstyleliteralstrong{\sphinxupquote{inp}} (\sphinxstyleliteralemphasis{\sphinxupquote{dict}}) \textendash{} The dictionary to convert into a MongoDB query.

\item {} 
\sphinxAtStartPar
\sphinxstyleliteralstrong{\sphinxupquote{previous\_key}} (\sphinxstyleliteralemphasis{\sphinxupquote{str}}\sphinxstyleliteralemphasis{\sphinxupquote{, }}\sphinxstyleliteralemphasis{\sphinxupquote{optional}}) \textendash{} The key to prepend to the keys of the dictionary when forming the keys of the query,
defaults to an empty string.

\end{itemize}

\sphinxlineitem{Returns}
\sphinxAtStartPar
The MongoDB query formed from the input dictionary.

\sphinxlineitem{Return type}
\sphinxAtStartPar
dict

\end{description}\end{quote}

\end{fulllineitems}

\index{list\_databases() (in module forensicfit.database.database)@\spxentry{list\_databases()}\spxextra{in module forensicfit.database.database}}

\begin{fulllineitems}
\phantomsection\label{\detokenize{forensicfit.database.database:forensicfit.database.database.list_databases}}
\pysigstartsignatures
\pysiglinewithargsret{\sphinxcode{\sphinxupquote{forensicfit.database.database.}}\sphinxbfcode{\sphinxupquote{list\_databases}}}{\sphinxparam{\DUrole{n,n}{host}\DUrole{o,o}{=}\DUrole{default_value}{\textquotesingle{}localhost\textquotesingle{}}}, \sphinxparam{\DUrole{n,n}{port}\DUrole{o,o}{=}\DUrole{default_value}{27017}}, \sphinxparam{\DUrole{n,n}{username}\DUrole{o,o}{=}\DUrole{default_value}{\textquotesingle{}\textquotesingle{}}}, \sphinxparam{\DUrole{n,n}{password}\DUrole{o,o}{=}\DUrole{default_value}{\textquotesingle{}\textquotesingle{}}}}{}
\pysigstopsignatures
\sphinxAtStartPar
List the names of all databases on a MongoDB server.

\sphinxAtStartPar
This function connects to a MongoDB server using the provided host, port, username, and password,
and returns a list of the names of all databases on the server.
\begin{quote}\begin{description}
\sphinxlineitem{Return type}
\sphinxAtStartPar
\sphinxcode{\sphinxupquote{List}}{[}\sphinxcode{\sphinxupquote{str}}{]}

\sphinxlineitem{Parameters}\begin{itemize}
\item {} 
\sphinxAtStartPar
\sphinxstyleliteralstrong{\sphinxupquote{host}} (\sphinxstyleliteralemphasis{\sphinxupquote{str}}\sphinxstyleliteralemphasis{\sphinxupquote{, }}\sphinxstyleliteralemphasis{\sphinxupquote{optional}}) \textendash{} The host IP address or hostname where the MongoDB server is running, defaults to ‘localhost’.

\item {} 
\sphinxAtStartPar
\sphinxstyleliteralstrong{\sphinxupquote{port}} (\sphinxstyleliteralemphasis{\sphinxupquote{int}}\sphinxstyleliteralemphasis{\sphinxupquote{, }}\sphinxstyleliteralemphasis{\sphinxupquote{optional}}) \textendash{} The port number to connect to the MongoDB server, defaults to 27017.

\item {} 
\sphinxAtStartPar
\sphinxstyleliteralstrong{\sphinxupquote{username}} (\sphinxstyleliteralemphasis{\sphinxupquote{str}}\sphinxstyleliteralemphasis{\sphinxupquote{, }}\sphinxstyleliteralemphasis{\sphinxupquote{optional}}) \textendash{} The username for authenticating with the MongoDB server, defaults to an empty string.

\item {} 
\sphinxAtStartPar
\sphinxstyleliteralstrong{\sphinxupquote{password}} (\sphinxstyleliteralemphasis{\sphinxupquote{str}}\sphinxstyleliteralemphasis{\sphinxupquote{, }}\sphinxstyleliteralemphasis{\sphinxupquote{optional}}) \textendash{} The password for authenticating with the MongoDB server, defaults to an empty string.

\end{itemize}

\sphinxlineitem{Returns}
\sphinxAtStartPar
A list of the names of all databases on the MongoDB server.

\sphinxlineitem{Return type}
\sphinxAtStartPar
List{[}str{]}

\end{description}\end{quote}

\end{fulllineitems}

\index{dump() (in module forensicfit.database.database)@\spxentry{dump()}\spxextra{in module forensicfit.database.database}}

\begin{fulllineitems}
\phantomsection\label{\detokenize{forensicfit.database.database:forensicfit.database.database.dump}}
\pysigstartsignatures
\pysiglinewithargsret{\sphinxcode{\sphinxupquote{forensicfit.database.database.}}\sphinxbfcode{\sphinxupquote{dump}}}{\sphinxparam{\DUrole{n,n}{db}\DUrole{o,o}{=}\DUrole{default_value}{None}}, \sphinxparam{\DUrole{n,n}{host}\DUrole{o,o}{=}\DUrole{default_value}{None}}, \sphinxparam{\DUrole{n,n}{port}\DUrole{o,o}{=}\DUrole{default_value}{None}}, \sphinxparam{\DUrole{n,n}{username}\DUrole{o,o}{=}\DUrole{default_value}{None}}, \sphinxparam{\DUrole{n,n}{password}\DUrole{o,o}{=}\DUrole{default_value}{None}}, \sphinxparam{\DUrole{n,n}{out}\DUrole{o,o}{=}\DUrole{default_value}{None}}, \sphinxparam{\DUrole{n,n}{collection}\DUrole{o,o}{=}\DUrole{default_value}{None}}}{}
\pysigstopsignatures
\sphinxAtStartPar
Dump a MongoDB database or collection to a BSON file.

\sphinxAtStartPar
This function uses the \sphinxtitleref{mongodump} command to dump a MongoDB database or collection to a BSON
file. The \sphinxtitleref{mongodump} command is a utility for creating a binary export of the contents of a
database.
\begin{quote}\begin{description}
\sphinxlineitem{Parameters}\begin{itemize}
\item {} 
\sphinxAtStartPar
\sphinxstyleliteralstrong{\sphinxupquote{db}} (\sphinxstyleliteralemphasis{\sphinxupquote{str}}\sphinxstyleliteralemphasis{\sphinxupquote{, }}\sphinxstyleliteralemphasis{\sphinxupquote{optional}}) \textendash{} The name of the database to dump. If not provided, all databases are dumped.

\item {} 
\sphinxAtStartPar
\sphinxstyleliteralstrong{\sphinxupquote{host}} (\sphinxstyleliteralemphasis{\sphinxupquote{str}}\sphinxstyleliteralemphasis{\sphinxupquote{, }}\sphinxstyleliteralemphasis{\sphinxupquote{optional}}) \textendash{} The host IP address or hostname where the MongoDB server is running. If not provided,
‘localhost’ is used.

\item {} 
\sphinxAtStartPar
\sphinxstyleliteralstrong{\sphinxupquote{port}} (\sphinxstyleliteralemphasis{\sphinxupquote{int}}\sphinxstyleliteralemphasis{\sphinxupquote{, }}\sphinxstyleliteralemphasis{\sphinxupquote{optional}}) \textendash{} The port number to connect to the MongoDB server. If not provided, 27017 is used.

\item {} 
\sphinxAtStartPar
\sphinxstyleliteralstrong{\sphinxupquote{username}} (\sphinxstyleliteralemphasis{\sphinxupquote{str}}\sphinxstyleliteralemphasis{\sphinxupquote{, }}\sphinxstyleliteralemphasis{\sphinxupquote{optional}}) \textendash{} The username for authenticating with the MongoDB server. If not provided, no authentication
is used.

\item {} 
\sphinxAtStartPar
\sphinxstyleliteralstrong{\sphinxupquote{password}} (\sphinxstyleliteralemphasis{\sphinxupquote{str}}\sphinxstyleliteralemphasis{\sphinxupquote{, }}\sphinxstyleliteralemphasis{\sphinxupquote{optional}}) \textendash{} The password for authenticating with the MongoDB server. If not provided, no authentication
is used.

\item {} 
\sphinxAtStartPar
\sphinxstyleliteralstrong{\sphinxupquote{out}} (\sphinxstyleliteralemphasis{\sphinxupquote{str}}\sphinxstyleliteralemphasis{\sphinxupquote{, }}\sphinxstyleliteralemphasis{\sphinxupquote{optional}}) \textendash{} The directory where the dump should be output. If not provided, the dump is output to the
‘dump’ directory in the current working directory.

\item {} 
\sphinxAtStartPar
\sphinxstyleliteralstrong{\sphinxupquote{collection}} (\sphinxstyleliteralemphasis{\sphinxupquote{str}}\sphinxstyleliteralemphasis{\sphinxupquote{, }}\sphinxstyleliteralemphasis{\sphinxupquote{optional}}) \textendash{} The name of the collection to dump. If not provided, all collections in the specified
database are dumped.

\end{itemize}

\sphinxlineitem{Return type}
\sphinxAtStartPar
None

\end{description}\end{quote}

\end{fulllineitems}

\index{restore() (in module forensicfit.database.database)@\spxentry{restore()}\spxextra{in module forensicfit.database.database}}

\begin{fulllineitems}
\phantomsection\label{\detokenize{forensicfit.database.database:forensicfit.database.database.restore}}
\pysigstartsignatures
\pysiglinewithargsret{\sphinxcode{\sphinxupquote{forensicfit.database.database.}}\sphinxbfcode{\sphinxupquote{restore}}}{\sphinxparam{\DUrole{n,n}{path}\DUrole{o,o}{=}\DUrole{default_value}{None}}, \sphinxparam{\DUrole{n,n}{db}\DUrole{o,o}{=}\DUrole{default_value}{None}}, \sphinxparam{\DUrole{n,n}{host}\DUrole{o,o}{=}\DUrole{default_value}{None}}, \sphinxparam{\DUrole{n,n}{port}\DUrole{o,o}{=}\DUrole{default_value}{None}}, \sphinxparam{\DUrole{n,n}{username}\DUrole{o,o}{=}\DUrole{default_value}{None}}, \sphinxparam{\DUrole{n,n}{password}\DUrole{o,o}{=}\DUrole{default_value}{None}}, \sphinxparam{\DUrole{n,n}{collection}\DUrole{o,o}{=}\DUrole{default_value}{None}}}{}
\pysigstopsignatures
\end{fulllineitems}



\subparagraph{Module contents}
\label{\detokenize{forensicfit.database:module-forensicfit.database}}\label{\detokenize{forensicfit.database:module-contents}}\index{module@\spxentry{module}!forensicfit.database@\spxentry{forensicfit.database}}\index{forensicfit.database@\spxentry{forensicfit.database}!module@\spxentry{module}}
\sphinxstepscope


\paragraph{forensicfit.utils package}
\label{\detokenize{forensicfit.utils:forensicfit-utils-package}}\label{\detokenize{forensicfit.utils::doc}}

\subparagraph{Submodules}
\label{\detokenize{forensicfit.utils:submodules}}
\sphinxstepscope


\subparagraph{forensicfit.utils.array\_tools module}
\label{\detokenize{forensicfit.utils.array_tools:module-forensicfit.utils.array_tools}}\label{\detokenize{forensicfit.utils.array_tools:forensicfit-utils-array-tools-module}}\label{\detokenize{forensicfit.utils.array_tools::doc}}\index{module@\spxentry{module}!forensicfit.utils.array\_tools@\spxentry{forensicfit.utils.array\_tools}}\index{forensicfit.utils.array\_tools@\spxentry{forensicfit.utils.array\_tools}!module@\spxentry{module}}\index{serializer() (in module forensicfit.utils.array\_tools)@\spxentry{serializer()}\spxextra{in module forensicfit.utils.array\_tools}}

\begin{fulllineitems}
\phantomsection\label{\detokenize{forensicfit.utils.array_tools:forensicfit.utils.array_tools.serializer}}
\pysigstartsignatures
\pysiglinewithargsret{\sphinxcode{\sphinxupquote{forensicfit.utils.array\_tools.}}\sphinxbfcode{\sphinxupquote{serializer}}}{\sphinxparam{\DUrole{n,n}{indict}}}{}
\pysigstopsignatures
\sphinxAtStartPar
Serilizes any given dictionary for mongodb.
\begin{quote}\begin{description}
\sphinxlineitem{Return type}
\sphinxAtStartPar
\sphinxcode{\sphinxupquote{dict}}

\sphinxlineitem{Parameters}
\sphinxAtStartPar
\sphinxstyleliteralstrong{\sphinxupquote{indict}} (\sphinxstyleliteralemphasis{\sphinxupquote{dict}}) \textendash{} input dictionary

\end{description}\end{quote}

\end{fulllineitems}

\index{vote\_calculator() (in module forensicfit.utils.array\_tools)@\spxentry{vote\_calculator()}\spxextra{in module forensicfit.utils.array\_tools}}

\begin{fulllineitems}
\phantomsection\label{\detokenize{forensicfit.utils.array_tools:forensicfit.utils.array_tools.vote_calculator}}
\pysigstartsignatures
\pysiglinewithargsret{\sphinxcode{\sphinxupquote{forensicfit.utils.array\_tools.}}\sphinxbfcode{\sphinxupquote{vote\_calculator}}}{\sphinxparam{\DUrole{n,n}{prediction}}}{}
\pysigstopsignatures\begin{quote}\begin{description}
\sphinxlineitem{Return type}
\sphinxAtStartPar
\sphinxcode{\sphinxupquote{ndarray}}

\end{description}\end{quote}

\end{fulllineitems}

\index{read\_bytes\_io() (in module forensicfit.utils.array\_tools)@\spxentry{read\_bytes\_io()}\spxextra{in module forensicfit.utils.array\_tools}}

\begin{fulllineitems}
\phantomsection\label{\detokenize{forensicfit.utils.array_tools:forensicfit.utils.array_tools.read_bytes_io}}
\pysigstartsignatures
\pysiglinewithargsret{\sphinxcode{\sphinxupquote{forensicfit.utils.array\_tools.}}\sphinxbfcode{\sphinxupquote{read\_bytes\_io}}}{\sphinxparam{\DUrole{n,n}{obj}}, \sphinxparam{\DUrole{n,n}{method}\DUrole{o,o}{=}\DUrole{default_value}{\textquotesingle{}numpy\textquotesingle{}}}}{}
\pysigstopsignatures
\sphinxAtStartPar
reads a binary file stored in mongodb and returns a numpy array
\begin{quote}\begin{description}
\sphinxlineitem{Return type}
\sphinxAtStartPar
\sphinxcode{\sphinxupquote{array}}

\sphinxlineitem{Parameters}
\sphinxAtStartPar
\sphinxstyleliteralstrong{\sphinxupquote{obj}} (\sphinxstyleliteralemphasis{\sphinxupquote{GridOut}}) \textendash{} output from a mongodb girdfs file

\sphinxlineitem{Returns}
\sphinxAtStartPar
numpy array containing the information loaded from gridfs file

\sphinxlineitem{Return type}
\sphinxAtStartPar
np.array

\end{description}\end{quote}

\end{fulllineitems}

\index{write\_bytes\_io() (in module forensicfit.utils.array\_tools)@\spxentry{write\_bytes\_io()}\spxextra{in module forensicfit.utils.array\_tools}}

\begin{fulllineitems}
\phantomsection\label{\detokenize{forensicfit.utils.array_tools:forensicfit.utils.array_tools.write_bytes_io}}
\pysigstartsignatures
\pysiglinewithargsret{\sphinxcode{\sphinxupquote{forensicfit.utils.array\_tools.}}\sphinxbfcode{\sphinxupquote{write\_bytes\_io}}}{\sphinxparam{\DUrole{n,n}{obj}}, \sphinxparam{\DUrole{n,n}{method}\DUrole{o,o}{=}\DUrole{default_value}{\textquotesingle{}numpy\textquotesingle{}}}}{}
\pysigstopsignatures\begin{quote}\begin{description}
\sphinxlineitem{Return type}
\sphinxAtStartPar
\sphinxcode{\sphinxupquote{BytesIO}}

\end{description}\end{quote}

\end{fulllineitems}


\sphinxstepscope


\subparagraph{forensicfit.utils.general module}
\label{\detokenize{forensicfit.utils.general:module-forensicfit.utils.general}}\label{\detokenize{forensicfit.utils.general:forensicfit-utils-general-module}}\label{\detokenize{forensicfit.utils.general::doc}}\index{module@\spxentry{module}!forensicfit.utils.general@\spxentry{forensicfit.utils.general}}\index{forensicfit.utils.general@\spxentry{forensicfit.utils.general}!module@\spxentry{module}}\index{copy\_doc() (in module forensicfit.utils.general)@\spxentry{copy\_doc()}\spxextra{in module forensicfit.utils.general}}

\begin{fulllineitems}
\phantomsection\label{\detokenize{forensicfit.utils.general:forensicfit.utils.general.copy_doc}}
\pysigstartsignatures
\pysiglinewithargsret{\sphinxcode{\sphinxupquote{forensicfit.utils.general.}}\sphinxbfcode{\sphinxupquote{copy\_doc}}}{\sphinxparam{\DUrole{n,n}{copy\_func}}}{}
\pysigstopsignatures
\sphinxAtStartPar
Use Example: copy\_doc(self.copy\_func)(self.func) or used as deco
\begin{quote}\begin{description}
\sphinxlineitem{Return type}
\sphinxAtStartPar
\sphinxcode{\sphinxupquote{Callable}}

\end{description}\end{quote}

\end{fulllineitems}


\sphinxstepscope


\subparagraph{forensicfit.utils.image\_tools module}
\label{\detokenize{forensicfit.utils.image_tools:module-forensicfit.utils.image_tools}}\label{\detokenize{forensicfit.utils.image_tools:forensicfit-utils-image-tools-module}}\label{\detokenize{forensicfit.utils.image_tools::doc}}\index{module@\spxentry{module}!forensicfit.utils.image\_tools@\spxentry{forensicfit.utils.image\_tools}}\index{forensicfit.utils.image\_tools@\spxentry{forensicfit.utils.image\_tools}!module@\spxentry{module}}\index{rotate\_image() (in module forensicfit.utils.image\_tools)@\spxentry{rotate\_image()}\spxextra{in module forensicfit.utils.image\_tools}}

\begin{fulllineitems}
\phantomsection\label{\detokenize{forensicfit.utils.image_tools:forensicfit.utils.image_tools.rotate_image}}
\pysigstartsignatures
\pysiglinewithargsret{\sphinxcode{\sphinxupquote{forensicfit.utils.image\_tools.}}\sphinxbfcode{\sphinxupquote{rotate\_image}}}{\sphinxparam{\DUrole{n,n}{image}}, \sphinxparam{\DUrole{n,n}{angle}}}{}
\pysigstopsignatures
\sphinxAtStartPar
Rotates the image by angle degrees
\begin{quote}\begin{description}
\sphinxlineitem{Parameters}
\sphinxAtStartPar
\sphinxstyleliteralstrong{\sphinxupquote{angle}} (\sphinxstyleliteralemphasis{\sphinxupquote{float}}) \textendash{} Angle of rotation.

\sphinxlineitem{Returns}
\sphinxAtStartPar
None.

\end{description}\end{quote}

\end{fulllineitems}

\index{gaussian\_blur() (in module forensicfit.utils.image\_tools)@\spxentry{gaussian\_blur()}\spxextra{in module forensicfit.utils.image\_tools}}

\begin{fulllineitems}
\phantomsection\label{\detokenize{forensicfit.utils.image_tools:forensicfit.utils.image_tools.gaussian_blur}}
\pysigstartsignatures
\pysiglinewithargsret{\sphinxcode{\sphinxupquote{forensicfit.utils.image\_tools.}}\sphinxbfcode{\sphinxupquote{gaussian\_blur}}}{\sphinxparam{\DUrole{n,n}{image}}, \sphinxparam{\DUrole{n,n}{window}\DUrole{o,o}{=}\DUrole{default_value}{(15, 15)}}}{}
\pysigstopsignatures
\sphinxAtStartPar
This method applies Gaussian Blur filter to the image.
\begin{quote}\begin{description}
\sphinxlineitem{Parameters}
\sphinxAtStartPar
\sphinxstyleliteralstrong{\sphinxupquote{window}} (\sphinxstyleliteralemphasis{\sphinxupquote{tuple int}}\sphinxstyleliteralemphasis{\sphinxupquote{, }}\sphinxstyleliteralemphasis{\sphinxupquote{optional}}) \textendash{} The window in which the gaussian blur is going to be applied.
The default is (15,15).

\sphinxlineitem{Return type}
\sphinxAtStartPar
None.

\end{description}\end{quote}

\end{fulllineitems}

\index{split\_v() (in module forensicfit.utils.image\_tools)@\spxentry{split\_v()}\spxextra{in module forensicfit.utils.image\_tools}}

\begin{fulllineitems}
\phantomsection\label{\detokenize{forensicfit.utils.image_tools:forensicfit.utils.image_tools.split_v}}
\pysigstartsignatures
\pysiglinewithargsret{\sphinxcode{\sphinxupquote{forensicfit.utils.image\_tools.}}\sphinxbfcode{\sphinxupquote{split\_v}}}{\sphinxparam{\DUrole{n,n}{image}}, \sphinxparam{\DUrole{n,n}{pixel\_index}\DUrole{o,o}{=}\DUrole{default_value}{None}}, \sphinxparam{\DUrole{n,n}{pick\_side}\DUrole{o,o}{=}\DUrole{default_value}{\textquotesingle{}L\textquotesingle{}}}, \sphinxparam{\DUrole{n,n}{flip}\DUrole{o,o}{=}\DUrole{default_value}{True}}}{}
\pysigstopsignatures
\sphinxAtStartPar
This method splits the image in 2 images based on the fraction that is
given in pixel\_index
\begin{quote}\begin{description}
\sphinxlineitem{Parameters}\begin{itemize}
\item {} 
\sphinxAtStartPar
\sphinxstyleliteralstrong{\sphinxupquote{pixel\_index}} (\sphinxstyleliteralemphasis{\sphinxupquote{float}}\sphinxstyleliteralemphasis{\sphinxupquote{, }}\sphinxstyleliteralemphasis{\sphinxupquote{optional}}) \textendash{} fraction in which the image is going to be split. The value should
be a number between zero and one. The default is 0.5.

\item {} 
\sphinxAtStartPar
\sphinxstyleliteralstrong{\sphinxupquote{pick\_side}} (\sphinxstyleliteralemphasis{\sphinxupquote{str}}\sphinxstyleliteralemphasis{\sphinxupquote{, }}\sphinxstyleliteralemphasis{\sphinxupquote{optional}}) \textendash{} The side in which will over write the image in the class. The
default is ‘L’.

\end{itemize}

\sphinxlineitem{Return type}
\sphinxAtStartPar
None.

\end{description}\end{quote}

\end{fulllineitems}

\index{to\_gray() (in module forensicfit.utils.image\_tools)@\spxentry{to\_gray()}\spxextra{in module forensicfit.utils.image\_tools}}

\begin{fulllineitems}
\phantomsection\label{\detokenize{forensicfit.utils.image_tools:forensicfit.utils.image_tools.to_gray}}
\pysigstartsignatures
\pysiglinewithargsret{\sphinxcode{\sphinxupquote{forensicfit.utils.image\_tools.}}\sphinxbfcode{\sphinxupquote{to\_gray}}}{\sphinxparam{\DUrole{n,n}{image}}, \sphinxparam{\DUrole{n,n}{mode}\DUrole{o,o}{=}\DUrole{default_value}{\textquotesingle{}SD\textquotesingle{}}}}{}
\pysigstopsignatures
\sphinxAtStartPar
Gray Scale image of the input image.

\sphinxAtStartPar
modes: ‘BT.470’ and ‘BT.709’
SD ‘BT.470’ : Y = 0.299 R + 0.587 G + 0.114 B
HD ‘BT.709’ : Y = 0.2125 R + 0.7154 G + 0.0721 B
\begin{quote}\begin{description}
\sphinxlineitem{Return type}
\sphinxAtStartPar
\sphinxcode{\sphinxupquote{ndarray}}

\sphinxlineitem{Returns}
\sphinxAtStartPar
\sphinxstylestrong{gray\_scale} \textendash{} Gray Scale image of the input image.

\sphinxlineitem{Return type}
\sphinxAtStartPar
cv2 object

\end{description}\end{quote}

\end{fulllineitems}

\index{to\_rbg() (in module forensicfit.utils.image\_tools)@\spxentry{to\_rbg()}\spxextra{in module forensicfit.utils.image\_tools}}

\begin{fulllineitems}
\phantomsection\label{\detokenize{forensicfit.utils.image_tools:forensicfit.utils.image_tools.to_rbg}}
\pysigstartsignatures
\pysiglinewithargsret{\sphinxcode{\sphinxupquote{forensicfit.utils.image\_tools.}}\sphinxbfcode{\sphinxupquote{to\_rbg}}}{\sphinxparam{\DUrole{n,n}{image}}}{}
\pysigstopsignatures
\end{fulllineitems}

\index{flip() (in module forensicfit.utils.image\_tools)@\spxentry{flip()}\spxextra{in module forensicfit.utils.image\_tools}}

\begin{fulllineitems}
\phantomsection\label{\detokenize{forensicfit.utils.image_tools:forensicfit.utils.image_tools.flip}}
\pysigstartsignatures
\pysiglinewithargsret{\sphinxcode{\sphinxupquote{forensicfit.utils.image\_tools.}}\sphinxbfcode{\sphinxupquote{flip}}}{\sphinxparam{\DUrole{n,n}{image}}}{}
\pysigstopsignatures
\end{fulllineitems}

\index{contours() (in module forensicfit.utils.image\_tools)@\spxentry{contours()}\spxextra{in module forensicfit.utils.image\_tools}}

\begin{fulllineitems}
\phantomsection\label{\detokenize{forensicfit.utils.image_tools:forensicfit.utils.image_tools.contours}}
\pysigstartsignatures
\pysiglinewithargsret{\sphinxcode{\sphinxupquote{forensicfit.utils.image\_tools.}}\sphinxbfcode{\sphinxupquote{contours}}}{\sphinxparam{\DUrole{n,n}{image}}, \sphinxparam{\DUrole{n,n}{mask\_threshold}\DUrole{o,o}{=}\DUrole{default_value}{60}}}{}
\pysigstopsignatures\begin{quote}

\sphinxAtStartPar
A list of pixels that create the contours in the image
\end{quote}
\begin{quote}\begin{description}
\sphinxlineitem{Returns}
\sphinxAtStartPar
\sphinxstylestrong{contours} \textendash{} A list of pixels that create the contours in the image

\sphinxlineitem{Return type}
\sphinxAtStartPar
list

\end{description}\end{quote}

\end{fulllineitems}

\index{largest\_contour() (in module forensicfit.utils.image\_tools)@\spxentry{largest\_contour()}\spxextra{in module forensicfit.utils.image\_tools}}

\begin{fulllineitems}
\phantomsection\label{\detokenize{forensicfit.utils.image_tools:forensicfit.utils.image_tools.largest_contour}}
\pysigstartsignatures
\pysiglinewithargsret{\sphinxcode{\sphinxupquote{forensicfit.utils.image\_tools.}}\sphinxbfcode{\sphinxupquote{largest\_contour}}}{\sphinxparam{\DUrole{n,n}{contours}}}{}
\pysigstopsignatures
\sphinxAtStartPar
A list of pixels forming the contour with the largest area
\begin{quote}\begin{description}
\sphinxlineitem{Returns}
\sphinxAtStartPar
\sphinxstylestrong{contour\_max\_area} \textendash{} A list of pixels forming the contour with the largest area

\sphinxlineitem{Return type}
\sphinxAtStartPar
list

\end{description}\end{quote}

\end{fulllineitems}

\index{remove\_background() (in module forensicfit.utils.image\_tools)@\spxentry{remove\_background()}\spxextra{in module forensicfit.utils.image\_tools}}

\begin{fulllineitems}
\phantomsection\label{\detokenize{forensicfit.utils.image_tools:forensicfit.utils.image_tools.remove_background}}
\pysigstartsignatures
\pysiglinewithargsret{\sphinxcode{\sphinxupquote{forensicfit.utils.image\_tools.}}\sphinxbfcode{\sphinxupquote{remove\_background}}}{\sphinxparam{\DUrole{n,n}{image}}, \sphinxparam{\DUrole{n,n}{contour}}, \sphinxparam{\DUrole{n,n}{outside}\DUrole{o,o}{=}\DUrole{default_value}{True}}, \sphinxparam{\DUrole{n,n}{pixel\_value}\DUrole{o,o}{=}\DUrole{default_value}{0}}}{}
\pysigstopsignatures
\sphinxAtStartPar
Removes the background outside or inside the contour
\begin{quote}\begin{description}
\sphinxlineitem{Parameters}\begin{itemize}
\item {} 
\sphinxAtStartPar
\sphinxstyleliteralstrong{\sphinxupquote{image}} (\sphinxstyleliteralemphasis{\sphinxupquote{np.array}}) \textendash{} \_description\_

\item {} 
\sphinxAtStartPar
\sphinxstyleliteralstrong{\sphinxupquote{contour}} (\sphinxstyleliteralemphasis{\sphinxupquote{np.array}}) \textendash{} \_description\_

\item {} 
\sphinxAtStartPar
\sphinxstyleliteralstrong{\sphinxupquote{outside}} (\sphinxstyleliteralemphasis{\sphinxupquote{bool}}\sphinxstyleliteralemphasis{\sphinxupquote{, }}\sphinxstyleliteralemphasis{\sphinxupquote{optional}}) \textendash{} \_description\_, by default True

\item {} 
\sphinxAtStartPar
\sphinxstyleliteralstrong{\sphinxupquote{pixel\_value}} (\sphinxstyleliteralemphasis{\sphinxupquote{int}}\sphinxstyleliteralemphasis{\sphinxupquote{, }}\sphinxstyleliteralemphasis{\sphinxupquote{optional}}) \textendash{} \_description\_, by default 0

\end{itemize}

\end{description}\end{quote}

\end{fulllineitems}

\index{get\_masked() (in module forensicfit.utils.image\_tools)@\spxentry{get\_masked()}\spxextra{in module forensicfit.utils.image\_tools}}

\begin{fulllineitems}
\phantomsection\label{\detokenize{forensicfit.utils.image_tools:forensicfit.utils.image_tools.get_masked}}
\pysigstartsignatures
\pysiglinewithargsret{\sphinxcode{\sphinxupquote{forensicfit.utils.image\_tools.}}\sphinxbfcode{\sphinxupquote{get\_masked}}}{\sphinxparam{\DUrole{n,n}{image}}, \sphinxparam{\DUrole{n,n}{mask\_threshold}}}{}
\pysigstopsignatures
\sphinxAtStartPar
Populates the masked image with the gray scale threshold
Returns

\end{fulllineitems}

\index{resize() (in module forensicfit.utils.image\_tools)@\spxentry{resize()}\spxextra{in module forensicfit.utils.image\_tools}}

\begin{fulllineitems}
\phantomsection\label{\detokenize{forensicfit.utils.image_tools:forensicfit.utils.image_tools.resize}}
\pysigstartsignatures
\pysiglinewithargsret{\sphinxcode{\sphinxupquote{forensicfit.utils.image\_tools.}}\sphinxbfcode{\sphinxupquote{resize}}}{\sphinxparam{\DUrole{n,n}{image}}, \sphinxparam{\DUrole{n,n}{size}}}{}
\pysigstopsignatures
\sphinxAtStartPar
This method resizes the image to the pixel size given.
\begin{quote}\begin{description}
\sphinxlineitem{Parameters}
\sphinxAtStartPar
\sphinxstyleliteralstrong{\sphinxupquote{size}} (\sphinxstyleliteralemphasis{\sphinxupquote{tuple int}}\sphinxstyleliteralemphasis{\sphinxupquote{,}}) \textendash{} The target size in which the image is going to be resized.

\sphinxlineitem{Return type}
\sphinxAtStartPar
None.

\end{description}\end{quote}

\end{fulllineitems}

\index{exposure\_control() (in module forensicfit.utils.image\_tools)@\spxentry{exposure\_control()}\spxextra{in module forensicfit.utils.image\_tools}}

\begin{fulllineitems}
\phantomsection\label{\detokenize{forensicfit.utils.image_tools:forensicfit.utils.image_tools.exposure_control}}
\pysigstartsignatures
\pysiglinewithargsret{\sphinxcode{\sphinxupquote{forensicfit.utils.image\_tools.}}\sphinxbfcode{\sphinxupquote{exposure\_control}}}{\sphinxparam{\DUrole{n,n}{image}}, \sphinxparam{\DUrole{n,n}{mode}\DUrole{o,o}{=}\DUrole{default_value}{\textquotesingle{}equalize\_hist\textquotesingle{}}}, \sphinxparam{\DUrole{o,o}{**}\DUrole{n,n}{kwargs}}}{}
\pysigstopsignatures
\sphinxAtStartPar
modifies the exposure
\begin{quote}\begin{description}
\sphinxlineitem{Return type}
\sphinxAtStartPar
\sphinxcode{\sphinxupquote{ndarray}}

\sphinxlineitem{Parameters}
\sphinxAtStartPar
\sphinxstyleliteralstrong{\sphinxupquote{mode}} (\sphinxstyleliteralemphasis{\sphinxupquote{str}}\sphinxstyleliteralemphasis{\sphinxupquote{, }}\sphinxstyleliteralemphasis{\sphinxupquote{optional}}) \textendash{} Type of exposure correction. It can be selected from the options:
\sphinxcode{\sphinxupquote{\textquotesingle{}equalize\_hist\textquotesingle{}}} or \sphinxcode{\sphinxupquote{\textquotesingle{}equalize\_adapthist\textquotesingle{}}}.
\sphinxhref{https://scikit-image.org/docs/stable/api/skimage.exposure.html\#equalize-hist}{equalize\_hist}
and \sphinxtitleref{equalize\_adapthist \textless{}https://scikit\sphinxhyphen{}image.org/docs/stable/api/skimage.exposure.html\#equalize\sphinxhyphen{}adapthist\textgreater{}}
use sk\sphinxhyphen{}image. by default ‘equalize\_hist’

\end{description}\end{quote}

\end{fulllineitems}

\index{apply\_filter() (in module forensicfit.utils.image\_tools)@\spxentry{apply\_filter()}\spxextra{in module forensicfit.utils.image\_tools}}

\begin{fulllineitems}
\phantomsection\label{\detokenize{forensicfit.utils.image_tools:forensicfit.utils.image_tools.apply_filter}}
\pysigstartsignatures
\pysiglinewithargsret{\sphinxcode{\sphinxupquote{forensicfit.utils.image\_tools.}}\sphinxbfcode{\sphinxupquote{apply\_filter}}}{\sphinxparam{\DUrole{n,n}{image}}, \sphinxparam{\DUrole{n,n}{mode}}, \sphinxparam{\DUrole{o,o}{**}\DUrole{n,n}{kwargs}}}{}
\pysigstopsignatures
\sphinxAtStartPar
Applies different types of filters to the image
\begin{quote}\begin{description}
\sphinxlineitem{Return type}
\sphinxAtStartPar
\sphinxcode{\sphinxupquote{ndarray}}

\sphinxlineitem{Parameters}
\sphinxAtStartPar
\sphinxstyleliteralstrong{\sphinxupquote{mode}} (\sphinxstyleliteralemphasis{\sphinxupquote{str}}) \textendash{} Type of filter to be applied. The options are
* \sphinxcode{\sphinxupquote{\textquotesingle{}meijering\textquotesingle{}}}: \textless{}Meijering neuriteness filter \sphinxurl{https://scikit-image.org/docs/stable/api/skimage.filters.html\#skimage.filters.meijering}\textgreater{}\_,
* \sphinxcode{\sphinxupquote{\textquotesingle{}frangi\textquotesingle{}}}: \textless{} Frangi vesselness filter \sphinxurl{https://scikit-image.org/docs/stable/api/skimage.filters.html\#skimage.filters.frangi}\textgreater{}\_,
* \sphinxcode{\sphinxupquote{\textquotesingle{}prewitt\textquotesingle{}}}: \textless{}Prewitt transform \sphinxurl{https://scikit-image.org/docs/stable/api/skimage.filters.html\#prewitt}\textgreater{}\_,
* \sphinxcode{\sphinxupquote{\textquotesingle{}sobel\textquotesingle{}}}: \textless{}Sobel filter \sphinxurl{https://scikit-image.org/docs/stable/api/skimage.filters.html\#skimage.filters.sobel}\textgreater{}\_,
* \sphinxcode{\sphinxupquote{\textquotesingle{}scharr\textquotesingle{}}}: \textless{}Scharr transform \sphinxurl{https://scikit-image.org/docs/stable/api/skimage.filters.html\#skimage.filters.scharr}\textgreater{},
* \sphinxcode{\sphinxupquote{\textquotesingle{}roberts\textquotesingle{}}}: \textless{}Roberts’ Cross operator \sphinxurl{https://scikit-image.org/docs/stable/api/skimage.filters.html\#examples-using-skimage-filters-roberts}\textgreater{}\_,
* \sphinxcode{\sphinxupquote{\textquotesingle{}sato\textquotesingle{}}}: \textless{}Sato tubeness filter \sphinxurl{https://scikit-image.org/docs/stable/api/skimage.filters.html\#skimage.filters.sato}\textgreater{}\_.

\end{description}\end{quote}

\end{fulllineitems}

\index{binerized\_mask() (in module forensicfit.utils.image\_tools)@\spxentry{binerized\_mask()}\spxextra{in module forensicfit.utils.image\_tools}}

\begin{fulllineitems}
\phantomsection\label{\detokenize{forensicfit.utils.image_tools:forensicfit.utils.image_tools.binerized_mask}}
\pysigstartsignatures
\pysiglinewithargsret{\sphinxcode{\sphinxupquote{forensicfit.utils.image\_tools.}}\sphinxbfcode{\sphinxupquote{binerized\_mask}}}{\sphinxparam{\DUrole{n,n}{image}}, \sphinxparam{\DUrole{n,n}{masked}}}{}
\pysigstopsignatures
\sphinxAtStartPar
This function return the binarized version of the tape
\begin{quote}\begin{description}
\sphinxlineitem{Returns}
\sphinxAtStartPar

\sphinxAtStartPar
.


\sphinxlineitem{Return type}
\sphinxAtStartPar
2d array of the image

\end{description}\end{quote}

\end{fulllineitems}

\index{imwrite() (in module forensicfit.utils.image\_tools)@\spxentry{imwrite()}\spxextra{in module forensicfit.utils.image\_tools}}

\begin{fulllineitems}
\phantomsection\label{\detokenize{forensicfit.utils.image_tools:forensicfit.utils.image_tools.imwrite}}
\pysigstartsignatures
\pysiglinewithargsret{\sphinxcode{\sphinxupquote{forensicfit.utils.image\_tools.}}\sphinxbfcode{\sphinxupquote{imwrite}}}{\sphinxparam{\DUrole{n,n}{fname}}, \sphinxparam{\DUrole{n,n}{image}}, \sphinxparam{\DUrole{n,n}{cmap}\DUrole{o,o}{=}\DUrole{default_value}{\textquotesingle{}gray\textquotesingle{}}}, \sphinxparam{\DUrole{o,o}{**}\DUrole{n,n}{kwargs}}}{}
\pysigstopsignatures
\sphinxAtStartPar
save any 2d numpy array (or list) to an image file
\begin{quote}\begin{description}
\sphinxlineitem{Parameters}\begin{itemize}
\item {} 
\sphinxAtStartPar
\sphinxstyleliteralstrong{\sphinxupquote{fname}} (\sphinxstyleliteralemphasis{\sphinxupquote{str}}) \textendash{} flie name to be saved

\item {} 
\sphinxAtStartPar
\sphinxstyleliteralstrong{\sphinxupquote{image}} (\sphinxstyleliteralemphasis{\sphinxupquote{np.array}}) \textendash{} 2d numpy array (or list) to be saved

\end{itemize}

\end{description}\end{quote}

\end{fulllineitems}


\sphinxstepscope


\subparagraph{forensicfit.utils.plotter module}
\label{\detokenize{forensicfit.utils.plotter:module-forensicfit.utils.plotter}}\label{\detokenize{forensicfit.utils.plotter:forensicfit-utils-plotter-module}}\label{\detokenize{forensicfit.utils.plotter::doc}}\index{module@\spxentry{module}!forensicfit.utils.plotter@\spxentry{forensicfit.utils.plotter}}\index{forensicfit.utils.plotter@\spxentry{forensicfit.utils.plotter}!module@\spxentry{module}}\index{get\_figure\_size() (in module forensicfit.utils.plotter)@\spxentry{get\_figure\_size()}\spxextra{in module forensicfit.utils.plotter}}

\begin{fulllineitems}
\phantomsection\label{\detokenize{forensicfit.utils.plotter:forensicfit.utils.plotter.get_figure_size}}
\pysigstartsignatures
\pysiglinewithargsret{\sphinxcode{\sphinxupquote{forensicfit.utils.plotter.}}\sphinxbfcode{\sphinxupquote{get\_figure\_size}}}{\sphinxparam{\DUrole{n,n}{dpi}}, \sphinxparam{\DUrole{n,n}{image\_shape}}, \sphinxparam{\DUrole{n,n}{zoom}\DUrole{o,o}{=}\DUrole{default_value}{4}}, \sphinxparam{\DUrole{n,n}{margin}\DUrole{o,o}{=}\DUrole{default_value}{4}}}{}
\pysigstopsignatures\begin{quote}\begin{description}
\sphinxlineitem{Return type}
\sphinxAtStartPar
\sphinxcode{\sphinxupquote{Tuple}}{[}\sphinxcode{\sphinxupquote{float}}, \sphinxcode{\sphinxupquote{float}}{]}

\end{description}\end{quote}

\end{fulllineitems}

\index{plot\_coordinate\_based() (in module forensicfit.utils.plotter)@\spxentry{plot\_coordinate\_based()}\spxextra{in module forensicfit.utils.plotter}}

\begin{fulllineitems}
\phantomsection\label{\detokenize{forensicfit.utils.plotter:forensicfit.utils.plotter.plot_coordinate_based}}
\pysigstartsignatures
\pysiglinewithargsret{\sphinxcode{\sphinxupquote{forensicfit.utils.plotter.}}\sphinxbfcode{\sphinxupquote{plot\_coordinate\_based}}}{\sphinxparam{\DUrole{n,n}{coordinates}}, \sphinxparam{\DUrole{n,n}{slopes}}, \sphinxparam{\DUrole{n,n}{stds}}, \sphinxparam{\DUrole{n,n}{mode}\DUrole{o,o}{=}\DUrole{default_value}{None}}, \sphinxparam{\DUrole{n,n}{ax}\DUrole{o,o}{=}\DUrole{default_value}{None}}, \sphinxparam{\DUrole{n,n}{plot\_slope}\DUrole{o,o}{=}\DUrole{default_value}{True}}, \sphinxparam{\DUrole{n,n}{plot\_error\_bars}\DUrole{o,o}{=}\DUrole{default_value}{True}}, \sphinxparam{\DUrole{n,n}{plot\_edge}\DUrole{o,o}{=}\DUrole{default_value}{True}}, \sphinxparam{\DUrole{n,n}{show}\DUrole{o,o}{=}\DUrole{default_value}{True}}, \sphinxparam{\DUrole{n,n}{dark\_bg}\DUrole{o,o}{=}\DUrole{default_value}{True}}, \sphinxparam{\DUrole{o,o}{**}\DUrole{n,n}{kwargs}}}{}
\pysigstopsignatures\begin{quote}\begin{description}
\sphinxlineitem{Return type}
\sphinxAtStartPar
\sphinxcode{\sphinxupquote{Axes}}

\end{description}\end{quote}

\end{fulllineitems}

\index{plot\_pair() (in module forensicfit.utils.plotter)@\spxentry{plot\_pair()}\spxextra{in module forensicfit.utils.plotter}}

\begin{fulllineitems}
\phantomsection\label{\detokenize{forensicfit.utils.plotter:forensicfit.utils.plotter.plot_pair}}
\pysigstartsignatures
\pysiglinewithargsret{\sphinxcode{\sphinxupquote{forensicfit.utils.plotter.}}\sphinxbfcode{\sphinxupquote{plot\_pair}}}{\sphinxparam{\DUrole{n,n}{obj\_1}}, \sphinxparam{\DUrole{n,n}{obj\_2}}, \sphinxparam{\DUrole{n,n}{text}\DUrole{o,o}{=}\DUrole{default_value}{None}}, \sphinxparam{\DUrole{n,n}{which}\DUrole{o,o}{=}\DUrole{default_value}{\textquotesingle{}boundary\textquotesingle{}}}, \sphinxparam{\DUrole{n,n}{mode}\DUrole{o,o}{=}\DUrole{default_value}{None}}, \sphinxparam{\DUrole{n,n}{savefig}\DUrole{o,o}{=}\DUrole{default_value}{None}}, \sphinxparam{\DUrole{n,n}{cmap}\DUrole{o,o}{=}\DUrole{default_value}{\textquotesingle{}gray\textquotesingle{}}}, \sphinxparam{\DUrole{n,n}{show}\DUrole{o,o}{=}\DUrole{default_value}{True}}, \sphinxparam{\DUrole{n,n}{figsize}\DUrole{o,o}{=}\DUrole{default_value}{None}}, \sphinxparam{\DUrole{n,n}{labels}\DUrole{o,o}{=}\DUrole{default_value}{None}}, \sphinxparam{\DUrole{n,n}{title}\DUrole{o,o}{=}\DUrole{default_value}{None}}, \sphinxparam{\DUrole{n,n}{zoom}\DUrole{o,o}{=}\DUrole{default_value}{4}}, \sphinxparam{\DUrole{o,o}{**}\DUrole{n,n}{kwargs}}}{}
\pysigstopsignatures\begin{quote}\begin{description}
\sphinxlineitem{Return type}
\sphinxAtStartPar
\sphinxcode{\sphinxupquote{Tuple}}{[}\sphinxcode{\sphinxupquote{Figure}}, \sphinxcode{\sphinxupquote{Axes}}{]}

\end{description}\end{quote}

\end{fulllineitems}

\index{plot\_pairs() (in module forensicfit.utils.plotter)@\spxentry{plot\_pairs()}\spxextra{in module forensicfit.utils.plotter}}

\begin{fulllineitems}
\phantomsection\label{\detokenize{forensicfit.utils.plotter:forensicfit.utils.plotter.plot_pairs}}
\pysigstartsignatures
\pysiglinewithargsret{\sphinxcode{\sphinxupquote{forensicfit.utils.plotter.}}\sphinxbfcode{\sphinxupquote{plot\_pairs}}}{\sphinxparam{\DUrole{n,n}{objs}}, \sphinxparam{\DUrole{n,n}{text}\DUrole{o,o}{=}\DUrole{default_value}{None}}, \sphinxparam{\DUrole{n,n}{which}\DUrole{o,o}{=}\DUrole{default_value}{\textquotesingle{}boundary\textquotesingle{}}}, \sphinxparam{\DUrole{n,n}{mode}\DUrole{o,o}{=}\DUrole{default_value}{None}}, \sphinxparam{\DUrole{n,n}{savefig}\DUrole{o,o}{=}\DUrole{default_value}{None}}, \sphinxparam{\DUrole{n,n}{cmap}\DUrole{o,o}{=}\DUrole{default_value}{\textquotesingle{}gray\textquotesingle{}}}, \sphinxparam{\DUrole{n,n}{show}\DUrole{o,o}{=}\DUrole{default_value}{False}}, \sphinxparam{\DUrole{n,n}{figsize}\DUrole{o,o}{=}\DUrole{default_value}{None}}, \sphinxparam{\DUrole{n,n}{labels}\DUrole{o,o}{=}\DUrole{default_value}{None}}, \sphinxparam{\DUrole{n,n}{title}\DUrole{o,o}{=}\DUrole{default_value}{None}}, \sphinxparam{\DUrole{n,n}{zoom}\DUrole{o,o}{=}\DUrole{default_value}{4}}, \sphinxparam{\DUrole{o,o}{**}\DUrole{n,n}{kwargs}}}{}
\pysigstopsignatures\begin{quote}\begin{description}
\sphinxlineitem{Return type}
\sphinxAtStartPar
\sphinxcode{\sphinxupquote{Tuple}}{[}\sphinxcode{\sphinxupquote{Figure}}, \sphinxcode{\sphinxupquote{Axes}}{]}

\end{description}\end{quote}

\end{fulllineitems}

\index{plot\_confusion\_matrix() (in module forensicfit.utils.plotter)@\spxentry{plot\_confusion\_matrix()}\spxextra{in module forensicfit.utils.plotter}}

\begin{fulllineitems}
\phantomsection\label{\detokenize{forensicfit.utils.plotter:forensicfit.utils.plotter.plot_confusion_matrix}}
\pysigstartsignatures
\pysiglinewithargsret{\sphinxcode{\sphinxupquote{forensicfit.utils.plotter.}}\sphinxbfcode{\sphinxupquote{plot\_confusion\_matrix}}}{\sphinxparam{\DUrole{n,n}{matrix}}, \sphinxparam{\DUrole{n,n}{class\_names}}, \sphinxparam{\DUrole{n,n}{title}\DUrole{o,o}{=}\DUrole{default_value}{\textquotesingle{}Confusion matrix\textquotesingle{}}}, \sphinxparam{\DUrole{n,n}{cmap}\DUrole{o,o}{=}\DUrole{default_value}{\textquotesingle{}Blues\textquotesingle{}}}, \sphinxparam{\DUrole{n,n}{normalize}\DUrole{o,o}{=}\DUrole{default_value}{False}}, \sphinxparam{\DUrole{n,n}{savefig}\DUrole{o,o}{=}\DUrole{default_value}{None}}, \sphinxparam{\DUrole{n,n}{ax}\DUrole{o,o}{=}\DUrole{default_value}{None}}, \sphinxparam{\DUrole{n,n}{show}\DUrole{o,o}{=}\DUrole{default_value}{True}}, \sphinxparam{\DUrole{n,n}{colorbar}\DUrole{o,o}{=}\DUrole{default_value}{True}}}{}
\pysigstopsignatures
\end{fulllineitems}

\index{plot\_kde\_distribution() (in module forensicfit.utils.plotter)@\spxentry{plot\_kde\_distribution()}\spxextra{in module forensicfit.utils.plotter}}

\begin{fulllineitems}
\phantomsection\label{\detokenize{forensicfit.utils.plotter:forensicfit.utils.plotter.plot_kde_distribution}}
\pysigstartsignatures
\pysiglinewithargsret{\sphinxcode{\sphinxupquote{forensicfit.utils.plotter.}}\sphinxbfcode{\sphinxupquote{plot\_kde\_distribution}}}{\sphinxparam{\DUrole{n,n}{distribution}}, \sphinxparam{\DUrole{n,n}{color}\DUrole{o,o}{=}\DUrole{default_value}{\textquotesingle{}blue\textquotesingle{}}}, \sphinxparam{\DUrole{n,n}{opacity}\DUrole{o,o}{=}\DUrole{default_value}{1.0}}, \sphinxparam{\DUrole{n,n}{label}\DUrole{o,o}{=}\DUrole{default_value}{\textquotesingle{}\textquotesingle{}}}, \sphinxparam{\DUrole{n,n}{ax}\DUrole{o,o}{=}\DUrole{default_value}{None}}, \sphinxparam{\DUrole{n,n}{savefig}\DUrole{o,o}{=}\DUrole{default_value}{None}}, \sphinxparam{\DUrole{n,n}{fill\_curve}\DUrole{o,o}{=}\DUrole{default_value}{True}}, \sphinxparam{\DUrole{n,n}{show}\DUrole{o,o}{=}\DUrole{default_value}{True}}}{}
\pysigstopsignatures
\end{fulllineitems}

\index{plot\_hist\_distribution() (in module forensicfit.utils.plotter)@\spxentry{plot\_hist\_distribution()}\spxextra{in module forensicfit.utils.plotter}}

\begin{fulllineitems}
\phantomsection\label{\detokenize{forensicfit.utils.plotter:forensicfit.utils.plotter.plot_hist_distribution}}
\pysigstartsignatures
\pysiglinewithargsret{\sphinxcode{\sphinxupquote{forensicfit.utils.plotter.}}\sphinxbfcode{\sphinxupquote{plot\_hist\_distribution}}}{\sphinxparam{\DUrole{n,n}{distribution}}, \sphinxparam{\DUrole{n,n}{color}\DUrole{o,o}{=}\DUrole{default_value}{\textquotesingle{}blue\textquotesingle{}}}, \sphinxparam{\DUrole{n,n}{opacity}\DUrole{o,o}{=}\DUrole{default_value}{1.0}}, \sphinxparam{\DUrole{n,n}{label}\DUrole{o,o}{=}\DUrole{default_value}{\textquotesingle{}\textquotesingle{}}}, \sphinxparam{\DUrole{n,n}{ax}\DUrole{o,o}{=}\DUrole{default_value}{None}}, \sphinxparam{\DUrole{n,n}{savefig}\DUrole{o,o}{=}\DUrole{default_value}{None}}, \sphinxparam{\DUrole{n,n}{show}\DUrole{o,o}{=}\DUrole{default_value}{True}}}{}
\pysigstopsignatures
\end{fulllineitems}



\subparagraph{Module contents}
\label{\detokenize{forensicfit.utils:module-forensicfit.utils}}\label{\detokenize{forensicfit.utils:module-contents}}\index{module@\spxentry{module}!forensicfit.utils@\spxentry{forensicfit.utils}}\index{forensicfit.utils@\spxentry{forensicfit.utils}!module@\spxentry{module}}

\subsubsection{Submodules}
\label{\detokenize{forensicfit:submodules}}
\sphinxstepscope


\paragraph{forensicfit.version module}
\label{\detokenize{forensicfit.version:module-forensicfit.version}}\label{\detokenize{forensicfit.version:forensicfit-version-module}}\label{\detokenize{forensicfit.version::doc}}\index{module@\spxentry{module}!forensicfit.version@\spxentry{forensicfit.version}}\index{forensicfit.version@\spxentry{forensicfit.version}!module@\spxentry{module}}

\subsubsection{Module contents}
\label{\detokenize{forensicfit:module-forensicfit}}\label{\detokenize{forensicfit:module-contents}}\index{module@\spxentry{module}!forensicfit@\spxentry{forensicfit}}\index{forensicfit@\spxentry{forensicfit}!module@\spxentry{module}}\index{has\_package() (in module forensicfit)@\spxentry{has\_package()}\spxextra{in module forensicfit}}

\begin{fulllineitems}
\phantomsection\label{\detokenize{forensicfit:forensicfit.has_package}}
\pysigstartsignatures
\pysiglinewithargsret{\sphinxcode{\sphinxupquote{forensicfit.}}\sphinxbfcode{\sphinxupquote{has\_package}}}{\sphinxparam{\DUrole{n,n}{name}}}{}
\pysigstopsignatures
\end{fulllineitems}



\chapter{Indices and tables}
\label{\detokenize{index:indices-and-tables}}\begin{itemize}
\item {} 
\sphinxAtStartPar
\DUrole{xref,std,std-ref}{genindex}

\item {} 
\sphinxAtStartPar
\DUrole{xref,std,std-ref}{modindex}

\item {} 
\sphinxAtStartPar
\DUrole{xref,std,std-ref}{search}

\end{itemize}


\renewcommand{\indexname}{Python Module Index}
\begin{sphinxtheindex}
\let\bigletter\sphinxstyleindexlettergroup
\bigletter{f}
\item\relax\sphinxstyleindexentry{forensicfit}\sphinxstyleindexpageref{forensicfit:\detokenize{module-forensicfit}}
\item\relax\sphinxstyleindexentry{forensicfit.core}\sphinxstyleindexpageref{forensicfit.core:\detokenize{module-forensicfit.core}}
\item\relax\sphinxstyleindexentry{forensicfit.core.analyzer}\sphinxstyleindexpageref{forensicfit.core.analyzer:\detokenize{module-forensicfit.core.analyzer}}
\item\relax\sphinxstyleindexentry{forensicfit.core.image}\sphinxstyleindexpageref{forensicfit.core.image:\detokenize{module-forensicfit.core.image}}
\item\relax\sphinxstyleindexentry{forensicfit.core.metadata}\sphinxstyleindexpageref{forensicfit.core.metadata:\detokenize{module-forensicfit.core.metadata}}
\item\relax\sphinxstyleindexentry{forensicfit.core.tape}\sphinxstyleindexpageref{forensicfit.core.tape:\detokenize{module-forensicfit.core.tape}}
\item\relax\sphinxstyleindexentry{forensicfit.database}\sphinxstyleindexpageref{forensicfit.database:\detokenize{module-forensicfit.database}}
\item\relax\sphinxstyleindexentry{forensicfit.database.database}\sphinxstyleindexpageref{forensicfit.database.database:\detokenize{module-forensicfit.database.database}}
\item\relax\sphinxstyleindexentry{forensicfit.utils}\sphinxstyleindexpageref{forensicfit.utils:\detokenize{module-forensicfit.utils}}
\item\relax\sphinxstyleindexentry{forensicfit.utils.array\_tools}\sphinxstyleindexpageref{forensicfit.utils.array_tools:\detokenize{module-forensicfit.utils.array_tools}}
\item\relax\sphinxstyleindexentry{forensicfit.utils.general}\sphinxstyleindexpageref{forensicfit.utils.general:\detokenize{module-forensicfit.utils.general}}
\item\relax\sphinxstyleindexentry{forensicfit.utils.image\_tools}\sphinxstyleindexpageref{forensicfit.utils.image_tools:\detokenize{module-forensicfit.utils.image_tools}}
\item\relax\sphinxstyleindexentry{forensicfit.utils.plotter}\sphinxstyleindexpageref{forensicfit.utils.plotter:\detokenize{module-forensicfit.utils.plotter}}
\item\relax\sphinxstyleindexentry{forensicfit.version}\sphinxstyleindexpageref{forensicfit.version:\detokenize{module-forensicfit.version}}
\end{sphinxtheindex}

\renewcommand{\indexname}{Index}
\printindex
\end{document}